\documentclass[a4paper,
fontsize=11pt,
%headings=small,
oneside,
numbers=noperiodatend,
parskip=half-,
bibliography=totoc,
final
]{scrartcl}

\usepackage[babel]{csquotes}
\usepackage{synttree}
\usepackage{graphicx}
\setkeys{Gin}{width=.4\textwidth} %default pics size

\graphicspath{{./plots/}}
\usepackage[ngerman]{babel}
\usepackage[T1]{fontenc}
%\usepackage{amsmath}
\usepackage[utf8x]{inputenc}
\usepackage [hyphens]{url}
\usepackage{booktabs} 
\usepackage[left=2.4cm,right=2.4cm,top=2.3cm,bottom=2cm,includeheadfoot]{geometry}
\usepackage{eurosym}
\usepackage{multirow}
\usepackage[ngerman]{varioref}
\setcapindent{1em}
\renewcommand{\labelitemi}{--}
\usepackage{paralist}
\usepackage{pdfpages}
\usepackage{lscape}
\usepackage{float}
\usepackage{acronym}
\usepackage{eurosym}
\usepackage{longtable,lscape}
\usepackage{mathpazo}
\usepackage[normalem]{ulem} %emphasize weiterhin kursiv
\usepackage[flushmargin,ragged]{footmisc} % left align footnote
\usepackage{ccicons} 
\setcapindent{0pt} % no indentation in captions

%%%% fancy LIBREAS URL color 
\usepackage{xcolor}
\definecolor{libreas}{RGB}{112,0,0}

\usepackage{listings}

\urlstyle{same}  % don't use monospace font for urls

\usepackage[fleqn]{amsmath}

%adjust fontsize for part

\usepackage{sectsty}
\partfont{\large}

%Das BibTeX-Zeichen mit \BibTeX setzen:
\def\symbol#1{\char #1\relax}
\def\bsl{{\tt\symbol{'134}}}
\def\BibTeX{{\rm B\kern-.05em{\sc i\kern-.025em b}\kern-.08em
    T\kern-.1667em\lower.7ex\hbox{E}\kern-.125emX}}

\usepackage{fancyhdr}
\fancyhf{}
\pagestyle{fancyplain}
\fancyhead[R]{\thepage}

% make sure bookmarks are created eventough sections are not numbered!
% uncommend if sections are numbered (bookmarks created by default)
\makeatletter
\renewcommand\@seccntformat[1]{}
\makeatother

% typo setup
\clubpenalty = 10000
\widowpenalty = 10000
\displaywidowpenalty = 10000

\usepackage{hyperxmp}
\usepackage[colorlinks, linkcolor=black,citecolor=black, urlcolor=libreas,
breaklinks= true,bookmarks=true,bookmarksopen=true]{hyperref}
\usepackage{breakurl}

%meta
\expandafter\def\expandafter\UrlBreaks\expandafter{\UrlBreaks%  save the current one
  \do\a\do\b\do\c\do\d\do\e\do\f\do\g\do\h\do\i\do\j%
  \do\k\do\l\do\m\do\n\do\o\do\p\do\q\do\r\do\s\do\t%
  \do\u\do\v\do\w\do\x\do\y\do\z\do\A\do\B\do\C\do\D%
  \do\E\do\F\do\G\do\H\do\I\do\J\do\K\do\L\do\M\do\N%
  \do\O\do\P\do\Q\do\R\do\S\do\T\do\U\do\V\do\W\do\X%
  \do\Y\do\Z}
%meta

\fancyhead[L]{M. Harbeck\\ %author
LIBREAS. Library Ideas, 40 (2021). % journal, issue, volume.
\href{https://doi.org/10.18452/23806}{\color{black}https://doi.org/10.18452/23806}
{}} % doi 
\fancyhead[R]{\thepage} %page number
\fancyfoot[L] {\ccLogo \ccAttribution\ \href{https://creativecommons.org/licenses/by/4.0/}{\color{black}Creative Commons BY 4.0}}  %licence
\fancyfoot[R] {ISSN: 1860-7950}

\title{\LARGE{Die Ethik des Digitalisierens: Fragen zum Umgang mit Materialien aus kolonialen Kontexten in der Massendigitalisierung}}% title
\author{Matthias Harbeck} % author

\setcounter{page}{1}

\hypersetup{%
      pdftitle={Die Ethik des Digitalisierens: Fragen zum Umgang mit Materialien aus kolonialen Kontexten in der Massendigitalisierung},
      pdfauthor={Matthias Harbeck},
      pdfcopyright={CC BY 4.0 International},
      pdfsubject={LIBREAS. Library Ideas, 40 (2021)},
      pdfkeywords={Bibliothek, Dekolonisierung, Digitalisierung, library, decolonization, digitization},
      pdflicenseurl={https://creativecommons.org/licenses/by/4.0/},
      pdfcontacturl={http://libreas.eu},
      baseurl={https://doi.org/10.18452/23806},
      pdflang={de},
      pdfmetalang={de}
     }



\date{}
\begin{document}

\maketitle
\thispagestyle{fancyplain} 

%abstracts

%body
Digitalisierung wirkt auf den ersten Blick wie der Versuch eines
antikolonialen Projektes, geht sie doch oftmals mit dem Ruf nach einer
breiteren Zugänglichmachung via Open Access einher: Material, das vorher
einer privilegierten Elite vorbehalten war, ist nun grenzüberschreitend
in aller Welt digital und kostenlos verfügbar. Es fallen keine teuren
Reisekosten mehr an, um Unikate in Spezialbibliotheken einzusehen. Die
FAIR-Prinzipien (\emph{Findable, Accessible, Interoperable, Reusable})
unterstreichen diesen Anspruch, der Sichtbarkeit und Verbreitung von
Wissen zu dienen -- viele Infrastruktureinrichtungen orientieren sich
mittlerweile an ihnen. Es bleibt aber zu hinterfragen, ob die digitale
Form wirklich allen einen besseren Zugang ermöglicht: Digitalisierung
setzt eine leistungsfähige digitale Infrastruktur auf
Rezipient*innenseite voraus, damit der Zugriff auf die Ressourcen
tatsächlich vereinfacht wird. Auch muss ein Wissen um die Angebote sowie
die Fähigkeiten diese zu navigieren, zu verstehen (und Sprachbarrieren
zu überwinden) und zu benutzen bei allen potentiellen Nutzenden
vorhanden sein, damit sich die Angebote nicht wieder nur an
privilegierte Gruppen richtet. Um Sprachbarrieren zu überwinden, würde
dies Metadaten (Schlagwörter, Beschreibungen) voraussetzen, die in breit
genutzten Wissenschaftssprachen recherchierbar sind. Und selbst wenn
diese Grundvoraussetzungen einer Nutzung der Digitalisate gegeben wäre,
bleibt bei Materialien aus kolonialen Kontexten\footnote{Dabei sind
  \enquote{koloniale Kontexte} explizit offengehalten und unterliegen
  weder einem Zeitschnitt noch einer regionalen Fokussierung (zum
  Beispiel auf ehemalige deutsche Kolonien). Eine kurze
  Begriffsdefinition, der ich mich anschließe, findet sich bei Ahrndt et
  al.~2018, S. 11--15, dabei insbesondere S. 14f.} die Frage, ob dann
das ihnen innewohnende koloniale Machtgefüge durch die neuen
Zugriffsmöglichkeiten tatsächlich aufgelöst, oder nicht doch eher durch
die Publikation teils rassistischer, aber in jedem Fall durch ihren
Zeitgeist beeinflusster Materialien in anderer, also digitaler Form
reproduziert wird. Damit könnte es zu einer antagonistischen
Gegenüberstellung der FAIR-Prinzipien mit den sogenannten
CARE-Prinzipien (\emph{Collective Benefit, Authority to Control,
Responsibility, Ethics}) kommen, bei der indigene Rechte an den eigenen
Daten (auch denen, die von anderen über sie erhoben wurden) und auch die
Hoheit darüber dem freien Zugang entgegengesetzt werden.\footnote{Vergleiche
  zu den CARE-Prinzipien: Global Indigenous Data Alliance 2021.} Der
folgende Artikel möchte dieses Spannungsverhältnis am Beispiel von
Materialien aus kolonialen Kontexten verdeutlichen und zur Diskussion
stellen, ob und inwieweit ethische Vorbehalte in der
Massendigitalisierung dieser Ressourcen berücksichtigt werden
können/sollten.

Bevor aber tiefer in die Materie der Digitalisierung eingegangen wird,
möchte ich kurz klarstellen, dass im Kontext dieses Artikels
\enquote{Digitalisierung} die Umwandlung oder Konvertierung von
gedruckten/analogen Materialien in eine digitale Form bedeutet. Obwohl
dies Auswirkungen auf das Informationsverhalten in der Gesellschaft im
Allgemeinen hat, werde ich die Frage der Digitalisierung als
gesellschaftliche Revolution auf einer größeren Ebene nicht berühren,
abgesehen von den direkten Auswirkungen, die ehemals analoge Materialien
entwickeln, wenn sie in eine (vermeintlich) leichter zugängliche und
verbreitungsfähige digitale Form umgewandelt werden.

Seit Jahren digitalisieren vor allem Bibliotheken und Archive Texte und
Dokumente, um ihren Nutzenden Informationen digital und damit zeit- und
ortsunabhängig zur Verfügung zu stellen, oft im Zusammenhang mit der
Open-Access/Open-Science-Bewegung oder einfach als Teil der sogenannten
Bestandserhaltung zur Schonung der fragilen Materialien in der Benutzung
(der Schritt der Digitalisierung ist in diesem Zusammenhang eingebettet
in Fragen der Langzeitarchivierung, um wirklich langfristig eine
Bestandserhaltung zu gewährleisten).

Ein frühes prominentes Beispiel war die Digitalisierung von Quellen aus
dem spanischen \emph{Archivo General de Indias} zum Gedenken an
Christoph Kolumbus' Begegnung mit der sogenannten Neuen Welt anlässlich
der 500-Jahr-Feier dieses historischen Moments im Jahr 1992.\footnote{Hohls
  2018.} Angesichts der Digitalisierungsstandards der 2010/20er-Jahre
verfehlen die meisten dieser frühen Digitalisate alle heutigen
Anforderungen und sind für gegenwärtige Recherchierende kaum nutzbar --
sie sind mit einer Auflösung von maximal 100 dpi, in schwarz/weiß
digitalisiert und oftmals ohne flächendeckende Texterkennung erzeugt
worden. Seither wurde die Digitalisierung aus technischer Sicht um ein
Vielfaches verbessert: Es wurden formale Standards etabliert
(Mindestauflösung von 300 dpi, Präsentationsformat PDF,
Archivierungsformat TIFF, Nachweis von Farbschemata und einer
Mindest-Farbbitrate von 24-Bit, Metadatenerfassung und
Metadatenformate).\footnote{DFG-Praxisregeln \enquote{Digitalisierung}
  {[}12/16{]}, 6ff.} Dadurch wurden die Auflösung und die Texterkennung
OCR (\emph{Optical Character Recognition}) deutlich verbessert und
letztere befindet sich stetig in der Weiterentwicklung. Der Vergleich
von damals mit heute zeigt, dass die Voraussetzungen, unter denen
digitalisiert wird, nicht fixiert, sondern im ständigen Wandel begriffen
sind. Dieser Wandel muss nicht auf technische Aspekte beschränkt sein.
Wenn sich ethische \enquote{Standards} ändern, könnten auch diese
Einfluss auf die Parameter zukünftiger Digitalisierungsprojekte haben.

Im Zuge der Open-Access-Bewegung ist die Massendigitalisierung für jede
größere wissenschaftliche Bibliothek und viele Spezialbibliotheken zu
einer Selbstverständlichkeit geworden -- auch für den
Fachinformationsdienst Sozial- und Kulturanthropologie (FID SKA). Der
Fachinformationsdienst digitalisiert ethnologische Werke aus einem
Veröffentlichungszeitraum von über 200 Jahren, um Forschenden mit
(meist) ethnologischem Hintergrund historische und manchmal auch neuere
Materialien zur Verfügung zu stellen, die sonst nicht so leicht
zugänglich sind. Diese Digitalisierung wird von der Deutschen
Forschungsgemeinschaft (DFG) mittlerweile im dritten Projekt und in der
Förderlinie \enquote{Digitalisierung und Erschließung} (Programm
Wissenschaftliche Literaturversorgungs- und Informationssysteme, LIS)
gefördert. Sie hat primär das Ziel, der deutschsprachigen Forschung
Material möglichst einfach zugänglich zu machen. Eine der Auflagen
dieser DFG-Förderung beinhaltet die Forderung, das in solchen Projekten
digitalisierte Material möglichst umfassend (maximal fünf Prozent können
ausgenommen werden) frei im Internet anzubieten.\footnote{DFG 2021, S.
  6.}

Diese Verfahrensweise führt bei dem überwiegend ethnologischen Material,
das im FID SKA digitalisiert wird, zu einigen Fragen:

\begin{enumerate}
\def\labelenumi{\arabic{enumi}.}
\item
  Wenn das Ziel der Digitalisierung ist, dass möglichst freier Zugang zu
  dem Wissen entsteht, wie gewährleisten wir, dass zum Beispiel die
  Länder und Regionen, in denen die Forschungen stattfanden, über die
  Digitalisierung Bescheid wissen und diese nutzen können? Wie können
  sprachliche Barrieren überwunden werden? Wie kann eine
  Informationspolitik aussehen, die über das eigene Land hinausgeht?
\item
  Das Material, das zum Beispiel in kolonialen Kontexten entstanden ist,
  bildet vielfach die Denkmuster, Einstellungen und Wissensordnungen
  seiner Zeit, mit all seinen Rassismen, machtpolitischen Verwerfungen
  und verletzenden Übergriffen ab. Die Transformation in eine digitale
  Form überführt Unrecht beziehungsweise ethisch Fragwürdiges in ein
  noch besser sichtbares Format und macht es -- gerade unter der
  Prämisse des freien Zugriffs und des oftmals vertretenen Anspruchs auf
  Open Access -- einfach multiplizierbar. Die ethischen Implikationen
  einer solchen Transformation werden von den digitalisierenden
  Einrichtungen erst allmählich diskutiert.
\item
  Das auf diese Weise zumindest im sogenannten Globalen Norden meist
  einfacher zugängliche Material ist in der neuen Form leicht
  reproduzierbar, multiplizierbar und aus seinem Kontext entnehmbar.
  Möchte und sollte man als Einrichtung unkontrollierten Zugriff
  gewähren oder haben die Einrichtungen eine ethische und didaktische
  Verantwortung, Präsentation und Zugriff zu regulieren und zu
  kontextualisieren?
\end{enumerate}

Und wie können Ursprungsgesellschaften beziehungsweise der Globale Süden
sowie die Bedürfnisse der Forschung bei diesen Entscheidungen und
Diskussionen einbezogen werden, um hier nicht wieder koloniale
Strukturen der Wissensordnung neu zu beleben? Für den
Fachinformationsdienst Sozial- und Kulturanthropologie ist dies eine
neue Leitfrage, mit der sich das laufende Digitalisierungsprojekt
auseinandersetzen muss.

2013 wurde mit Finanzierung durch die DFG mit der Digitalisierung am
damaligen Sondersammelgebiet Volks- und Völkerkunde, dem
Vorgängerprojekt des FID SKA, begonnen. Der Fokus lag dabei zunächst --
in Absprache mit unserem wissenschaftlichen Beirat -- auf
deutschsprachigen ethnologischen Periodika. Unser Ziel war es, nicht nur
historische Materialien zu digitalisieren, sondern in die nahe
Vergangenheit zu reichen, um den gesamten Zeitraum der ethnologischen
Forschung abzudecken. Durch Vereinbarungen mit Verlagen und der
Verwertungsgesellschaft Wort (VG Wort) gelang es uns sogenannte
\emph{Moving Walls} (von zwei bis fünf Jahren) zu verhandeln und damit
Materialien bis fast in die Gegenwart digitalisieren zu können. Zu den
ältesten Titeln in diesem Bestand gehören die frühen Jahrgänge der
Zeitschriften \emph{Zeitschrift für Ethnologie}\footnote{ZDB-ID:
  201359-9.} (ab 1869) und \emph{Globus : illustrierte Zeitschrift für
Länder- und Völkerkunde}\footnote{ZDB-ID: 217030-9.} (ab 1862). Das
Korpus wurde in einem zweiten Projekt ab 2016 um weitere
Zeitschriftentitel -- vor allem auch aus den Museen -- erweitert, und
Anfang 2021 ist das dritte Projekt gestartet, das diesmal Monographien
von 1800 bis 1920 und DDR-Dissertationen zu ethnologischen Themen
umfasst. Mit allen drei Projekten werden wir nach Abschluss eine
Zeitspanne von 1800 bis in die frühen 2020er-Jahre abdecken. Im dritten
Projekt war mit der Beantragung auch die Frage nach dem Umgang mit
Materialien aus kolonialen Kontexten aufgeworfen.

Da viele dieser digitalisierten Materialien im FID SKA und auch in
anderen Institutionen (Museen, Bibliotheken, Archiven) aus kolonialen
Kontexten stammen, spiegeln die Sprache und der Inhalt tendenziell deren
Zeitgeist wider. Diskriminierende Sprache und entmenschlichende oder
sonstige (aus heutiger Sicht) unethische Darstellungen sind häufig
vertreten. Bei vielen Abbildungen kennen wir die Umstände ihrer
Entstehung schlichtweg nicht. Innerhalb eines Projektes zur
Massendigitalisierung ist eine umfassende Recherche einzelner Personen,
Abbildungen und Forschungskontexte nicht möglich, weswegen diese
Entstehungsumstände bis auf Weiteres unbekannt bleiben, wenn sie nicht
in den Bildunterschriften explizit genannt (und damit in die Metadaten
übernommen) werden. Ein paar Beispiele solcher Materialien illustrieren
unterschiedlich problematische Kontexte, die in den Digitalisaten
auftreten können. Da wir die Bilder nicht unnötig reproduzieren möchten,
verlinke ich auf die Digitalisate bei digi.evifa.de:

Beispiel 1:
\url{https://digi.evifa.de/viewer/image/DE-11-002121697/110/}

\emph{B. (1926): Unter den Ocainas-Indianern. Mit zwei Abbildungen auf
Tafel 16, in: Der Erdball 1, 2, 1926, S. 70--71. Tafel 16 Abb. 1.
Schmuckbemalung der Ocainasweiber. {[}\ldots{]} Abb. 2. Schluß des
großen Tanzes bei den Ocainasmädchen. Erst wenn der Tanz vorbei ist,
mischen sich die Männer unter die jungen Mädchen. Die Tafelseite ist
beschriftet mit dem Hinweis: \enquote{Zu: Domville, Unter Wilden am
Amazones.}}

Beispiel 2: \url{https://digi.evifa.de/viewer/image/BV046142989/45/}

\emph{Boas, Franz (1927): Primitive art. Oslo: H. Aschehoug \& Co.~(W.
Nygaard), S. 45. Plate II. Andaman Islander.}

Beispiel 3: \url{https://digi.evifa.de/viewer/image/BV047081776/52/}

\emph{Frizzi, Ernst (1914): Ein Beitrag zur Ethnologie von Bougainville
und Buka mit spezieller Berücksichtigung der Nasioi. Leipzig und Berlin:
B.G. Teubner, S. 44. Fig. 65. Nasioimann mit Ziernarben an Brust und
Armen.}

Beispiel 4:
\url{https://digi.evifa.de/viewer/image/DE-11-002121697/248/}

\emph{Schlee Pascha (1926): Die Frau im Islam. Mit 8 Abbildungen auf
Tafelseite 33--35, in: Der Erdball 1, 5, Tafel 34. Abb.4. Die Türkin vor
der Zeit der Republik. Abb. 5. Mohammedanische Negerin. Abb. 6. Die
Türkin jetzt.}

Beispiel 5: \url{https://digi.evifa.de/viewer/image/DE-11-002121697/60/}

\emph{Sieber, J. (1926): Vorstellungen der Bantu- und Sudan-Neger über
die Ursachen der Krankheiten. Hierzu Tafelseite VII und VIII mit 6
Abbildungen, in: Der Erdball, 1, 1, S. Tafelseite VIII. Tafelseite VIII
1. Wute-Häuptling Meosi mit einem Teil seiner Familie. (Die kleinen,
aufgeweckten schwarzen Buben wurden auf der Missions-Station
verbunden.). 2. Physiognomien einer Träger-Karawane. (Wute-, Bafia-,
Bati- und Banén-Leute.). 3. Torwächter und Sprechtrommel am Eingang des
Königspalastes in Bamum. 4. Haarschmuck von Tikar-Frauen.
(Hervortretende Kropfbildung)}

In ihrer Präsentation der Anderen als überwiegend Nackte -- zum Teil
gepaart mit Unterschriften wie \enquote{Unter Wilden} oder wie bei Boas'
Werk im Zusammenspiel mit dem Werktitel \enquote{Primitive Art} -- wird
die Assoziation der zivilisatorischen Unterlegenheit geweckt.
Darstellungen biologistischer Körpervermessungen
(\enquote{Physiognomien}), die Subjekte zu Objekten degradieren und die
Etablierung (oder Reaktivierung) von Stereotypen (wie bei der
islamischen Frau durch feste Typen, die vorgestellt werden) sind
weitere, wiederkehrende Elemente bei diesen bildlichen Darstellungen.
Auch die Frage, wie freiwillig die jeweilige Fotografie in Kauf genommen
oder vielleicht sogar befürwortet wurde, lässt sich anhand des
Bildmaterials selten erschließen.\footnote{Zur Freiwilligkeit siehe die
  Diskussion bei Harbeck und Strickert 2020.} Für sich genommen ist das
Bildmaterial oft nicht verständlich, im Kontext des freien Zugangs im
Internet wird dies problematisch, da die Forschungszusammenhänge --
gerade bei Tafelseiten -- oftmals für Lai*innen nicht ersichtlich sind.

Verbunden mit den allgemeinen Fragen der freien Bereitstellung dieser
Ressourcen auf unseren Servern sind Fragen der Nutzung dieser
Materialien für die Öffentlichkeitsarbeit und in der Informationspolitik
zu den Materialien: Für unsere Webseite
\href{http://www.evifa.de}{www.evifa.de} haben wir
Public-Domain-Abbildungen aus unserem eigenen Digitalisierungsprojekt
zur Illustration der Seite verwendet.

Zunächst wurden dabei auch Bilder verwendet, auf denen Menschen
abgebildet sind. Im Zuge der internen Diskussion um den Gebrauch und
Kontext der Bilder wurden fast alle ersetzt, da die Umstände der
Herstellung dieser Motive unklar waren. Gab es eine Einwilligung? Wurde
die Einwilligung freiwillig gegeben?

Ein weiteres Problem ergibt sich aus der Werbung für unsere
Digitalisierungsbemühungen über Twitter: Über den RSS-Feed unseres
Digitalisierungsservers spiegeln wir automatisch die Titelseiten unserer
Digitalisate auf unserem Twitter-Account @fid\_evifa
(\url{https://twitter.com/fid_evifa}).

Beispiele wären hier die Coverabbildungen von zwei Werken Ludwig
Wilsers: \emph{Rassen und Völker}. Leipzig {[}1912?{]} und \emph{Das
Hakenkreuz nach Ursprung, Vorkommen und Bedeutung}. Zeitz 1917.

Diese Beispiele lösten eine interne Diskussion über das
Spannungsverhältnis von Mehrwerten der Informationspolitik gegenüber
ethischen Vorbehalten aus -- die Debatte soll mit den Zielgruppen des
FID SKA fortgeführt und eine stringente Strategie entwickelt werden.
Denn die Umwandlung dieser Materialien von physischen Gegenständen in
ein digitales, offen zugängliches Format erweitert die möglichen
Verbreitungswege, ihre potenzielle Reichweite und die Chancen, aus dem
Zusammenhang gerissen zu werden. Einzelbilder und Cover per Social Media
zu verbreiten, kann diese Entkontextualisierung noch verstärken.
Generell kann jede*r aber auch aus den Digitalisaten selbst die meist
frei zugänglichen Bilder (oder Textzitate) verwenden und sie aus dem
Kontext reißen, vielleicht sogar aus unethischen Gründen. Auf jedem
dieser Bilder könnten tabuisierte Rituale und Gegenstände abgebildet
sein.\footnote{Vergleiche zu dem Komplex der Schwierigkeit der Abbildung
  solcher Materialien den Sammelband \emph{Heikles Erbe} von von Poser
  und Baumann 2016.} Ohne einen klaren Hinweis in den Metadaten oder auf
den Bildern selbst würde es einen erheblichen Zeit- und Rechercheaufwand
erfordern, solche schwierigen Darstellungen zu identifizieren.

Viele Digitalisierungsprojekte in den Geisteswissenschaften stehen vor
der Frage, ob es verantwortungsvoll ist, alles in eine digitale Form zu
überführen UND diese digitalen Objekte offen und barrierefrei im
Internet anzubieten. Seit ungefähr zwei Jahren wird diese Frage auch
stärker öffentlich diskutiert. Für den FID SKA hatte dies die
Konsequenz, dass wir mit dem letzten Digitalisierungsantrag bei der DFG
(das Projekt begann im Januar 2021, wurde aber im Januar 2020
eingereicht und 2019 konzipiert) diese Frage aufgriffen. Die meisten
Massendigitalisierungsprojekte, die auf Drittmittel angewiesen sind,
werden nur finanziert, wenn fast alles offen zugänglich ist, fast keines
dieser Projekte hat Forschungsteile integriert. Das wichtigste
Förderprogramm in diesem Zusammenhang -- \enquote{Digitalisierung und
Erschließung} der DFG -- umfasst in der Regel keine Forschungsteile und
keinen eingeschränkten Zugang, Offenheit ist hier die Maxime --
schließlich wird es mit Steuergeldern finanziert. Ein Dialog mit
Förderprogrammen über alternative Lösungen muss erst noch initiiert
werden. Immerhin konnten zwei kontextualisierende Workshops in den
Förderantrag integriert werden und es ist als Erfolg zu verbuchen, dass
die Gremien der Deutschen Forschungsgemeinschaft dem Vorhaben zustimmen,
diese Fragen zu diskutieren.

Der Fachinformationsdienst Sozial- und Kulturanthropologie hat das Thema
im Panel \enquote{Everything open for everyone? How Open Science is
challenging and expanding ethnographic research practices} beim 15.
Kongress der Internationalen Gesellschaft für Ethnologie und Folklore
(SIEF) mit einem Vortrag zu \enquote{Access or ethics? Digitization of
imperial times materials and its consequences} eingebracht und hat es in
zwei eigenen Workshops am 8. Juli 2021 -- einer mit Forschenden und
einer mit Bibliothekar*innen und Digitalisierenden -- wieder
aufgenommen. Dies sollen erste Schritte sein, um auf verschiedene
Communitys -- auch die Forschenden in den Ursprungsgesellschaften --
zuzugehen und eine Policy im Umgang mit diesen ethischen Problemfeldern
im Digitalisierungsprozess zu entwickeln. Die Positionen können dabei
sehr unterschiedlich sein und es wird spannend zu sehen sein, wie diese
übereingebracht werden können: Im Rahmen der Workshops gab es recht
unterschiedliche Stimmen, auch aus den so genannten
Herkunftsgesellschaften. Einerseits wurde neben einer angemessenen
Kontextualisierung angemahnt, die Materialien umfassend in den Regionen
bekannt zu machen. Dieses Anliegen wurde mit der Frage verbunden, wie
dieses Material für jene auffind- und nutzbar gemacht werden könnte, die
kein Deutsch lesen können. Andererseits wurde aber auch auf Inhalte
verwiesen, die durch ihre bloße Sichtbarkeit im Netz lokale
Verantwortliche in Erklärungsnöte und soziale Bedrängnis brächten.
Gemeinsame Projekte zur inhaltlichen Erschließung mittels verknüpfter,
kontrollierter Vokabulare und gemeinsame Sichtungen des Materials im
Vorwege einer Digitalisierung wären hier mögliche Wege. Die europäischen
Forschenden, die am SIEF-Panel teilnahmen, sahen die Offenheit
wesentlich kritischer und plädierten eher für eine restriktivere
Digitalisierungspolitik. Einen Ausgleich dieser Positionen zu finden,
ist eine anstehende Aufgabe des FID SKA. Und bestenfalls ergibt sich aus
diesen Aktivitäten auch ein Dialog mit den Fördereinrichtungen über neue
Wege der Digitalisierung, die ethische Fragen und Standards diskutieren
und berücksichtigen.

\hypertarget{quellen}{%
\subsubsection{Quellen}\label{quellen}}

B. (1926): Unter den Ocainas-Indianern. Mit zwei Abbildungen auf Tafel
16, in: Der Erdball 1, 2, 1926, S. 70--71.
\url{https://digi.evifa.de/viewer/image/DE-11-002121697/90/} (Werks-URN:
\url{https://digi.evifa.de/viewer/resolver?urn=urn:nbn:de:kobv:11-d-4738180}).

Boas, Franz: Primitive art. Oslo: H. Aschehoug \& Co.~(W. Nygaard) 1927.
\url{https://digi.evifa.de/viewer/image/BV046142989/5/} (Werks-URN:
\url{https://digi.evifa.de/viewer/resolver?urn=urn:nbn:de:kobv:11-d-6493742}).

Frizzi, Ernst: Ein Beitrag zur Ethnologie von Bougainville und Buka mit
spezieller Berücksichtigung der Nasioi. Leipzig und Berlin: B.G.
Teubner, 1914. \url{https://digi.evifa.de/viewer/image/BV047081776/7/}

Schlee Pascha: Die Frau im Islam. Mit 8 Abbildungen auf Tafelseite
33--35, in: Der Erdball 1, 5, 1926, S. 161--164.
\url{https://digi.evifa.de/viewer/image/DE-11-002121697/207/}
(Werks-URN:
\url{https://digi.evifa.de/viewer/resolver?urn=urn:nbn:de:kobv:11-d-6648200}).

Sieber, J.: Vorstellungen der Bantu- und Sudan-Neger über die Ursachen
der Krankheiten. Hierzu Tafelseite VII und VIII mit 6 Abbildungen, in:
Der Erdball, 1, 1, 1926, S. 32--36.
\url{https://digi.evifa.de/viewer/image/DE-11-002121697/42/} (Werks-URN:
\url{https://digi.evifa.de/viewer/resolver?urn=urn:nbn:de:kobv:11-d-4738180}).

Wilser, Ludwig: Rassen und Völker. Leipzig {[}1912?{]}

Wilser, Ludwig: Das Hakenkreuz nach Ursprung, Vorkommen und Bedeutung.
Zeitz 1917.

\hypertarget{literaturverzeichnis}{%
\subsubsection{Literaturverzeichnis}\label{literaturverzeichnis}}

Ahrndt, Wiebke; Czech, Hans-Jörg; Fine, Jonathan; Förster, Larissa;
Geißdorf, Michael; Glaubrecht, Matthias et al.~(Hg.) (2018): Leitfaden
zum Umgang mit Sammlungsgut aus kolonialen Kontexten. Deutscher
Museumsbund. Berlin: Deutscher Museumsbund e. V. Online verfügbar unter
\url{https://www.museumsbund.de/wp-content/uploads/2018/05/dmb-leitfaden-kolonialismus.pdf},
zuletzt geprüft am 28.06.2021.

DFG (2021): Merkblatt und ergänzender Leitfaden. Digitalisierung und
Erschließung. DFG-Vor-druck 12.15, auf:
\url{http://www.dfg.de/formulare/12_15/12_15_de.pdf}, zuletzt geprüft am
24.06.2021.

DFG-Praxisregeln \enquote{Digitalisierung} {[}12/16{]}. Online verfügbar
unter \url{http://www.dfg.de/formulare/12_151/12_151_de.pdf}, zuletzt
geprüft am 24.06.2021.

Global Indigenous Data Alliance (2021): CARE Principles of Indigenous
Data Governance --- Global Indigenous Data Alliance. Online verfügbar
unter \url{https://www.gida-global.org/care}, zuletzt aktualisiert am
24.06.2021, zuletzt geprüft am 24.06.2021.

Harbeck, Matthias; Strickert, Moritz (2020): Freiwilligkeit und Zwang.
Eine Diskussion im Kontext der frühen ethnologischen Fotografie. Visual
History. Online verfügbar unter
\url{https://visual-history.de/2020/09/28/freiwilligkeit-und-zwang/},
zuletzt geprüft am 28.06.2021.

Hohls, Rüdiger (2018): Digital Humanities und digitale
Geschichtswissenschaften. Clio-online. Online verfügbar unter
\url{https://guides.clio-online.de/guides/arbeitsformen-und-techniken/digital-humanities/2018},
zuletzt aktualisiert am 26.05.2021, zuletzt geprüft am 17.06.2021.

Poser, Alexis von; Baumann, Bianca (Hg.) (2016): Heikles Erbe. Koloniale
Spuren bis in die Gegenwart. Niedersächsisches Landesmuseum Hannover.
Dresden: Sandstein Verlag. Online verfügbar unter
\url{http://verlag.sandstein.de/reader/98-250_HeiklesErbe}, zuletzt
geprüft am 08.05.2019.

%autor
\begin{center}\rule{0.5\linewidth}{0.5pt}\end{center}

\textbf{Matthias Harbeck} studierte Geschichte, Politik und Ethnologie
in Hamburg und Leiden sowie Bibliotheks- und Informationswissenschaften
in Berlin. Seit 2009 ist er Fachreferent an der Universitätsbibliothek
der Humboldt-Universität zu Berlin. Seit 2016 leitet er den von der
Deutschen Forschungsgemeinschaft (DFG) geförderten
Fachinformationsdienst Sozial- und Kulturanthropologie.

\end{document}
