\textbf{Kurzfassung}: \textit{Zielsetzung} -- Der Artikel fasst die Argumentation
meiner Dissertation zusammen. Darin wird untersucht, wie die Arbeit mit
wissenschaftlichen Publikationen in Bibliotheken und bei
Informationsdienstleistern daran mitwirkt, Beiträge zum
Wissenschaftssystem aus dem \enquote{Globalen Norden} zuprivilegieren. Die
dadurch reproduzierte soziale Ungerechtigkeit kann als \enquote{Kolonialität}
bezeichnet werden. Schließlich stellt der Artikel Optionen vor, wie
Bibliotheken ihre eigene Dekolonialisierung in die Wege leiten können.

\textit{Forschungsmethoden} -- Die Zusammenfassung ist rein argumentativ und
bezieht sich nur implizitauf die vielfältigen empirischen Studien der
Dissertation.

\textit{Ergebnisse} -- Die Durchlässigkeit für wissenschaftliche
Kommunikationsmedien aus dem \enquote{Globalen Süden} wird als zentrale Weiche
für Kolonialität erkannt, insbesondere die Abtrennung der
Regionalwissenschaften von den Kerndisziplinen, die Inklusionskriterien
von bibliographischen Datenbanken und Paketprodukten von
Informationsdienstleistern, Szientometrie und die damit verbundene
Quantifizierung von wissenschaftlicher Kommunikation, die
Kommerzialisierung von Open Access und die Orientierung von
Bestandsentwicklung am \enquote{Bedarf}.

\textit{Schlussfolgerungen} -- Neutralität als einer der Kerne bibliothekarischer
Berufsethik sollte in Europa als kulturell demütige Neutralität
rekonzeptualisiert werden, um Kolonialität entgegen zu treten, die
eigene privilegierte Position zu reflektieren und einen erhöhten Aufwand
zu akzeptieren. Dies führt zwar zu mehr Komplexität im
Wissenschaftssystem, bietet aber die Chance auf produktive Irritation
und Anreize für Kooperation.

\begin{center}\rule{0.5\linewidth}{0.5pt}\end{center}

\textbf{Abstract}: \textit{Objective} -- The article summarises the argumentation
of my PhD thesis that investigates how the work with scholarly
publications in libraries and by information service providers
privileges contributions to the research system from the \enquote{Global
North}. The thereby reproduced social injustice can be termed
\enquote{coloniality}. The article presents options for libraries to initiate
their own decolonization.

\textit{Methods} -- The summary is solely argumentative and refers to the diverse
empirical studies of the PhD thesis only implicitly.

\textit{Results} -- The permeability for scholarly communication media from the
\enquote{Global South} is recognised as an important passage for coloniality,
in particular the separation of area studies from the core disciplines,
the inclusion criteria of bibliographic databases and package products
from information service providers, scientometrics and the associated
quantification of scholarly communication, the commercialization of Open
Access and the focus of collection development on \enquote{demand}.

\textit{Conclusions} -- Neutrality as one of the core values of the librarian`s
professional ethics should be re-conceptualized in Europe as culturally
humble neutrality, in order to counter coloniality, reflect on one's own
privileged position and accept increased effort. Although this leads to
more complexity in the research system, it offers the opportunity for
productive irritation and incentives for cooperation.
