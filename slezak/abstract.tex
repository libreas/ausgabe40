Bibliotheken verwalten Wissen. Bibliotheken sind Räume für
Menschen. Was passiert, wenn all das auf kolonialer Vergangenheit
aufbaut? Die C3-Bibliothek für Entwicklungspolitik in Wien hat 2018
begonnen, sich aktiv mit der eigenen Geschichte rassistischer Praktiken
auseinanderzusetzen. Wir verstehen Dekolonialisierung von Bibliotheken
als Prozess, der aktiv gestaltet werden muss. Dieser braucht einen
Kompetenzaufbau für die inhaltliche Auseinandersetzung mit Rassismen in
überwiegend weißen Institutionen durch Erwerben von theoretischem Wissen
und kritische, selbstreflexive Auseinandersetzung mit rassistischen
Denkweisen wie auch eine diskursive und partizipative Ausgestaltung mit
breiter Einbindung der Akteur*innen. Damit Letzteres gelingen kann gilt
es auf der Ebene der Nutzer*innen und Bürger*innen eine direkte und
offene Thematisierung zu initiieren, die aktive Teilhabe und Engagement
aller an der Debatte fördert. Für die C3-Bibiothek ist dafür die
Vermittlung von Informationskompetenz als transformatorisches Werkzeug
zentral. Sie ermöglicht ein Engagement, das sich manipulativen,
entmenschlichenden und kolonisierenden Prinzipien widersetzen kann. Der
Beitrag stellt erste Erfahrungen aus der Praxis und Reflexionsergebnisse
dem Austausch mit gleichgesinnten Bibliotheken im deutschsprachigen Raum
zur Verfügung.
