\begin{center}\rule{0.5\linewidth}{0.5pt}\end{center}

\textbf{Gabriele Slezak} studierte Afrikawissenschaften mit Schwerpunkt
Soziolinguistik (Dr.in) an der Universität Wien, der Université de
Ouagadougou und der Universität Bayreuth. Seit 1997 lehrt und forscht
sie an der Universität Wien mit einem Schwerpunkt im Bereich
Mehrsprachigkeit in institutionellen Kontexten, Bildungssysteme und
transdisziplinäre Forschung in Westafrika. Seit 1998 arbeitet sie in der
ÖFSE, aktuell als Leitung der Öffentlichkeitsarbeit und
Wissenschaftskommunikation.

\textbf{Sarah Schmelzer} studierte Slawistik und Literaturwissenschaft
(Mag.a) und absolvierte ein postgraduales Studium Bibliotheks- und
Informationsmanagement an der Donau Universität Krems (MSc). Nach
verschiedenen beruflichen Stationen, darunter die Leitung der Bibliothek
des Goethe-Instituts in St.~Petersburg, verantwortet sie seit 2012 den
Bereich Bibliothek in der ÖFSE.

\textbf{Andrea Ruscher} studierte Geschichte (BA) und Global Studies
(MA) an der Universität Wien und arbeitet seit April 2019 im Bereich
Bibliothek der ÖFSE in der C3-Bibliothek für Entwicklungspolitik. In
Rahmen dieser Tätigkeit bearbeitete sie ein historisches
Bestandssegments exemplarisch und untersuchte darin präsente Rassismen
im historischen Kontext.

\textbf{Erem Celebi} studiert zurzeit Molekulare Biotechnologie (BSc) am
FH-Campus Wien. Nachdem er den C3-Award im Jahr 2019 für seine VwA zum
Thema \enquote{Die Frau als Ware -- Frauenhandel im 21. Jahrhundert} gewann,
konzipierte er im Rahmen eines C3-Projekts ein Modul zum Thema
\enquote{Rassismen in Österreich}.

\textbf{Dani Baumgartner} studierte Soziologie (BA) und Gender Studies
und absolvierte 2017/2018 den ULG Library and Information Studies an der
ÖNB. Seit 2016 arbeitet Dani Baumgartner für die Frauen*solidarität in
der C3-Bibliothek für Entwicklungspolitik.

\textbf{Jonas Paintner} studiert(e) Publizistik (Bakk.phil.) sowie
Internationale Entwicklung in Wien und Global Political Economy (MA) in
London. Nach Tätigkeit in der Wissenschaftskommunikation der ÖFSE,
freiberuflicher rassismuskritischer Bildungsarbeit und einem Lehrauftrag
zu Migrationspädagogik an der Universität Krems forscht er derzeit im
Postgraduiertenprogramm des Deutschen Instituts für Entwicklungspolitik
(DIE) zu Modalitäten transnationaler Wissenskooperationen.
