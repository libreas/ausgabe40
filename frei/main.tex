\documentclass[a4paper,
fontsize=11pt,
%headings=small,
oneside,
numbers=noperiodatend,
parskip=half-,
bibliography=totoc,
final
]{scrartcl}

\usepackage[babel]{csquotes}
\usepackage{synttree}
\usepackage{graphicx}
\setkeys{Gin}{width=.4\textwidth} %default pics size

\graphicspath{{./plots/}}
\usepackage[ngerman]{babel}
\usepackage[T1]{fontenc}
%\usepackage{amsmath}
\usepackage[utf8x]{inputenc}
\usepackage [hyphens]{url}
\usepackage{booktabs} 
\usepackage[left=2.4cm,right=2.4cm,top=2.3cm,bottom=2cm,includeheadfoot]{geometry}
\usepackage{eurosym}
\usepackage{multirow}
\usepackage[ngerman]{varioref}
\setcapindent{1em}
\renewcommand{\labelitemi}{--}
\usepackage{paralist}
\usepackage{pdfpages}
\usepackage{lscape}
\usepackage{float}
\usepackage{acronym}
\usepackage{eurosym}
\usepackage{longtable,lscape}
\usepackage{mathpazo}
\usepackage[normalem]{ulem} %emphasize weiterhin kursiv
\usepackage[flushmargin,ragged]{footmisc} % left align footnote
\usepackage{ccicons} 
\setcapindent{0pt} % no indentation in captions

%%%% fancy LIBREAS URL color 
\usepackage{xcolor}
\definecolor{libreas}{RGB}{112,0,0}

\usepackage{listings}

\urlstyle{same}  % don't use monospace font for urls

\usepackage[fleqn]{amsmath}

%adjust fontsize for part

\usepackage{sectsty}
\partfont{\large}

%Das BibTeX-Zeichen mit \BibTeX setzen:
\def\symbol#1{\char #1\relax}
\def\bsl{{\tt\symbol{'134}}}
\def\BibTeX{{\rm B\kern-.05em{\sc i\kern-.025em b}\kern-.08em
    T\kern-.1667em\lower.7ex\hbox{E}\kern-.125emX}}

\usepackage{fancyhdr}
\fancyhf{}
\pagestyle{fancyplain}
\fancyhead[R]{\thepage}

% make sure bookmarks are created eventough sections are not numbered!
% uncommend if sections are numbered (bookmarks created by default)
\makeatletter
\renewcommand\@seccntformat[1]{}
\makeatother

% typo setup
\clubpenalty = 10000
\widowpenalty = 10000
\displaywidowpenalty = 10000

\usepackage{hyperxmp}
\usepackage[colorlinks, linkcolor=black,citecolor=black, urlcolor=libreas,
breaklinks= true,bookmarks=true,bookmarksopen=true]{hyperref}
\usepackage{breakurl}

%meta
\expandafter\def\expandafter\UrlBreaks\expandafter{\UrlBreaks%  save the current one
  \do\a\do\b\do\c\do\d\do\e\do\f\do\g\do\h\do\i\do\j%
  \do\k\do\l\do\m\do\n\do\o\do\p\do\q\do\r\do\s\do\t%
  \do\u\do\v\do\w\do\x\do\y\do\z\do\A\do\B\do\C\do\D%
  \do\E\do\F\do\G\do\H\do\I\do\J\do\K\do\L\do\M\do\N%
  \do\O\do\P\do\Q\do\R\do\S\do\T\do\U\do\V\do\W\do\X%
  \do\Y\do\Z}
%meta

\fancyhead[L]{E. Frei\\ %author
LIBREAS. Library Ideas, 40 (2021). % journal, issue, volume.
\href{https://doi.org/10.18452/23804}{\color{black}https://doi.org/10.18452/23804}
{}} % doi 
\fancyhead[R]{\thepage} %page number
\fancyfoot[L] {\ccLogo \ccAttribution\ \href{https://creativecommons.org/licenses/by/4.0/}{\color{black}Creative Commons BY 4.0}}  %licence
\fancyfoot[R] {ISSN: 1860-7950}

\title{\LARGE{Wenn aus \enquote{Menschenfressern} Affen werden: Rassismen in Bibliotheken am Beispiel von \enquote{Hatschi Bratschis Luftballon}}}% title
\author{Elisa Frei} % author

\setcounter{page}{1}

\hypersetup{%
      pdftitle={Wenn aus "Menschenfressern" Affen werden: Rassismen in Bibliotheken am Beispiel von "Hatschi Bratschis Luftballon"},
      pdfauthor={Elisa Frei},
      pdfcopyright={CC BY 4.0 International},
      pdfsubject={LIBREAS. Library Ideas, 40 (2021)},
      pdfkeywords={Bibliothek, Dekolonisierung, library, decolonization},
      pdflicenseurl={https://creativecommons.org/licenses/by/4.0/},
      pdfcontacturl={http://libreas.eu},
      baseurl={https://doi.org/10.18452/23804},
      pdflang={de},
      pdfmetalang={de}
     }


\date{}
\begin{document}

\maketitle
\thispagestyle{fancyplain} 

%abstracts
\begin{abstract}
\noindent
\textbf{Kurzfassung:} Dieser Artikel thematisiert den Umgang von Bibliotheken mit
Medien, deren Inhalte Rassismen enthalten, exemplarisch anhand des
rassistischen Kinderbuches \enquote{Hatschi Bratschis Luftballon} von Franz
Karl Ginzkey in den Büchereien Wien und der Fachbereichsbibliothek für
Germanistik, Nederlandistik und Skandinavistik. Die Grundlage der
Untersuchung bilden zwei Expert:inneninterviews, die mit Vertreter:innen
der beiden Institutionen geführt wurden. Ziel des Artikels ist es, die
unterschiedlichen Argumentationen der beiden Bibliotheken für die
Aufnahme des Kinderbuchs in den jeweiligen Bestand darzustellen. Die
Gegenüberstellung der Argumentationen beider Bibliotheksarten -- bei den
Büchereien Wien handelt es sich um eine öffentliche und bei der
Fachbereichsbibliothek um eine wissenschaftliche Bibliothek -- zeigen,
dass diese aufgrund von Bestandspolitik und Zielpublikum divergieren,
zum Teil aber mit ähnlichen Fragen und Herausforderungen konfrontiert
sind.

\begin{center}\rule{0.5\linewidth}{0.5pt}\end{center}

\noindent \textbf{Abstract:} In this article the handling of libraries with information
sources that contain racism is examined by the racist children's book
\enquote{Hatschi Bratschis Luftballon} by Franz Karl Ginzkey in the Büchereien
Wien and the Fachbereichsbibliothek für Germanistik, Nederlandistik und
Skandinavistik. Two expert interviews which were conducted with
representatives of both institutions provide the basis for this
analysis. The purpose of this article is to describe the different
argumentations of the libraries to include the children's book into
their library collections. The comparison of the arguments of both types
of libraries -- the Büchereien Wien are public and the
Fachbereichsbibliothek is a scientific library -- show that they diverge
due to inventory policy and their target audience, however, they are
confronted with similar questions and challenges.
\end{abstract}

%body
\hypertarget{einleitung}{%
\section{Einleitung}\label{einleitung}}

Im Oktober 2019 wurde das \emph{Phil} -- ein Wiener Café, in dem sich
auch eine Buchhandlung befindet -- in österreichischen Medien sowie über
Social Media (Instagram, Facebook) dafür kritisiert, das Faksimile des
Originals von \enquote{Hatschi Bratschis Luftballon} von Franz Karl Ginzkey zu
verkaufen. Das Buch war zuvor vom \emph{Ibera} Verlag neu herausgegeben
worden. Anlässlich der medialen Aufmerksamkeit, die das Kinderbuch
erhielt, wurde über den inhärenten Rassismus dieses Werkes debattiert
(vergleiche Der Standard 2019; Matzinger 2019; ZARA 2019).

Die Kritiker:innen warfen der Buchhandlung vor, dass der Inhalt des
Kinderbuches auf (kolonial-)rassistischen Stereotypen basiere und dessen
Verkauf zur Reproduktion von Rassismen beitrage (Der Standard 2019).
Mariella Leydolt, Buchhändlerin im \emph{Phil}, wies als Antwort auf die
Kritik darauf hin, dass sich das Buch nicht bei den Kinderbüchern,
sondern in der \enquote{Austriaca Abteilung} befinde, da es als
\enquote{Zeitdokument} gefasst wird. Sie betonte den \enquote{kunsthistorischen}
Wert der im \emph{Phil} erhältlichen Ausgabe, da es sich dabei um einen
Faksimile-Druck des Originals aus dem Jahr 1904 \enquote{mit der alten Schrift
und den alten Bildern} handle. Der rassistische Inhalt sei ihr bewusst,
sie gehe allerdings davon aus, dass die Käufer:innen dazu in der Lage
seien, das Buch zu kontextualisieren (Matzinger 2019). Die Buchhandlung
selbst würde eine entsprechende Rahmung bieten: \enquote{Letztlich ist das Phil
der Kommentar zum Buch: dieser liberale Raum, unsere Grundhaltung,
unsere anderen Bücher} (Leydolt zitiert nach Matzinger 2019).

\enquote{Hatschi Bratschis Luftballon} ist in unterschiedlichen Versionen
nicht nur in anderen Buchhandlungen oder im Onlineversandhandel käuflich
zu erwerben (Der Standard 2019a), sondern kann auch in österreichischen
Bibliotheken entliehen werden. Die Frage, ob
Bibliotheksmitarbeiter:innen ähnliche Argumentationen für das
Vorhandensein des Kinderbuches in den Bibliotheksbeständen nutzen, wie
die erwähnte Buchhändlerin, bildet das Ausgangsinteresse dieses
Artikels.

In Anlehnung an Publikationen aus den USA und Großbritannien werden
Bibliotheken als Stützen \emph{weißer}\footnote{Der Begriff \enquote{weiß}
  wird klein und kursiv geschrieben, \enquote{um die üblicherweise unmarkierte
  weiße Position und das Machtverhältnis} (Attia 2016: 230;
  Hervorhebungen im Original) zwischen der \emph{weißen}
  Mehrheitsgesellschaft und all jenen, die aus dieser exkludiert und als
  `anders' konstruiert werden, hervorzuheben.} Normativität gefasst
(vergleiche Chou / Pho 2018; Schlessman-Tarango 2017; Muddiman et.
al.~2000). Damit ist gemeint, dass Bibliotheken Wissensarchive
darstellen, die Teil eines Systems sind, in dem manche privilegierter
sind als andere. Das theoretische Fundament dieses Artikels bilden
Ansätze der postkolonialen Theorie, die davon ausgeht, dass die
Gegenwart von kolonialen Denkweisen durchdrungen ist. Ein zentrales
Element dieses kolonialen Denkens ist die Konstruktion einer
hegemonialen Differenz zwischen ‚uns' und den vermeintlich ‚Anderen'.
Die Überzeugung, \emph{weiße} Menschen könnten \enquote{im Namen und im
Interesse anderer sprechen} (Sonderegger 2008: 46), hat die Zeit des
Kolonialismus überdauert und wirkt in die postkoloniale Gegenwart
hinein.

Clara M. Chu, Professorin und Direktorin des \emph{Mortenson Center for
International Library Programs} in Illinois, argumentiert, dass
bestehende Systeme und Praktiken der bibliothekarischen Arbeit dazu
führen würden, gewissen Menschen Zugang sowie Repräsentation zu
verwehren. Diese Exklusion bezeichnet sie als eine \enquote{culture of
silence} (Chu zitiert nach Honma / Chu 2018: 459), die das System
erhält und \emph{weiße} Privilegien sowie dichotome Machtgefüge als
unhinterfragte Normen erscheinen lässt. Dieser \enquote{culture of silence}
könne allerdings dadurch begegnet werden, marginalisierte Communities
aktiv in die Bibliotheksarbeit einzubinden und ihnen zu ermöglichen,
alternative Wissensarchive zu entwickeln. Diese könnten dazu beitragen,
das vorherrschende Paradigma der Fremdrepräsentation durch
Selbstrepräsentation abzulösen (ebenda: 459).

Das Ziel dieses Artikels ist es aufzuzeigen, aus welchen Beweggründen
\enquote{Hatschi Bratschis Luftballon} in den Beständen zweier Wiener
Bibliotheken zur Verfügung gestellt wird und welche Argumentationen
dieser Entscheidung zugrunde liegen. Untersucht wurden dabei eine
Öffentliche (\emph{Büchereien Wien}) und eine Wissenschaftliche
Bibliothek (\emph{Fachbereichsbibliothek für Germanistik, Nederlandistik
und Skandinavistik}), da von der These ausgegangen wurde, dass die
unterschiedlichen Funktionen dieser Bibliotheksarten divergierende
Begründungen bedingen.

Es ist nicht das Ziel dieses Artikels, die unterschiedlichen Rassismen,
die sich auch durch die zahlreichen Neueditionen von \enquote{Hatschis
Bratschis Luftballon} ziehen, im Detail darzustellen. Allerdings sei
darauf hingewiesen, dass die Editionen ab 1943 und insbesondere ab 1960
textlich und bildlich stark abgeändert wurden. In den Editionen vor 1943
war Hatschi, der Entführer des kleinen Jungen Fritzchen, noch \enquote{der
Türke aus dem Türkenland} und wurde in späteren Editionen zum
\enquote{Zauberer aus dem Morgenland}. Die wohl weitreichendste Änderung
betraf die schriftliche und bildliche Umwandlung der
\enquote{Menschenfresser}-Szene. In der erwähnten Szene flog Fritzchen über
ein \enquote{Inselland im Meer}. Doch verweilen kann Fritzchen hier nicht, da
sich ihm die \enquote{Menschenfresser} (Ginzkey [1904] 2019: Bl. 25)
dieser Insel bereits mit Messern nähern. Schnell fliegt der Ballon
weiter, aber einige \enquote{Menschenfresser} haben schon nach dem Ballon
gegriffen. Sie können sich allerdings nicht lange halten und fallen der
Reihe nach ins Wasser und versinken. Bei den Zeichnungen der
\enquote{Menschenfresser} handelt es sich um rassistische Karikaturen
Schwarzer\footnote{Der Begriff \enquote{Schwarz} wird in diesem Zusammenhang
  großgeschrieben und ist \enquote{eine Selbstbezeichnung mit
  Widerstandspotenzial} und wird als \enquote{Analysebegriff [betrachtet],
  der alle People of Color umfasst, die von der weißen
  Mehrheitsgesellschaft als `anders' markiert werden} (Greve 2013: 37;
  Hervorhebungen im Original).} Menschen. In den Editionen ab 1960
werden die \enquote{Menschenfresser} zu Affen, die dasselbe Schicksal ereilt
(Ginzky [1904] (2011) \footnote{In der Edition aus dem Jahr 1968
  sind, wie auch im Original und im Faksimile, keine
  Seitennummerierungen vorhanden. Zur besseren Nachvollziehbarkeit
  greife ich daher bei der Inhaltsangabe auf die Edition aus dem Jahr
  2011 zurück, die kaum nennenswerte textliche Änderungen zu
  vorhergehenden Versionen enthält.}: 35 folgende).

Das Kinderbuch spiegelt historische Negativbilder über die ‚Anderen'
wider und lässt Rückschlüsse auf die Gesellschaft zu, innerhalb derer
sich Ginzkey als Autor des Buches bewegte -- es zeigt was zum Zeitpunkt
der Veröffentlichung sag- und denkbar war.

Die Autorin und Journalistin Duygu Özkans erläutert in ihrem Buch zum
Thema \enquote{Türkenbelagerung}, dass der Name des Zauberers, Hatschi, ein
Verweis auf die muslimische Pilgerfahrt nach Mekka sei, die \enquote{Haddsch}
genannt wird. Das Sujet der Kindesentführung sei der Versuch, eine
Parallele zur \enquote{Knabenlese} herzustellen, \enquote{im Zuge derer christliche
Buben aus dem Balkan für das osmanische Heer rekrutiert wurden} (Özkan
2011: 10).

Özkan legt dar, dass Ginzkey in seinem Kinderbuch dichotome Bilder über
die Türkei erzeugt, die von negativen Assoziationen dominiert sind und
gleichermaßen das kollektive Bewusstsein widerspiegeln, welches zur Zeit
der Entstehung des Buches in der Gesellschaft vorherrschte. Diese
historischen Negativbilder durchdringen nun den aktuellen Diskurs um die
zeitgenössische türkische Community. Rechte und rechtsextreme
Bewegungen, wie zum Beispiel die \emph{Identitäre Bewegung Österreich}
nutzen \enquote{alte Feindbilder} zum Osmanischen Reich, um ihre Belange einer
restriktiven Asylpolitik zu legitimieren (vergleiche APA-OTS 2019; Özkan
2011: 10 folgende).

Die Berücksichtigung der Situiertheit von Autor:innen ist vor allem dann
unerlässlich, wenn über die vermeintlich ‚Anderen' aus ‚westlicher'
Perspektive gesprochen wird, mit dem Ziel Macht- und
Herrschaftsverhältnisse aufzudecken. Epistemische Gewalt, die vor allem
im kolonialen Kontext zur Legitimierung von Ungleichheitsverhältnissen
diente, bleibt auch in der Gegenwart wirkmächtig (vergleiche (Spivak
[1988] 2008: 42, 60).

Ginzkey wurde im Jahr 1871 in Pula (heute Kroatien) geboren. Sein Vater
Franz Ginzkey war für die österreichische Marinebasis in Pula als
Ingenieur tätig. Die Mutter, Mathilde Ginzkey, starb ein Jahr nach der
Geburt des Sohnes. Franz Karl Ginzkey besuchte Schulen in Pula und
Triest, die ihn auf eine Militärlaufbahn vorbereiten sollten. Bis zu
seiner Pensionierung im Jahr 1920 war er bei zwei Infanterieregimenten
tätig, arbeitete als Terrain-Zeichner für das \emph{k. und k.
Militärgeographische Institut} in Wien und unternahm nebenbei immer
wieder Versuche in Richtung einer literarischen Karriere (vergleiche
Heydemann 1985).

Ginzkey und seine Werke wurden sowohl während der Zeit des
Austrofaschismus als auch des Nationalsozialismus von Seiten des Regimes
akzeptiert und gefördert. Nach 1945 gelang es Ginzkey, die Zeit des
Austrofaschismus sowie die des Nationalsozialismus aus seiner
persönlichen Biographie auszuklammern. So kam es in den fünfziger Jahren
zu einer \enquote{Ginzkey-Renaissance} (Hawle 1989: 109). 1951 erhielt er den
Professorentitel, 1958 wurde ihm der \emph{Große Österreichische
Staatspreis} verliehen, in Seewalchen -- seinem Hauptwohnsitz seit 1944
-- wurde er zum Ehrenbürger ernannt und nach seinem Tod, im Jahr 1963,
am Wiener Zentralfriedhof in einem Ehrengrab bestattet (ebenda: 109
folgende). Sein Kinderbuch ist in der österreichischen
Bibliothekslandschaft weit verbreitet.

\hypertarget{hatschi-bratschis-luftballon-in-zwei-wiener-bibliotheken}{%
\section{\enquote{Hatschi Bratschis Luftballon} in zwei Wiener
Bibliotheken}\label{hatschi-bratschis-luftballon-in-zwei-wiener-bibliotheken}}

Bibliotheken sind Dienstleistungsbetriebe für ihre jeweiligen
Nutzer:innen und haben die Aufgabe, Zugang zu Literatur und anderen
Medien kostengünstig -- wenn nicht sogar kostenfrei -- sicherzustellen,
Unterstützung bei der Recherche anzubieten und Medien- sowie
Informationskompetenz zu vermitteln. Sie sind nicht kommerziell
ausgerichtet und werden in den meisten Fällen aus Steuermitteln
finanziert (Gantert 2016: 6 folgende).

Öffentliche Bibliotheken sind für die Literaturversorgung der gesamten
Bevölkerung einer bestimmten Region zuständig und der Bestand ist
vielfältig. Medien, die der \enquote{allgemeinen, politischen und beruflichen
Bildung} (Gantert 2016: 27) dienen, sind in diesen Bibliotheken ebenso
vorhanden wie solche, die dem Aspekt der Unterhaltung sowie
Freizeitaktivitäten zuzuordnen sind.

Wissenschaftliche Bibliotheken richten sich hauptsächlich an
wissenschaftliches Personal und Studierende, um Lehre und Forschung zu
unterstützen. Außerdem wird Fachliteratur zur Verfügung gestellt, die
bestimmte Berufsgruppen für die Ausübung ihrer Tätigkeit benötigen.
Wissenschaftliche wie Öffentliche Bibliotheken sind meist frei
zugänglich, auch wenn erstere die Barrieren teilweise höher stecken,
indem es beispielsweise Altersbeschränkungen oder Eintrittsgelder
(\emph{Österreichische Nationalbibliothek}) gibt (vergleiche Gantert
2016: 9).

Um herauszufinden, warum sich das Buch in Bibliotheksbeständen befindet
und ob es abhängig vom Bibliothekstyp unterschiedliche
Argumentationsweise gibt, wurden Interviews geführt. Befragt wurden die
für die Medienauswahl verantwortlichen Mitarbeiter:innen der
\emph{Büchereien Wien} und der \emph{Fachbereichsbibliothek für
Germanistik, Nederlandistik und Skandinavistik} der
Universitätsbibliothek Wien.

\hypertarget{hatschi-bratschis-luftballon-in-den-buxfcchereien-wien}{%
\subsection{\texorpdfstring{\enquote{Hatschi Bratschis Luftballon} in den
\emph{Büchereien
Wien}}{\enquote{Hatschi Bratschis Luftballon} in den Büchereien Wien}}\label{hatschi-bratschis-luftballon-in-den-buxfcchereien-wien}}

Die \emph{Büchereien Wien} umfassen neben der \emph{Hauptbücherei} 38
Zweigstellen sowie das \emph{Bibliothekspädagogische Zentrum}. Insgesamt
stehen den Benutzer:innen über 1,5 Millionen Medien zur Verfügung. In
den \emph{Büchereien Wien} befinden sich laut Online-Katalog (Stand:
Juni 2021), über die Zweigstellen verteilt, Ausgaben von \enquote{Hatschi
Bratschis Luftballon} aus den Jahren 1968, 2006 und 2011. Bei der
Ausgabe aus dem Jahr 2006 handelt es sich um eine Medienkombination mit
CD und Noten. \enquote{Hatschi Bratschis Luftballon} wurde hier als
Kindermusical vertont. Die älteste Ausgabe stammt aus dem Jahr 1968, was
bedeutet, dass alle Editionen -- ob in Buch- oder CD-Form -- in
bildlicher und sprachlicher Hinsicht Abwandlungen des Originals
darstellen.

Um der Frage nachzugehen, warum sich \enquote{Hatschi Bratschis Luftballon} im
Bestand der Büchereien Wien befindet, wurde ein schriftliches
Expertinneninterview\footnote{Das schriftliche Interview wurde der
  Autorin von Veronika Freytag im August 2020 übermittelt.} mit Veronika
Freytag, der Leiterin des Lektorats und Medienankaufs der
\emph{Büchereien Wien}, geführt. Dabei konnten zu den Themenbereichen
Bestandspolitik und Umgang mit rassistischen Inhalten in Kinderbüchern
folgende Informationen gewonnen werden:

Die \emph{Hauptbücherei} agiert bei Bestellungen autonomer als die
restlichen Zweigstellen, deren Bestellvorgänge zu einem großen Teil vom
zentralen Lektorat abhängen. Der autonome Medienankauf der Zweigstellen
bei Lieferant:innen findet in einem begrenzten Ausmaß statt.
Neuerscheinungen und -- in besonderen Fällen -- Neuauflagen bilden den
größten Teil des Medienankaufs. Das zentrale Lektorat tätigt hier eine
Vorauswahl ehe die Medien angekauft werden, wohingegen \enquote{Medien mit
älterem Erscheinungsdatum [\ldots] nur über den Selbstankauf und in
geringer Zahl erworben [werden]} (Interview Freytag 2020). Der
Ankauf von \enquote{Hatschi Bratschis Luftballon} wird demnach ausschließlich
über den Selbstankauf getätigt (Telefonat Freytag 2021).

Die Medienauswahl erfolgt, wie auch die Einteilung des Medienbudgets,
dezentral und wird von den Mitarbeiter:innen des jeweiligen Standortes
entschieden, da sie als Expert:innen des spezifischen Medienbedarfs
ihrer Nutzer:innen gelten. Der Ankauf wird außerdem auf die regelmäßig
durchgeführte \enquote{Ausleih- und Bestandsstatistik} abgestimmt.

Medien werden aus dem Bestand ausgeschieden, wenn sie \enquote{inhaltlich
veraltet, optisch unattraktiv/abgenutzt/beschädigt, nicht oder schlecht
ausgeliehen [\ldots] [sind]}, oder aufgrund von \enquote{Alter
(Erscheinungsdatum), Platzmangel, nicht zu den Zielgruppen/zum
Bestandskonzept passen} (Interview Freytag 2020).

Veronika Freytag betont, dass der wichtigste Grundsatz der
Bibliotheksarbeit der Informationszugang sei. Das Heranziehen ethischer
Gründe bei der Erwerbsentscheidung oder der Aussonderung gewisser Medien
wird von ihr als \enquote{heikel} bezeichnet, da es dem Informationsgrundsatz
widersprechen würde (ebenda).

Freytag nahm bei einem im Mai 2021 geführten Telefonat Bezug auf
Debatten, die sich um die bibliothekarischen Grundsätze der Neutralität
sowie Pluralität ranken -- meist im Kontext der Aufnahme von
rechtsextremer Literatur in den Bestand von Bibliotheken. Ein Verfechter
jener Grundsätze ist Hermann Rösch, Professor am \emph{Institut für
Informationswissenschaften} an der \emph{Technischen Hochschule Köln}.
Röschs Plädoyer für die Neutralität von Bibliotheken begreift Joachim
Eberhardt, der Leiter der \emph{Lippischen Landesbibliothek}, als
Plädoyer für die moralische Verpflichtung, rechtsextreme Literatur in
den Bibliotheksbestand aufzunehmen und stellt sich dieser Argumentation
vehement entgegen. Es gäbe zwar Gründe für die Aufnahme rechtsextremer
Werke in den Bibliotheksbestand, diese seien allerdings sachlich --
durch zum Beispiel die öffentliche Diskussion eines Werkes -- zu
argumentieren (Eberhardt 2019: 107; vergleiche Rösch 2018). Unter dem
Deckmantel des freien Informationszugangs und ohne sachliche Gründe
rechtsextreme Literatur anzubieten, lehnt er ab (Eberhardt 2019: 97).

Freytag legte dar, dass sich die Neutralitätsdebatte auf den Ankauf
neuerer Literatur beziehen würde und es schwierig sei, diese Maßstäbe
auf (historische) Kinderbücher umzulegen. Würden sich Mitarbeiter:innen
dafür entscheiden, alle Editionen von \enquote{Hatschi Bratschis Luftballon}
aus dem Bestand zu nehmen, käme dies einer Bevormundung der Nutzer:innen
gleich. Schließlich seien Bibliothekar:innen im aktuellen Verständnis
keine Pädagog:innen und können Leser:innen nicht die Fähigkeit
absprechen, sich eine eigene Meinung zu bilden. \enquote{Hatschi Bratschis
Luftballon} sei außerdem in erster Instanz ein Buch, das Kindern von
Erwachsenen vorgelesen wird. Dabei könne das Vorgelesene erklärt und
eingeordnet werden (Telefonat Freytag 2021). Außerdem sei zu bedenken,
dass die Inhalte von angebotenen Medien nicht per se mit den Meinungen
der Bibliotheksmitarbeiter:innen und der Bibliotheksnutzer:innen
übereinstimmen würden. Anders formuliert: Die Inhalte lassen keine
Rückschlüsse auf die Privatmeinung der beiden Personengruppen zu --
geschweige denn auf die Motive, aufgrund derer Medien entlehnt werden
(Interview Freytag 2020).

So betont Freytag, dass die Darstellung \enquote{Hatschi Bratschis} auch in
jenen Editionen, die sich im Bestand der \emph{Büchereien Wien}
befinden, relativ offensichtlich stereotype Darstellungen über die
‚Anderen' transportiere, indem die ‚Fremde' und die dort lebenden
Menschen als bedrohlich dargestellt werden. Der politische Diskurs in
Europa sei nach wie vor von Stereotypen geprägt, die sich auch im
Kinderbuch wiederfinden ließen. Der als gefährlich konstruierte
\enquote{Hatschi Bratschi} wird im Kinderbuch den ‚Anderen' zugeordnet. Dazu
merkt Freytag an, dass es sich dabei um eine gefährliche Verknüpfung
handle, da \enquote{Männer mit dunkler Haut und schwarzem Bart zu unserer
Gesellschaft gehören} (ebenda).

Da \enquote{Hatschi Bratschis Luftballon} zwar inhaltlich veraltet sei,
erfülle der Titel zwar ein Ausscheidungskriterium, als \enquote{Klassiker der
österreichischen Kinderliteratur} und seiner ungebrochenen Rezeption
bleibe er allerdings trotz der rassistischen Stereotype, die das Buch
selbst in den redigierten Fassungen enthält, im Bestand der
\emph{Büchereien Wien} (Interview Freytag 2020). Die Nachfrage sei im
Vergleich zu anderen Bilderbüchern sehr hoch: \enquote{9 Exemplare hatten
letztes Jahr zwischen 10 und 15 Entlehnungen} (ebenda).

Grundsätzlich ist die Auseinandersetzung mit Rassismen in Kinderbüchern
Teil der bibliothekarischen Arbeit der \emph{Büchereien Wien}. Zwei
gängige und von den \emph{Büchereien Wien} begrüßte Umgangsformen mit
rassistischen Inhalten sind zum einen Umformulierungen und zum anderen
Hinweise auf die Historizität bestimmter rassistischer
Begrifflichkeiten. So wurde bei \enquote{Jim Knopf und Lukas der
Lokomotivführer} im Online-Katalog darauf hingewiesen, dass es sich bei
bestimmten im Text verwendeten Begrifflichkeiten -- wie dem
\enquote{N-Wort}\footnote{In Anlehnung an Gudrun Hentges Sammelbandtitel gehe
  ich von der These aus: \enquote{Sprache Macht Rassismus} (2014). Um die
  weitere Reproduktion dieser Begrifflichkeit zu vermeiden, verwende ich
  eine Begriffsabwandlung.} -- um historisch abwertende Bezeichnungen
handelt (Interview Freytag 2020).

Freytag nannte außerdem zwei weitere Beispiele aus der Kinderabteilung,
um darzustellen, wie mit Büchern umgegangen wird, deren Inhalt als
rassistisch gewertet wird: Dabei handelt es sich um den Comic \enquote{Tim im
Kongo} von Hergé sowie \enquote{Barbar auf Reisen} von Jean de Brunhoff.
\enquote{Tim im Kongo} wurde aufgrund seines rassistischen Inhaltes aus der
Kinderabteilung in die Erwachsenenabteilung und kurz darauf in das
Magazin der \emph{Büchereien Wien} verlegt. Befinden sich Medien in
einem der zwei Magazine der \emph{Büchereien Wien}, können diese erst
nach Vorbestellung genutzt werden. \enquote{Tim im Kongo} wurde in das Magazin
verlegt und nicht aus dem Bestand ausgesondert, weil es sich um einen
frühen Band der bei den Leser:innen beliebten Comicbuchreihe \enquote{Tim und
Struppi} handelt, deren Bände vollständig im Bestand der
\emph{Büchereien Wien} sind. \enquote{Barbar auf Reisen} wurde hingegen
aufgrund geringer Rezeption aus dem Bestand der \emph{Büchereien Wien}
aussortiert und nach Neuankauf durch eine Zweigstelle ebenfalls ins
Magazin verlegt (Interview Freytag 2020; Telefonat Freytag 2021).

\hypertarget{hatschi-bratschis-luftballon-in-der-fachbereichsbibliothek-fuxfcr-germanistik-nederlandistik-und-skandinavistik}{%
\subsection{\texorpdfstring{\enquote{Hatschi Bratschis Luftballon} in der
\emph{Fachbereichsbibliothek für Germanistik, Nederlandistik und
Skandinavistik}}{\enquote{Hatschi Bratschis Luftballon} in der Fachbereichsbibliothek für Germanistik, Nederlandistik und Skandinavistik}}\label{hatschi-bratschis-luftballon-in-der-fachbereichsbibliothek-fuxfcr-germanistik-nederlandistik-und-skandinavistik}}

Laut Online-Katalog der \emph{Universitätsbibliothek} \emph{Wien}
befinden sich Ausgaben aus den Jahren 1903 (hier handelt es sich um die
Erstausgabe, die meinen Recherchen nach erstmals 1904 publiziert wurde)
1943, 1951, 1960, 1968, 2006 und das Faksimile der Erstausgabe aus dem
Jahr 2019 in der \emph{Hauptbibliothek} und zusätzlich eine Ausgabe aus
dem Jahr 2006 sowie das Faksimile in der \emph{Fachbereichsbibliothek}
\emph{für Germanistik, Nederlandistik und Skandinavistik}. Zur Ausgabe
aus dem Jahr 2006 gehört eine Musik-CD (es handelt sich vermutlich um
jene Edition aus dem Jahr 2006, die sich auch im Bestand der
\emph{Büchereien Wien} befindet), die allerdings nicht im
Freihandbereich, sondern im \enquote{Magazin: AV-Medien} der
\emph{Fachbereichsbibliothek} aufgestellt ist.

Mit dem Leiter der \emph{Fachbereichsbibliothek}, Stefan
Alker-Windbichler, wurde ebenfalls ein schriftliches
Experteninterview\footnote{Das schriftliche Interview wurde der Autorin
  von Stefan Alker-Windbichler im September 2020 übermittelt.} geführt.
In Bezug auf die Bestandspoltik und den Umgang mit rassistischen
Inhalten konnten spezifische bibliotheksinterne Praktiken in Erfahrung
gebracht werden.

So erfolgt der Bestandsaufbau der einzelnen bibliothekarischen Entitäten
in gegenseitiger Absprache: \enquote{Um die für den Literaturaufwand zur
Verfügung stehenden Budgetmittel effizient einzusetzen, koordinieren die
bibliothekarischen Einheiten ihren Bestandsaufbau untereinander und
kooperieren miteinander} (Universitätsbibliothek 2012: 3).

Die \emph{Fachbereichsbibliothek} \emph{für Germanistik, Nederlandistik
und Skandinavistik} ist, wie der Name bereits erahnen lässt, für die
Literaturversorgung mehrerer Studienrichtungen verantwortlich. Das
Zielpublikum der \emph{Fachbereichsbibliothek} umfasst in erster Linie
Universitätsangehörige, die vor Ort mit Literatur für Studium, Forschung
und Lehre versorgt werden. Der Bestand der \emph{Fachbereichsbibliothek}
beträgt ungefähr 167.000 Bände. Die Signaturengruppe \enquote{Kinder- und
Jugendliteratur}, die sich im Freihandbereich der
\emph{Fachbereichsbibliothek} befindet, umfasst ungefähr 440 Werke. Bei
etwa 180 Titeln handelt es sich um Kinder- und Jugendliteratur, der Rest
ist Forschungsliteratur zum Thema. Kinder- und Jugendliteratur bildet
folglich nur einen kleinen Teil des Bestands der
\emph{Fachbereichsbibliothek} und wird zudem nicht systematisch
gesammelt (Interview Alker-Windbichler 2020).

Medien dieser Literaturgruppe werden angekauft, um den spezifischen
Bedarf für Lehrveranstaltungen zu decken oder zur \enquote{Ergänzung schon
vorhandener Bestände} (ebenda). Außerdem werden in Wien erschienene
\enquote{Pflichtexemplare} in diese Bestandsgruppe aufgenommen.

Alker-Windbichler führt an, dass gesellschaftliche, mediale und
akademische Debatten Einfluss auf Erwerbsentscheidungen haben, was
allerdings aufgrund des \enquote{fehlenden Sammelprofils} nicht auf Kinder-
und Jugendliteratur zutreffe. Auch ein festgelegter Erwerbungsetat sei
für diese Gruppe aus demselben Grund nicht vorhanden. Ankaufsvorschläge
werden berücksichtigt, vorausgesetzt sie entsprechen dem Grundsatz der
Literaturversorgung von Universitätsangehörigen. Das bedeutet für
Kinder- und Jugendbücher, dass sie im Rahmen einer Lehrveranstaltung
benötigt werden müssen.

Die Erwerbungsentscheidung für das Faksimile der Erstausgabe von
\enquote{Hatschi Bratschis Luftballon} aus dem Jahr 2019 wurde von
Alker-Windbichler selbst getroffen. Seine Entscheidung begründet er mit
dem bereits erwähnten Grundsatz der Literaturversorgung des
Zielpublikums der \emph{Fachbereichsbibliothek}, da am \emph{Institut
für Germanistik} wiederholt Seminare zum Thema \enquote{Bilderbuchforschung}
angeboten werden würden. Außerdem habe der Erwerb der \enquote{Ergänzung des
schon vorhandenen Bestandes} gedient (Interview Alker-Windbichler
2020).

Zusätzlich gibt es -- wie eingangs erwähnt wurde -- auch eine Version
des Kinderbuches aus dem Jahr 2006, die sich in der Freihandaufstellung
neben dem Faksimile befindet. So findet durch die Aufstellung eine
Kontextualisierung des Faksimiles statt, das \enquote{schon durch [seine]
Aufmachung das Abgründige des Werkes sichtbar} mache.

Die Nutzung von \enquote{Hatschi Bratschis Luftballon} ausgehend von den
Entlehnzahlen wird als \enquote{relativ kontinuierlich} beschrieben, wobei
festzuhalten ist, dass diese Zahlen nicht der tatsächlichen Nutzung
entsprechen, da das Buch durch die Freihandaufstellung -- wie auch in
den \emph{Büchereien Wien} -- vor Ort genutzt werden kann. Aus den
Entlehnzahlen Rückschlüsse auf die öffentliche oder wissenschaftliche
Rezeption zu ziehen, ist allerdings schwer, \enquote{weil
Vergleichszahlen\footnote{Die Entlehnzahlen liegen weit unter jenen der
  Büchereien Wien (vergleiche Interview Freytag 2020: Zeile 74--77).}
ebenso fehlen wie Informationen über die Entlehner*innen und ihre
Nutzungsabsichten} (ebenda).

Die Aussortierungskriterien sind ähnlich wie bei den \emph{Büchereien
Wien} vom äußeren Zustand des Werkes abhängig. Außerdem werden Bücher
aus dem Bestand ausgesondert, wenn eine Neuauflage vorliegt. Anders als
bei den \emph{Büchereien Wien} haben Entlehnzahlen aber keinen Einfluss
auf die Aussonderungsentscheidung: \enquote{Systematisches Ausscheiden wegen
Nichtgängigkeit gibt es nicht} (ebenda).

Innerhalb der \emph{Universitätsbibliothek} existieren neben dem seit
2004 betriebenen Arbeitsbereich \emph{NS-Provenienzforschung}, im Rahmen
dessen die Bestände der \emph{Universitätsbibliothek} systematisch auf
NS-Raubgut\footnote{Der anfängliche Untersuchungsrahmen betraf
  Erwerbungen, die zwischen 1938 und 1945 getätigt wurden. Im Zuge der
  Forschungen wurde dieser Rahmen jedoch auf die Jahre zwischen 1933 und
  1938 sowie die Nachkriegsjahre ausgedehnt (Stumpf 2019: 67 folgende).}
überprüft werden (Stumpf 2019: 67), zwei weitere Maßnahmen, deren Ziel
es ist, Bibliotheksnutzer:innen auf bedenkliche Erwerbungen aufmerksam
zu machen. Bei den Maßnahmen handelt es sich um sogenannte
\enquote{Stempeluhren} und \enquote{Denkzettel}, die auf Initiative der
\emph{Fachbereichsbibliothek Kunstgeschichte} eingeführt wurden, nachdem
Bibliotheksnutzer:innen von Stempeln mit NS-Symbolik in verschiedenen
Büchern der \emph{Universitätsbibliothek Wien} irritiert waren. Diese
\enquote{Stempeluhren} und \enquote{Denkzettel} sollen eine Annäherung an die
Thematik des Umgangs mit rassistischen Inhalten in wissenschaftlichen
Bibliotheken darstellen, auch wenn sie lediglich auf formale Aspekte --
den Erwerbungszeitraum -- abzielen.

Im Oktober 2019 fand zu diesem Thema eine Podiumsdiskussion unter dem
Titel \enquote{Zum Umgang mit NS-Symbolen im universitären Kontext} statt. Es
wurden sogenannte Stempeluhren entwickelt, die an manchen Standorten der
\emph{Universitätsbibliothek} aufliegen und den Nutzer:innen die
Möglichkeit bieten, anhand der Uhren den Erwerbungszeitraum des Buches
festzustellen (Fachbereichsbibliothek Kunstgeschichte o.\,J.). Neben den
Stempeln aus der NS-Zeit sind in den Büchern der
\emph{Universitätsbibliothek} auch die Stempel anderer politischer
Systeme Österreichs abgedruckt: Monarchie, \enquote{Deutschösterreich},
Austrofaschismus, Zweite Republik sowie das Universitätssiegel, das seit
2004 in Verwendung ist (Stumpf 2015: 552 folgende).

Die \enquote{Denkzettel} stellen eine Art Ergänzung zur \enquote{Stempeluhr} dar und
sollen Nutzer:innen die Möglichkeit geben, andere Nutzer:innen darauf
aufmerksam zu machen, dass es sich um ein Werk handelt, das während des
Nationalsozialismus erworben wurde. Dadurch solle ein \enquote{kollektiver
Bearbeitungsprozess} (Fachbereichsbibliothek Kunstgeschichte o.\,J.)
ermöglicht werden. Die Werke werden so als Zeitdokumente
kontextualisiert und \enquote{zugleich distanziert sich die
Universitätsbibliothek Wien von allen diskriminierenden und
gewaltverherrlichenden Inhalten} (ebenda).

In Bezugnahme auf diese Formulierungen merkt Alker-Windbichler an, dass
hier die Gefahr bestünde, durch solche \enquote{Einzelaktionen} die
rassistischen Inhalte von Werken zu relativieren: \enquote{[E]s reicht
jedenfalls nicht, sich durch die Markierung besonders eindeutiger Werke
wie `Hatschi Bratschis Luftballon' von allen diskriminierenden und
gewaltverherrlichenden Inhalten zu distanzieren} (Interview
Alker-Windbichler 2020).

In der \emph{Fachbereichsbibliothek Germanistik, Nederlandistik und
Skandinavistik} liegen weder \enquote{Stempeluhren} noch \enquote{Denkzettel} bei,
was von Alker-Windbichler damit begründet wird, dass solche Maßnahmen
nur auf formale Kriterien hinweisen würden (Interview Alker-Windbichler
2020). Die inhaltliche Auseinandersetzung mit rassistischen Medien
erfordere eine viel umfangreichere Beschäftigung mit der Thematik, da
der Großteil der deutschsprachigen Literatur von solchen Inhalten -- in
der ein oder anderen Form -- geprägt sei. Die inhaltliche Auswertung des
Bestandes wäre dafür notwendig und sei dementsprechend \enquote{komplexer}
(ebenda).

Alker-Windbichlers Ansicht nach kann die formale Markierung von Werken
durch \enquote{Denkzettel} oder auch \enquote{Stempeluhren} ein \enquote{Denkanstoß} für
Nutzer:innen sein, sich mit der Erwerbsgeschichte beziehungsweise der
Geschichte des Bibliotheksbestands auseinanderzusetzen. Der
Erwerbszeitraum sei allerdings \enquote{kein überzeugendes Kriterium} für eine
tiefergehende inhaltliche Auseinandersetzung. Vielmehr müsste dafür
\enquote{das Handeln der Bibliotheken insgesamt im Kontext der Forschung
[befragt werden]} (Interview Alker-Windbichler 2020).

\hypertarget{resuxfcmee-und-ausblick}{%
\section{Resümee und Ausblick}\label{resuxfcmee-und-ausblick}}

Sowohl öffentliche als auch wissenschaftliche Bibliotheken begegnen im
Umgang mit Medien, deren Inhalte Rassismen enthalten, Herausforderungen,
die Fragen aufwerfen: Widerspricht die Entscheidung gegen den Erwerb
eines Werks aufgrund seines diskriminierenden Inhalts dem Grundsatz der
Neutralität? Sollten Bibliothekar:innen ihre Nutzer:innen auf
problematische Inhalte aufmerksam machen? Wo endet die Vermittlung von
Informationskompetenz und wo beginnt Bevormundung?

In der deutschsprachigen Debatte stellen sich Bibliotheken vor allem in
Hinblick auf rechtsextreme oder rechtsradikale Literatur diese Fragen.
Würde diese Debatte auf rassistische Inhalte ausgedehnt, stünden
Bibliotheken vor der Problematik, dass -- wie Alker-Windbichler treffend
feststellte -- ein Großteil der deutschsprachigen Literatur davon
betroffen wäre und hinterfragt werden müsste.

Der Bestand öffentlicher Bibliotheken spiegelt in gewissem Maße die
Interessen seiner Nutzer:innen wider. \enquote{Hatschi Bratschis Luftballon}
wird in den \emph{Büchereien Wien} nach wie vor rezipiert, die Nachfrage
nach Ginzkeys \enquote{Klassiker der österreichischen Kinderliteratur} bleibt
bestehen und deshalb befindet sich das Kinderbuch im Bestand dieser
Öffentlichen Bibliothek. Es wäre zu kurz gefasst anzunehmen, die
Nachfrage nach diesem Kinderbuch decke sich mit einem Interesse an
Medien mit rassistischen Inhalten. Bibliotheken kennen die Motivation
ihrer Entlehner:innen nicht und vermutlich sind sich auch einige Eltern
nicht bewusst, dass \enquote{Hatschi Bratschis Luftballon} rassistische
Stereotype reproduziert.

Ist allerdings ein Bewusstsein dafür vorhanden und wird jegliche
kritische Auseinandersetzung mit den inhärenten Rassismen als
vermeintlich übertriebene Political Correctness abgetan, wird der Kampf
um Deutungshoheit deutlich: \enquote{Es ist das eine, Rassismus zu
reproduzieren, weil man ihn nicht erkennt. Es ist etwas anderes,
Rassismus zu reproduzieren, weil man die Perspektiven anderer Menschen
nicht anerkennt} (Hasters 2020). Alice Hasters, die durch ihr Buch
\enquote{Was weiße Menschen nicht über Rassismus hören wollen, aber wissen
sollten} (2019) Bekanntheit erlangte, fasst in diesem Statement
treffend zusammen, dass die Verteidigung oder auch wissentliche Hinnahme
von Rassismus mit dem Bedürfnis einhergeht, diskriminierte
Personengruppen aus unserer Gesellschaft zu exkludieren. Wie eingangs
erwähnt, haben Bibliotheken die Möglichkeit, diesen exkludierenden
Mechanismen entgegenzutreten, indem sie marginalisierte Communities
aktiv in die Bibliotheksarbeit einbinden und ihnen ermöglichen,
Wissensarchive zu entwickeln, die Alternativen zu jenem \emph{weißen}
Wissensarchiv aufzeigen, welches die Bibliothekslandschaft des globalen
Nordens prägt (Honma / Chu 2018: 459).

Im Falle der \emph{Fachbereichsbibliothek Germanistik, Skandinavistik
und Nederlandistik} wurde \enquote{Hatschi Bratschis Luftballon} für den
spezifischen Bedarf von Lehrveranstaltungen zur Bilderbuchforschung
angekauft. Wäre \enquote{Hatschi Bratschis Luftballon} nicht im Bestand
Wissenschaftlicher (oder auch Öffentlicher) Bibliotheken, würde dies die
wissenschaftliche Analyse der darin vorkommenden Rassismen erschweren.
Dennoch müssen sich auch Wissenschaftliche Bibliotheken fragen, welche
Werke angeboten werden und -- in Anlehnung an Clara M. Chus \enquote{culture of
silence} -- welche nicht.

\hypertarget{literatur}{%
\section{Literatur}\label{literatur}}

APA-OTS (2019): Neofaschistisches \enquote{Gedenken} am Kahlenberg verhindert.
In:
\url{https://www.ots.at/presseaussendung/OTS_20190908_OTS0013/neofaschistisches-gedenken-am-kahlenberg-verhindert}
[27.02.2020{].

Attia, Iman (2016): Rassismustheoretische Perspektiven auf
sozialpädagogische Fallarbeit. In: Michel-Schwartze, Brigitta (Hg.): Der
Zugang zum Fall. Beobachtungen, Deutungen, Interventions-ansätze.
Wiesbaden: Springer Fachmedien, 229--242.

Bibliotheks- und Archivwesen der Universität Wien (o.\,J.): Über uns.
In: \url{https://bibliothek.univie.ac.at/ueber_uns.html}
[24.9.2020].

Chou, Rose L.; Pho, Annie (eds.) (2018): Pushing the margins. Women of
color and intersectionality in LIS. Sacramento: Library Juice Press.

Der Standard (2019): Rassismusvorwürfe gegen Wiener Café wegen eines
Kinderbuchs. In:
\url{https://www.derstandard.at/story/2000110109834/rassismusvorwuerfe-gegen-wiener-kaffeehaus-wegen-kinderbuch}
[15.12.2019].

Der Standard (2019a): Nach Protesten: Thalia und Amazon werden \enquote{Hatschi
Bratschis Luftballon} weiterhin verkaufen. In:
\url{https://www.derstandard.at/story/2000110496893/nach-protesten-thalia-und-amazon-werden-hatschi-bratschis-luftballon-weiterhin}
[15.12.2019].

Eberhardt, Joachim (2019): Rechte Literatur in Bibliotheken? Zur
Argumentation von Hermann Rösch. In: O-Bib. Das offene
Bibliotheksjournal, 6/3, 96--108.

Fachbereichsbibliothek Kunstgeschichte (o.\,J.): NS-Symbole in der
Fachbereichsbibliothek Kunstgeschichte? In:
\url{https://kunstgeschichte.univie.ac.at/ueber-uns/institutsnachrichten/bibliotheksstempel/}
[24.9.2020].

Gantert, Klaus (2016): Bibliothekarisches Grundwissen. Berlin / Boston:
Walter de Gruyter GmbH.

Ginzkey, Franz Karl [1904] (2011): Hatschi Bratschis Luftballon.
Erweiterte Auflage mit Illustrationen von Rolf Rettich, Grete Hartmann,
Ernst Dombrowski und Alena Schulz. Langenzersdorf: Trans-World
Musikverlag.

Ginzkey, Franz Karl [1904] (2019): Hatschi Bratschis Luftballon.
Eine Dichtung für Kinder. Faksimile der Erstausgabe aus dem Jahr 1904,
Wien: European University Press / Iberia.

Hasters, Alice (2020): Was weiße Menschen nicht über Rassismus hören
wollen, aber wissen sollten. München: hanserblau.

Hasters, Alice (2020): Warum weiße Menschen so gerne gleich sind. In:
\url{https://www.deutschlandfunk.de/identitaeten-7-7-warum-weisse-menschen-so-gerne-gleich-sind.1184.de.html?dram:article_id=466836}
[28.6.2021].

Hawle, Christian (1989): Wer war Franz Karl Ginzkey? Leben, Werk,
Wirken. In: Hangler, Reinhold; Hawle, Christian; Kilgus, Hartmuth;
Kriechbaum, Gerhard (Hg.): Der Fall Franz Karl Ginzkey und Seewalchen.
Eine Dokumentation. Vöcklabruck: Mauthausen-Aktiv-Vöcklabruck, 97--115.

Heydemann, Klaus (1985): Literatur und Markt. Werdegang und Durchsetzung
eines kleinmeisterlichen Autors in Österreich (1891--1938).
Habilitationsschrift, Universität Wien.

Honma, Todd; Chu, Clara M. (2018): Positionality, Epistemology, And New
Paradigms for LIS: A Critical Dialog With Clara M. Chu. In: Chou, Rose
L.; Pho, Annie (eds.) (2018): Pushing the margins. Women of color and
intersectionality in LIS. Sacramento: Library Juice Press, 447--465.

Matzinger, Lukas (2019): Warum führen Sie \enquote{Hatschi Bratschis
Luftballon}, Frau Leydolt? In:
\url{https://www.falter.at/zeitung/20191016/warum-fuehren-sie-hatschi-bratschis-luftballon--frau-leydolt/_584c2be682}
[15.12.2019].

Muddiman, Dave; Durrani, Shiraz; Dutch, Martin; Linley, Rebecca;
Pateman, John; Vincent, John (2000): Open to All? The Public Library and
Social Exclusion. In: Library and Information Commission Research Report
84: \url{http://eprints.rclis.org/6283/} [2.6.2020].

Ochsenhofer, Sigrid (1993): Kinder- und Jugendliteratur zu Beginn des
20. Jahrhunderts am Beispiel von Franz Karl Ginzkey. Diplomarbeit,
Universität Wien.

Osterhammel, Jürgen (1995): Kolonialismus. Geschichte -- Formen --
Folgen. München: C.H. Beck.

Özkan, Duygu (2011): Türkenbelagerung. Wien: Metroverlag.

Rösch, Hermann (2018): Zum Umgang mit umstrittener Literatur in
Bibliotheken aus ethischer Perspektive. Am Beispiel der Publikationen
rechtsradikaler und rechtspopulistischer Verlage. In: Bibliotheksdienst,
52/10-11, 773--783.

Schlessman-Tarango, Gina (ed.) (2017): Topographies of Whiteness.
Mapping Whiteness in Library and Information Science. Sacramento:
Library Juice Press.

Sonderegger, Arno (2008): Geschichte und Gedenken im Banne des
Eurozentrismus. In: Gomes, Bea; Schicho, Walter; Sonderegger, Arno
(Hg.\_innen): Rassismus. Beiträge zu einem vielgesichtigen Phänomen.
Wien: Mandelbaum, 45--72.

Spivak, Gayatri Chakravorty (2008) [1988]: Can the Subaltern Speak?
Postkoloniale und subal-terne Artikulation, aus dem Englischen übersetzt
von Alexander Joskowicz, Stefan Nowotny. Wien: Turia + Kant.
[Original: Spivak, Gayatri Chakravorty: Can the Subaltern Speak? In:
Cary Nelson \& Lawrence Grossberg (eds.): Marxism and the Interpretation
of Culture. Chicago: University of Illinois Press, 271--314.]

Stumpf, Markus (2015): Kontaminierte Bücher -- Exemplarspezifika und
Eigentumsnachweise in den Büchern der Universitätsbibliothek Wien. In:
Mitteilungen der VÖB 68/3, 546--565.

UB -- Universitätsbibliothek (2012): Sammelrichtlinien der
Universitätsbibliothek Wien. In:
\url{https://bibliothek.univie.ac.at/files/Sammelrichtlinien_2012.pdf}
[19.9.2020].

ZARA (2019): Kontroverse. Reproduktion von rassistischen Bildern in
historischen Kinderbüchern. In:
\url{https://zh-cn.facebook.com/zara.or.at/posts/-die-aktuelle-kontroverse-rund-um-das-kinderbuch-hatschi-bratschis-luftballon-ze/3094539020618916/}
[20.02.2020].

\begin{center}\rule{0.5\linewidth}{0.5pt}\end{center}

%autor
\textbf{Elisa Frei} studierte Internationale Entwicklung an der
Universität Wien und der Universidad de Salamanca. Sie schrieb ihre
Masterarbeit zum Thema \enquote{Ethnologische Museen im 21. Jahrhundert} und
analysierte den Umgang zweier ethnologischer Museen (dem Berliner
Ethnologischen Museum und dem Weltmuseum Wien) mit der
Kolonialgeschichte. Während ihres Geschichtestudiums an der Universität
Wien nahm sie an Seminaren zur \enquote{Vielstimmigkeit kolonialer Diskurse in
Bibliotheken} teil, im Rahmen derer dieser Artikel entstand.

\end{document}
