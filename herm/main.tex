\documentclass[a4paper,
fontsize=11pt,
%headings=small,
oneside,
numbers=noperiodatend,
parskip=half-,
bibliography=totoc,
final
]{scrartcl}

\usepackage[babel]{csquotes}
\usepackage{synttree}
\usepackage{graphicx}
\setkeys{Gin}{width=.4\textwidth} %default pics size

\graphicspath{{./plots/}}
\usepackage[ngerman]{babel}
\usepackage[T1]{fontenc}
%\usepackage{amsmath}
\usepackage[utf8x]{inputenc}
\usepackage [hyphens]{url}
\usepackage{booktabs} 
\usepackage[left=2.4cm,right=2.4cm,top=2.3cm,bottom=2cm,includeheadfoot]{geometry}
\usepackage{eurosym}
\usepackage{multirow}
\usepackage[ngerman]{varioref}
\setcapindent{1em}
\renewcommand{\labelitemi}{--}
\usepackage{paralist}
\usepackage{pdfpages}
\usepackage{lscape}
\usepackage{float}
\usepackage{acronym}
\usepackage{eurosym}
\usepackage{longtable,lscape}
\usepackage{mathpazo}
\usepackage[normalem]{ulem} %emphasize weiterhin kursiv
\usepackage[flushmargin,ragged]{footmisc} % left align footnote
\usepackage{ccicons} 
\setcapindent{0pt} % no indentation in captions

%%%% fancy LIBREAS URL color 
\usepackage{xcolor}
\definecolor{libreas}{RGB}{112,0,0}

\usepackage{listings}

\urlstyle{same}  % don't use monospace font for urls

\usepackage[fleqn]{amsmath}

%adjust fontsize for part

\usepackage{sectsty}
\partfont{\large}

%Das BibTeX-Zeichen mit \BibTeX setzen:
\def\symbol#1{\char #1\relax}
\def\bsl{{\tt\symbol{'134}}}
\def\BibTeX{{\rm B\kern-.05em{\sc i\kern-.025em b}\kern-.08em
    T\kern-.1667em\lower.7ex\hbox{E}\kern-.125emX}}

\usepackage{fancyhdr}
\fancyhf{}
\pagestyle{fancyplain}
\fancyhead[R]{\thepage}

% make sure bookmarks are created eventough sections are not numbered!
% uncommend if sections are numbered (bookmarks created by default)
\makeatletter
\renewcommand\@seccntformat[1]{}
\makeatother

% typo setup
\clubpenalty = 10000
\widowpenalty = 10000
\displaywidowpenalty = 10000

\usepackage{hyperxmp}
\usepackage[colorlinks, linkcolor=black,citecolor=black, urlcolor=libreas,
breaklinks= true,bookmarks=true,bookmarksopen=true]{hyperref}
\usepackage{breakurl}

%meta
%meta

\fancyhead[L]{P. Herm\\ %author
LIBREAS. Library Ideas, 40 (2021). % journal, issue, volume.
\href{https://doi.org/10.18452/23801}{\color{black}https://doi.org/10.18452/23801}
{}} % doi 
\fancyhead[R]{\thepage} %page number
\fancyfoot[L] {\ccLogo \ccAttribution\ \href{https://creativecommons.org/licenses/by/4.0/}{\color{black}Creative Commons BY 4.0}}  %licence
\fancyfoot[R] {ISSN: 1860-7950}

\title{\LARGE{Postkoloniale Perspektiven auf den Kosmos Bibliothek}}% title
\author{Paula Herm} % author

\setcounter{page}{1}

\hypersetup{%
      pdftitle={Postkoloniale Perspektiven auf den Kosmos Bibliothek},
      pdfauthor={Paula Herm},
      pdfcopyright={CC BY 4.0 International},
      pdfsubject={LIBREAS. Library Ideas, 40 (2021)},
      pdfkeywords={Bibliothek, Dekolonisierung, library, decolonization},
      pdflicenseurl={https://creativecommons.org/licenses/by/4.0/},
      pdfcontacturl={http://libreas.eu},
      baseurl={https://doi.org/10.18452/23801},
      pdflang={de},
      pdfmetalang={de}
     }



\date{}
\begin{document}

\maketitle
\thispagestyle{fancyplain} 

%abstracts

%body
Dieser Beitrag beruht auf der Masterarbeit der Autorin mit dem Titel
\enquote{Koloniale Spuren in bibliothekarischen Sammlungen und
Wissensordnungen -- eine Untersuchung am Beispiel der Staatsbibliothek
zu Berlin} (2019) am Institut für Bibliotheks- und
Informationswissenschaft der Humboldt-Universität zu Berlin. Besonderer
Dank der Autorin gilt Aisha Othman, Ulla Wimmer und Michaela
Scheibe.\footnote{Anmerkung Redaktion: alle Links im Beitrag wurden
  zuletzt erneut am 21.12.2021 geprüft.}

\hypertarget{einleitung-postkolonialituxe4t-nicht-nur-ein-thema-fuxfcr-museen}{%
\section{1. Einleitung: Postkolonialität -- nicht nur ein Thema
für
Museen?}\label{einleitung-postkolonialituxe4t-nicht-nur-ein-thema-fuxfcr-museen}}

Spätestens seit der Debatte um das Berliner Humboldt-Forum und der
Veröffentlichung des Berichts \emph{Zurückgeben. Über die Restitution
afrikanischer Kulturgüter Ende des Jahres 2018} von Felwine Sarr und
Bénédicte Savoy\footnote{Sarr, Felwine/Savoy, Bénédicte: Zurückgeben:
  Über die Restitution afrikanischer Kulturgüter. Berlin: Matthes \&
  Seitz, 2019.} steht die Frage nach dem Umgang mit dem kolonialen Erbe
für Museen prominent auf der Agenda. Bereits davor fanden Tagungen zum
Erfahrungsaustausch und zu einzelnen Aspekten wie der differenzierten
Definition der \enquote{Sensibilität} von Objekten in musealen
Sammlungen und den praktischen Herausforderungen der Erfassung von
kolonialer Provenienz statt.\footnote{Siehe beispielsweise
  \enquote{Nicht nur Raubkunst! Sensible Dinge in Museen und
  wissenschaftlichen Sammlungen}, 21./22.1.2016, Mainz,
  \url{https://www.ifeas.uni-mainz.de/nicht-nur-raubkunst-sensible-dinge-in-museen-und-universitaeren-sammlungen/},
  die an die wegweisende Publikation des Buches von Margit Berner,
  Anette Hoffmann und Britta Lange \enquote{Sensible Sammlungen. Aus dem
  anthropologischen Depot} (Philo Fine Arts, 2011) anknüpfte;
  \enquote{Provenienzforschung zu ethnologischen Sammlungen der
  Kolonialzeit}, 7./8.4.2017, München, aus der ein von Larissa Förster,
  Iris Edenheiser, Sarah Fründt und Heike Hartmann herausgegebener
  Tagungsband hervorging,
  \url{https://edoc.hu-berlin.de/bitstream/handle/18452/19769/Provenienzforschung.pdf};
  ``Schwieriges

  Erbe. Koloniale Objekte -- Postkoloniales Wissen'', 24.4.2017,
  Stuttgart, ein Tagungsbericht von Myriam Gröpl und Sara Capdeville
  findet sich unter:

  \url{https://kolonialismus.blogs.uni-hamburg.de/2017/05/19/tagungsbericht-schwieriges-erbe-koloniale-objekte-postkoloniales-wissen-24-04-2017-linden-museum-stuttgart/}}
Der Deutsche Museumsbund (DMB) erstellte einen Leitfaden zum Umgang, der
nun in dritter Fassung vorliegt,\footnote{\url{https://www.museumsbund.de/wp-content/uploads/2021/03/mb-leitfanden-web-210228-02.pdf}
  (zugegriffen am 15.5.2021).} das Deutsche Zentrum Kulturgutverluste
(DZK) richtete ein neues Handlungsfeld zur Kolonialprovenienz in seinen
Rängen ein und legte ein neues eigenes Förderprogramm auf\footnote{\url{https://www.kulturgutverluste.de/Webs/DE/Forschungsfoerderung/Projektfoerderung-Bereich-Kulturgut-aus-kolonialem-Kontext/Index.html}}
und die Kulturstiftung der Länder richtete eine Kontaktstelle für
Sammlungsgut aus kolonialen Kontexten in Deutschland\footnote{\url{https://www.kulturstiftung.de/kontaktstelle-sammlungsgut-aus-kolonialen-kontexten-in-deutschland/}}
ein. Es wurden Digitalisierungsvorhaben und internationale
beziehungsweise transkulturelle Kooperationen angestoßen.\footnote{Wie
  zum Beispiel das Projekt \enquote{Digital Benin},
  \url{https://digital-benin.org/}} Und: Es wurde auch -- bislang
vereinzelt -- restituiert.\footnote{So wurden zum Beispiel im Februar
  2019 die Witbooi-Objekte an Namibia restitutiert, siehe
  \url{https://www.lindenmuseum.de/sehen/rueckblick/witbooi-objekte} und
  es wurde die Rückgabe der Säule von Cape Cross an Namibia beschlossen,
  deren offizieller Vollzug sich jedoch aufgrund der Corona-Pandemie
  verzögert, \url{https://www.bundestag.de/presse/hib/845318-845318}}

Auch wenn bislang die Dekolonisierung von musealen Objektsammlungen (wie
-- perspektivisch -- jüngst die Dekoloniale-Veranstaltungen am 25. und 26.
Juni 2021 in Berlin\footnote{\url{https://www.dekoloniale.de/de}}) die
Aufmerksamkeit weitgehend auf sich zieht, geht das Thema des Umgangs mit
der kolonialen Vergangenheit und den Auswirkungen des
europäischen/deutschen Kolonialismus bis in die Gegenwart nicht nur
Museen an, sondern wirft auch Fragen für BID-Einrichtungen wie
Bibliotheken und Archive auf. Denn immerhin sind dies Orte, an denen
Wissen und Material -- auch aus kolonialen Kontexten -- gesammelt und
aufbewahrt werden. Wie sieht es also im Bereich der Bibliotheks-,
Informations- und Dokumentationseinrichtungen aus -- inwiefern stellen
sich dort postkoloniale, also durch die Kolonisierung und deren
Nachwirkungen verursachte, begründete, Fragen beziehungsweise inwiefern
könnte Postkolonialität und Dekolonialität/Dekolonisierung\footnote{Der
  in diesem Zusammenhang ebenfalls verwendete Begriff der Dekolonialität
  legt durch sein Präfix den Fokus der angestrebten Überwindung
  kolonialer Denkweisen und Verhältnisse nicht auf die zeitliche
  Dimension wie es das Präfix \enquote{post} tut, sondern antizipiert
  gewissermaßen die Erreichung dieses Ziel oder geht bereits den
  nächsten Schritt. Bei einem dekolonialen Ansatz im Umgang mit Texten,
  Diskursen, kulturellen Manifestationen oder materiellen Verhältnissen
  wird also bewusst auf eine vom Kolonialismus bereinigte Perspektive
  abgezielt.} dort ein Thema sein? Was bedeuten diese Begriffe bezogen
auf die systemische Rolle und die konkrete Praxis von Bibliotheken und
Informationseinrichtungen? Der vorliegende Beitrag umreißt die
Problematik und bietet eine erste Übersicht.

\hypertarget{eine-systemtheoretische-sicht-die-bibliothek-als-speicherort-des-kollektiven-geduxe4chtnisses}{%
\section{2. Eine systemtheoretische Sicht: Die Bibliothek als
Speicherort des kollektiven
Gedächtnisses}\label{eine-systemtheoretische-sicht-die-bibliothek-als-speicherort-des-kollektiven-geduxe4chtnisses}}

Will man das Verhältnis von BID-Einrichtungen und Kolonialität
ergründen, ist es hilfreich, sich zunächst einmal das Verhältnis
zwischen der Einrichtung Bibliothek und Wissenschaft sowie der
Gesellschaft ganz allgemein zu vergegenwärtigen. Einem
kulturwissenschaftlichen Ansatz folgend geht es hier um eine
systemtheoretische Sicht auf die gesellschaftliche Position der
Bibliothek als Institution. Auf der Grundlage des Luhmann'schen
Verständnisses von Wissenschaft als sozialem Funktionssystem, dessen
basale Operation Kommunikation ist, bezeichnet Konrad Umlauf
Bibliotheken als \enquote{Nukleus} und als \enquote{ein Subsystem der
Wissenschaft als arbeitsteilige Ausgliederung} innerhalb dieses
Kommunikationssystems.\footnote{Umlauf, Konrad: Handbuch Bibliothek:
  Geschichte -- Aufgaben -- Perspektiven, Stuttgart: Metzler, 2012, S.
  11--16.} Dabei bestünden die Leistungen dieses Subsystems in der
Bewältigung, Aufbewahrung und Organisation von einer Menge
zirkulierender Information. Damit ist auch klar, dass der
wissenschaftlichen Kommunikation vorenthalten wird, was nicht in den
Bestand einer Bibliothek aufgenommen wird oder was unauffindbar ist.
Umlauf kommentiert diese Bedeutungszuweisungen durch Auswahl und
Anordnung dahingehend, dass die Bibliothek die historische Dimension
jener Kontextualisierung kaum deutlich mache.\footnote{Ebenda; die
  Anleihen bei Luhmann zur systemtheoretischen Perspektive gerade im
  Hinblick auf Autopoiesis und Kommunikation werden auch erläutert in:
  Plassmann, Engelbert/Rösch, Hermann/ Seefeldt, Jürgen/Umlauf, Konrad:
  Bibliotheken und Informationsgesellschaft in Deutschland: Eine
  Einführung, Wiesbaden: Harrassowitz, 2011, S. 37--50; siehe dazu auch
  Rösch, Hermann: Academic Libraries und Cyberinfrastructure in den USA:
  Das System wissenschaftlicher Kommunikation zu Beginn des 21.
  Jahrhunderts, Wiesbaden: Dinges \& Frick, 2008, 11--30.}

Aufgrund dieser Einbindung in gesellschaftliche Kommunikationsprozesse,
in denen die Institution Bibliothek in Interaktion mit Wissenschaft und
Gesellschaft steht, kann von ihr mit Tanja Hebers Worten als
\enquote{Speichersystem des kulturellen Gedächtnisses} gesprochen
werden.\footnote{Heber, Tanja: Die Bibliothek als Speichersystem des
  kulturellen Gedächtnisses, München: Tectum-Verlag, 2009, insbesondere
  S. 31--113.} Um den Begriff des kulturellen Gedächtnisses noch genauer
zu definieren, sei auf Aleida und Jan Assmanns Definition verwiesen, die
es als eine in einer Kultur tief verankerte Tradition und als die über
Generationen hinweg überlieferten Texte, Bilder und Riten verstehen, die
das Zeit- und Geschichtsbewusstsein, das Selbst- und Weltbild
prägen.\footnote{Assmann, Jan: Thomas Mann und Ägypten. Mythos und
  Monotheismus in den Josephsromanen. München: Beck, 2006, S. 70.} Der
Begriff des Selbstbilds verweist hierbei wiederum auf das Konzept der
kollektiven Identität, einer Identität, welche Gesellschaften durch die
Imagination von Selbstbildern und das Ausbilden einer Kultur der
Erinnerung über die Generationenfolge hinweg
aufrechterhalten.\footnote{Assman, Jan: Das kulturelle Gedächtnis:
  Schrift, Erinnerung und politische Identität in frühen

  Hochkulturen, München: Beck, 2007, S. 18; Assmann, Aleida:
  Erinnerungsräume. Formen und Wandlungen des kulturellen Gedächtnisses.
  München: Beck, 2006, S. 19.}

\hypertarget{eine-postkoloniale-und-von-critical-whiteness-studies-inspirierte-perspektive-was-bedeutet-das}{%
\section{3. Eine postkoloniale und von Critical Whiteness Studies
inspirierte Perspektive -- was bedeutet
das?}\label{eine-postkoloniale-und-von-critical-whiteness-studies-inspirierte-perspektive-was-bedeutet-das}}

Der postkolonialen Theorie geht es darum, \enquote{die verschiedenen
Ebenen kolonialer Begegnungen in textlicher, figuraler, räumlicher,
historischer, politischer und wirtschaftlicher Perspektive zu
analysieren}.\footnote{do Mar Castro Varela, María/Dhawan, Nikita:
  Postkoloniale Theorie: Eine kritische Einführung, Bielefeld:
  Transcript, 2015, Klappentext.} Sie will \enquote{die
sozio-historischen Interdependenzen und Verflechtungen zwischen den
Ländern des \enquote*{Südens} und des \enquote*{Nordens}
{[}herausarbeiten{]}}.\footnote{Ebenda.} Dabei hat sie durchaus einen
subversiven Antrieb, schließlich zielt sie darauf ab, \enquote{die
Konsequenzen des kolonialen Diskurses in seinen komplexen
imperialistischen, patriarchalen und rassistischen Manifestationen
herauszufordern, um die aus demselben resultierenden Wahrheitsregimes
und Repräsentationsstrategien zu subvertieren},\footnote{do Mar Castro
  Varela, María/Dhawan, Nikita: Postkolonialer Feminismus und die Kunst
  der Selbstkritik,

  in: Steyerl, Hito und Gutierrez Rodriguez, Encarnación: Spricht die
  Subalterne deutsch? Münster: Unrast,

  2003, S. 272.} also freizulegen und umzukehren.

Einen maßgeblichen Grundstein für die postkoloniale Theorie in den
Kulturwissenschaften legte Edward Said mit seinem 1978 erstmalig
erschienenen Werk \emph{Orientalism}. Darin kritisiert der Autor -- eben
als \enquote{Orientalismus} -- den Blick westlicher sogenannter
Orientexperten auf die Gesellschaften in Asien, Nordafrika und dem Nahen
Osten als mystifizierend und überheblich.\footnote{Said, Edward W.:
  Orientalism, New York: Pantheon Books, 1978.} Er postuliert, dass
westliche Wissenschaftler*innen die Kulturen des \enquote{Orients} mit
einer eurozentristischen Haltung betrachten, präsentieren und dadurch
überhaupt erst erschaffen. Mit Hilfe einer von der Foucault'schen
Diskursanalyse inspirierten Methode, die als koloniale Diskursanalyse
bezeichnet werden kann, skizzierte Said, wie der koloniale Diskurs dabei
nicht nur die kolonisierten Subjekte, sondern gleichzeitig die
Kolonisatoren hervorgebracht hat. Durch die Annahme der Überlegenheit
des Westens sei das auf \enquote{orientalistische} Weise produzierte
Wissen ein Instrument zur Aufrechterhaltung und Legitimierung kolonialer
Machtstrukturen. Dass nicht nur wirtschaftliche und politische Fragen
\enquote{Schlachtfelder} darstellen, drückte Edward Said in seinem
Folgewerk \emph{Culture and Imperialism} (1993) aus, in dem er die Macht
von Narrativen als einen der wesentlichen Verknüpfungspunkte von Kultur
und Imperialismus bezeichnete. Erzählungen und Sichtweisen haben also
normative Wirkung, wirken kulturell und damit auch gesellschaftlich
(fort). In kolonialen Diskursen wird durch verschiedene
Repräsentationspraktiken eine Form rassizifizierten Wissens über
nicht-europäische \enquote*{Andere} produziert, von denen sich das
\enquote*{Wir} der Europäer abgrenzt und dadurch seine Identität
konstituiert. Dabei implizieren diese Diskurse und Praktiken den
Anspruch, eine gleichsam \enquote{natürliche} europäische
Dominanzstellung aufrechtzuerhalten.

Eine zentrale Methode postkolonial kritischer Diskursanalyse besteht
daher in der Entlarvung von direkten und indirekten Zuschreibungen von
Selbst- und Fremdbildern in kolonialen Narrativen durch binäre
Zuschreibungen, die Subjekte und Objekte hierarchisieren und als
Rechtfertigung einer anhaltenden Hegemonie dienen sollen. Frantz Fanon
prägte in seinem Werk \emph{Les Damnés de la Terre} (1961) den Begriff
des kolonialen Manichäismus, der auf der Gegenüberstellung von
Unterdrückern und Unterdrückten entlang rassistischer
Stereotypisierungen beruht.\footnote{Fanon, Frantz: Les Damnés de la
  Terre, Paris: La Découverte, 2004, S. 45, 52, 89, 138.} Abdul
JanMohamed fasst die kolonialistische Mentalität zusammen als von einer
manichäischen Allegorie von Schwarz und Weiß, Gut und Böse, Heil und
Verdammung, Zivilisation und Wildheit, Überlegenheit und Unterlegenheit,
Intelligenz und Emotion, Selbst und Anderem, Subjekt und Objekt
bestimmten Haltung.\footnote{JanMohamed, Abdul R.: Manichean Aesthetics:
  The Politics of Literature in Colonial Africa, Amherst: University of
  Massachusetts Press, 1983, S. 4.}

In kolonialen Diskursen werden also in einer binären Opposition sowohl
Identitätszuschreibungen der Kolonisierten als auch der Kolonisierenden
transportiert. Das europäische Weiße Selbst konstituiert sich im Spiegel
eines rassifiziert essentialisieren Anderen. Während der
\enquote{Orientalismus} das \enquote*{Andere}/die \enquote*{Anderen}
studierte und später in der wissenschaftlichen Bearbeitung durch die
Black Studies vor allem in den USA in den 1960er und 1970er Jahren die
Wirkung von Rassismen auf Schwarze und People of Colour, aber auch
Aspekte des Widerstands und der Emanzipation untersucht wurden, wenden
die seit den 1990er Jahren bestehenden Critical Whiteness Studies den
Blick auf das rassifizierende Subjekt.

Susan Arndt stellt klar, dass \enquote{es nicht darum {[}geht{]},
ontologisierend oder essentialisierend die Existenz des
\enquote*{\emph{weißen} Menschen} oder einer \enquote*{\emph{weißen}
Kultur}, zu postulieren, vielmehr ist Weißsein als eine Konstruktion des
Rassismus zu lesen, die kollektive Wahrnehmungs-, Wissens- und
Handlungsmuster konstituiert hat. Damit präsentiert sich Weißsein als
eine historisch und kulturell geprägte symbolische und soziale Position,
die mit Macht und Privilegien einhergeht und sich auch unabhängig von
Selbstwahrnehmungen und jenseits offizieller Institutionen individuell
wie kollektiv manifestiert.}\footnote{Arndt, Susan: Mythen des weißen
  Subjekts: Verleugnung und Hierarchisierung von Rassismus, in: Eggers,
  Maureen M./Kilomba, Grada/Piesche, Peggy (Hrsg.): Mythen, Masken und
  Subjekte: Kritische Weißseinsforschung in Deutschland, Münster:
  Unrast, 2005, S. 343.}

Zentrales Anliegen der seit 2005 auch im und für den deutschsprachigen
Kontext betriebenen kritischen Weißseins-Forschung ist also das
Sichtbarmachen der Normierung, Naturalisierung und Dehistorisierung
Weißer Privilegien. Weißsein wird demnach als kritische Analysekategorie
herangezogen, um die Maßgeblichkeit und unterstellte Normalität des
\enquote*{Weißen}, demgegenüber Abweichungen \enquote*{unnormal} und
\enquote*{anders} ist, wahrzunehmen. Dabei ist die Frage nach Weißsein
aufgrund der inneren Zusammenhänge eng verknüpft mit der
Auseinandersetzung mit Rassismus, Kolonialismus und kolonialer
Geschichte. Arndt begreift \enquote{Rassismus {[}...{]} als Komplex von
Gefühlen, Vorurteilen, Vorstellungen, Ängsten, Phantasien und
Handlungen, mit denen Weiße aus einer weißen hegemonialen Position
heraus Schwarze und People of Color strukturell und diskursiv
positionieren und einem breiten Spektrum ihrer Gewalt aussetzen.
{[}Er{]} baut auf von Weißen in Europa entwickelten
\enquote*{Rassentheorien} auf, die den Anspruch auf Wissenschaftlichkeit
erhoben haben}. \footnote{Arndt, Susan: Ebenda, S. 341.}

Beim Sichtbarmachen geht es auch um das Benennen der eigenen Position,
denn: \enquote{Weil jeder von einem bestimmten Ort, einer bestimmten Zeit, aus
einer bestimmten Geschichte und Kultur sieht, spricht und schreibt, sind
auch Fragestellungen und Ergebnisse an die Position zurückgebunden, von
der aus sie formuliert sind}.\footnote{Strohschein, Juliane: weiße
  wahr-nehmungen: der koloniale blick, weißsein und fotografie,
  Diplomica: 2014, S. 7, Bezug nehmend auf Hall, Stuart (Hrsg.):
  Cultural Identity and Cinematic Representation, in: Framework 36:
  68--81, 1993, S. 68. So wie Juliane Strohschein geht es mir im
  Hinblick auf die Selbstverortung: Meine Positionierung als weiß, steht
  -- selbstverständlich -- in Zusammenhang mit der Art, in der ich
  Weißsein thematisiere.} Schließlich geht es um Hierarchien und
Positionen -- und damit um das Identifizieren von Macht in einem System,
in dem einige die Profiteur*innen auf Kosten anderer, der
Unterprivilegierten, sind.

\hypertarget{der-postkoloniale-blick-auf-die-rolle-aufgaben-und-funktionsweisen-der-institution-bibliothek}{%
\section{4. Der postkoloniale Blick auf die Rolle, Aufgaben und
Funktionsweisen der Institution
Bibliothek}\label{der-postkoloniale-blick-auf-die-rolle-aufgaben-und-funktionsweisen-der-institution-bibliothek}}

Im Rahmen einer postkolonial kritischen Analyse lassen sich nun die
verschiedenen Ebenen bibliothekarischer Funktionen und Funktionsweisen
in den Blick nehmen, und zwar unter der Fragestellung \emph{was} wird
\emph{wie} dargestellt?

\hypertarget{die-ebene-der-bestandsauswahl}{%
\subsection{4.1 Die Ebene der
Bestandsauswahl}\label{die-ebene-der-bestandsauswahl}}

Erwerb und Bestandsaufbau in Bibliotheken beinhaltet, selbst wenn diese
Selektion nach einem definierten Erwerbungsprofil erfolgt und damit
keineswegs willkürlich ist, mit jeder Entscheidung, einen Titel in den
Bestand aufzunehmen oder nicht, die Ausübung von Macht. Das in Frage
stehende Werk -- und damit das Thema, das es behandelt, und die
jeweilige Perspektive, die es vertritt -- bekommt entweder die Chance,
im auf dem Bibliotheksbestand basierenden Wissenssystem wahrgenommen und
rezipiert zu werden -- oder eben nicht.

Zur Illustration kann exemplarisch auf die Afrikastudien verwiesen
werden. Das Fach der Afrikanistik ist in besonderem Maße mit der
Herausforderung eines anhaltenden ideologisch erstarrten Wissens- und
Interpretationsmonopols des \enquote*{Nordens} über den \enquote*{Süden}
konfrontiert.\footnote{Wolff, H. Ekkehard: Die Dämmerung der alten
  weißen Männer -- Zur Kontroverse um den Afrikabeauftragten der
  Bundesregierung Günter Nooke, in: The Mouth. Critical Studies on
  Language, Culture and Society, 7.3.2019.} So wird seitens Beteiligter
im Wissenschaftssystem die Forderung artikuliert, diesem Missstand mit
einer \enquote{Afrikanisierung des Wissens} zu begegnen.\footnote{Falola,
  Toyin/Jennings, Christian (Hrsg.): Africanizing Knowledge: African
  Studies across the Disciplines (New Brunswick: Transaction Publishers,
  2002.} Allerdings stehen afrikanische Perspektiven im Zusammenhang mit
ihrer Rezeption in Europa vor dem Dilemma, dass dieses Wissen nur im
Rahmen genau jener konzeptuellen Sprache artikuliert und verständlich
vermittelt werden kann, die afrikanische Menschen verzerrt darstellt und
unterdrückt.\footnote{Macamo, Elísio: Urbane Scholarship: Studying
  Africa, Understanding the World, in: Africa, 88:1, 2018, S. 1--10.}
Insofern kann von einem multidimensionalen \emph{metropolitan bias}
gesprochen werden, der sich in der Unterrepräsentation der afrikanischen
Wissenschaft weltweit durch Disziplinen, Forschungsfelder und
Fachrichtungen niederschlägt.\footnote{Mama, Amina: Is It Ethical to
  Study Africa? Preliminary Thoughts on Scholarship and Freedom, in:
  African Studies Review, 50:1, 2007, S. 1--26. \emph{Bias} meint dabei
  unbewusste Tendenzen, Vorannahmen.} Darüber hinaus ist festzustellen,
dass die europäische Fixierung auf Schriftlichkeit zur Dokumentierung
und Tradierung von Wissen eine mangelnde Sichtbarkeit und Rezeption von
mündlichem Wissen zur Folge hat, was abermals die Einseitigkeit des
Narrativs verstärkt.

Digitalisierungsvorhaben wie das Projekt \enquote{Bewahrung und
Weitergabe kollektiver Erinnerung in Afrika: Afrikanische Zeugenberichte
und mündliche Literatur in der frühkolonialen Geschichte am Beispiel
Kameruns}\footnote{Ein Projekt der Fondation AfricAvenir International,
  Douala, in Kooperation mit dem Phonogrammarchiv an der
  Österreichischen Akademie der Wissenschaften in Wien mit Förderung der
  Gerda-Henkel-Stiftung:
  \url{https://www.gerda-henkel-stiftung.de/archivdouala}} unter der
Leitung von Kum'a Ndumbe III. können eine alternative
Geschichtsschreibung in Europa rezipierbar machen und damit als
Korrektiv für die eurozentristische Geschichtsschreibung fungieren.
Jedoch könnten solche Vorhaben in absehbarer Zukunft eine seltene
Ausnahme bleiben. Immerhin hat die International Federation of Library
Associations (IFLA) im Mai 2019 in einem Statement bezüglich indigenen
traditionellen Wissens deklariert, dass sie die Notwendigkeit anerkennt,
indigenes und lokales traditionelles Wissen im Interesse von indigenen
Völkern und dem Rest der Welt zu schützen.\footnote{IFLA, Statement on
  Indigenous Traditional Knowledge, 22.5.2019,
  \url{https://www.ifla.org/publications/ifla-statement-on-indigenous-traditional-knowledge}}
Würden die von der IFLA angedachten Maßnahmen zur Bewahrung und
Zugänglichmachung (mittels Digitalisierung und Zugänglichmachung im
Sinne des Open Access) tatsächlich flächendeckend durchgeführt werden,
wäre dies ein Baustein auf dem Weg zu einem Paradigmenwechsel, der die
Bibliothek als Reproduktionsstätte hauptsächlich Weißen Wissens hin zu
einer von Wissens- und Erzählungsvielfalt geprägten Plattform
transformiert.

\hypertarget{die-ebene-der-anordnung-und-kontextualisierung}{%
\subsection{4.2 Die Ebene der Anordnung und
Kontextualisierung}\label{die-ebene-der-anordnung-und-kontextualisierung}}

Im Hinblick auf die Ebene der Anordnung und Kontextualisierung von
Quellen durch Bibliotheken kann festgestellt werden, dass die
Wissensrepräsentation zunächst einmal, neutral betrachtet, zum Ziel hat,
Wissen derart abzubilden, dass es optimal gesucht und gefunden werden
kann,\footnote{Stock, Wolfgang G./Stock, Mechtild:
  Wissensrepräsentation: Informationen auswerten und bereitstellen,
  München: Oldenbourg, 2008, S. XI.} um so eine Brücke zwischen
Ressourcen und Nutzenden\footnote{Wiesenmüller, Heidrun: RSWK Reloaded
  -- Verbale Sacherschließung im Jahr 2018, in: BuB -- Forum Bibliothek
  und Information, 2.1.2018.} zu bauen.

Ein kritischer Blick offenbart jedoch, dass in herkömmlichen Methoden
der Wissensordnung in Bibliotheken -- also in der systematischen wie in
der verbalen Erschließung durch Klassifikationen und Schlagwortvergabe
-- unausweichlich Machtmechanismen wirken. So hat das in Beziehung
Setzen von Titeln im Prozess der Katalogisierung mittels einer
autoritären Hierarchisierung und mittels Assoziationen in Gegensätzen,
Synonymen oder Unterkategorien potenziell auch eine reduktionistische
und exkludierende Wirkung.\footnote{Adler, Melissa: Cruising the
  Library: Perversities in the Organization of Knowledge, New York:
  Fordham University Press, 2017, S. xii.} Laut Melissa Adler führt die
Universalität bibliothekarischer Wissensorganisationssysteme zu gewissen
\enquote{Blindheiten}. Sie bemerkt in ihrem 2017 erschienenen Werk
\emph{Cruising the Library -- Perversities in the Organization of
Knowledge}:

\enquote{The organization of unified subjects around a heteropatriarchal
universality that assumes whiteness inhibits analysis that interweaves
sexualities with racial and ethnic dimensions.}\footnote{Adler, Melissa:
  Ebenda., S. xvi.}

In eine ähnliche Richtung kritisiert Karin Aleksander in dem Artikel
\emph{\enquote{Die Frau im Bibliothekskatalog}} im Hinblick auf die
Vergabe von Schlagworten, dass stereotypische Geschlechterrollenmodelle
die unbewusste Basis für die Ansetzungen bilden.\footnote{Aleksander,
  Karin: Die Frau im Bibliothekskatalog, in: LIBREAS. Library Ideas, 25,
  2014.} Nach ihrer Bewertung ist

\enquote{{[}d{]}iese Unterordnung von Frauenaspekten unter
Männerallgemeinheiten {[}...{]} am schwersten zu durchschauen und
deshalb auch nur mit wachsender Erkenntnis, Überzeugung und
Voranschreiten der gesellschaftlichen Entwicklung zu
verändern.}\footnote{Aleksander, Karin: Ebenda.}

Um die große Herausforderung einer systemischen Transformation zu einer
echten Gleichberechtigung und Gleichwertigkeit der Geschlechter zu
unterstreichen, zitiert sie Pierre Bourdieu:

\enquote{Wenn es darum geht, die soziale Welt zu denken, kann man die
Schwierigkeiten bzw. Risiken gar nicht hoch genug veranschlagen. Die
Macht des Präkonstruierten liegt darin, daß es zugleich in die Dinge und
in die Köpfe eingegangen ist und sich deshalb mit einer Scheinevidenz
präsentiert, die unbemerkt durchgeht, weil sie selbstverständlich ist.
Der Bruch ist eigentlich eine Konversion des Blicks, und vom Unterricht
in soziologischer Forschung kann man sagen, daß er zuallererst lehren
muß,}mit anderen Augen zu sehen'' \ldots{} Und das ist nicht möglich
ohne eine echte Konversion, eine metanoia, eine mentale Revolution,
einen Wandel der ganzen Sicht der sozialen Welt.

Was man den \enquote*{epistemologischen Bruch} nennt, also die
vorübergehende Außerkraftsetzung der gewöhnlichen Präkonstruktionen und
der gewöhnlich bei der Realisierung dieser Konstruktionen angewandten
Prinzipien, setzt oft einen Bruch mit den Denkweisen, Begriffen,
Methoden voraus, die allen Anschein des common sense, der gewöhnlichen
Alltags- und Wissenschaftsvernunft (also alles dessen, was die
herrschende positivistische Disposition honoriert und anerkennt) für
sich haben.''\footnote{Aleksander, Karin: Ebenda., mit Bezug auf Pierre
  Bourdieu/Wacquant, Loïc J. D.: Reflexive Anthropologie, Frankfurt am
  Main: Suhrkamp, 1996, S. 284f.}

Die Erkenntnis über die tiefsitzende Verankerung von Vorurteilen, die
Aleksander in Bezug auf die Kategorie Gender konstatiert, lässt sich bei
einer postkolonial kritischen Bibliothekskatalog-Lesart auch auf
Kategorien wie \enquote{Rasse} oder Hautfarbe übertragen. In dem
Dreischritt Erkennen -- Benennen -- Verändern auf dem Weg zu einer
weniger diskriminierenden Wissensordnung ist sie nur die erste Etappe.

\hypertarget{bibliothek-als-abbild-der-wissenschaft}{%
\subsection{4.3 Bibliothek als Abbild der
Wissenschaft}\label{bibliothek-als-abbild-der-wissenschaft}}

Melissa Adler stellt den Zusammenhang zwischen bibliothekarischer
Wissensorganisation und nationaler Identität heraus, indem sie die
Library of Congress-Klassifikation als eine \enquote{history of
nation-building} interpretiert, in die rassifizierte und sexualisierte
Subjekte eingeschrieben oder nicht eingeschrieben werden.\footnote{Adler,
  Melissa: am angegebenen Ort, S. xvi, xvii.}

Dabei werden, wie bereits weiter oben (unter 2.) mit Referenz zu Konrad
Umlaufs diesbezüglicher Äußerung angeführt, die historischen Dimensionen
der Kontextualisierung von Titeln durch Katalogisierung kaum markiert.
Tanja Heber geht in ihrem schon zitierten Buch \emph{Die Bibliothek als
Speichersystem des kulturellen Gedächtnisses} auf die Interdependenz
zwischen Staat und Bibliothek ein, indem sie darauf verweist, dass der
Aufbau von Staat und Verwaltung die Grundlage für die Organisation
bibliothekarischer Arbeit bildet.\footnote{Heber, Tanja: am angegebenen
  Ort, S. 13.}

Mit ihrem Ansatz der Einteilung, Strukturierung, Normierung,
Klassifizierung, Domestizierung und Disziplinierung der Welt spiegelt
die Bibliothek als Einrichtung die aufklärerischen Ideale von
Wissenschaft und Wissenschaftlichkeit wider.

Sebastian Conrad und Shalini Randeria beschreiben in ihrem Buch
\emph{Jenseits des Eurozentrismus: Postkoloniale Perspektiven in den
Geschichts- und Kulturwissenschaften} die Rolle und Wirkung des modernen
Wissens im kolonialen Projekt dahingehend, dass es

\enquote{(...) nicht nur Instrument und Waffe {[}war{]}, sondern selbst
Produkt eines Kontextes diskursiver Praktiken. Die kulturellen und
sozialen Zusammenhänge der kolonialen Epoche hatten daher in den
Produkten der europäischen Wissensordnung ihre Spuren
hinterlassen.}\footnote{Conrad, Sebastian/Randeria, Shalini: Jenseits
  des Eurozentrismus: Postkoloniale Perspektiven in den Geschichts- und
  Kulturwissenschaften, Frankfurt am Main: Campus Verlag, 2013.}

Insofern besteht eine enge Wechselbeziehung zwischen Kolonialismus und
Wissenschaft: Die Wissensproduktion des Westens lässt sich in den
Kontext der gewaltvollen kolonialen Erfahrung einordnen, die dieses
Wissen produziert und strukturiert hat.\footnote{Strohschein, Juliane:
  am angegebenen Ort, S. 42.} Was für die biologische und medizinische
sogenannte Rassenforschung mittlerweile weitläufig bekannt ist, gilt
aber für eine Vielzahl weiterer Disziplinen wie zum Beispiel die
Regionalstudien,\footnote{Deren Vorläufer waren die sogenannten
  Kolonialwissenschaften, wie sie zum Beispiel an der Londoner School of
  Oriental and African Studies betrieben wurden, siehe Kwaschik, Anne:
  Der Griff nach dem Weltwissen: Zur Genealogie von Area Studies im 19.
  und 20. Jahrhunder\emph{t}, Göttingen: Vandenhoeck \& Ruprecht, 2018.}
Ethnologie, Geographie, Recht, Philosophie.

Diese Verwobenheit der Wissenschaft(en) mit dem Kolonialismus analysiert
Wael Hallaq in seinem grundlegenden Werk \emph{Restating Orientalism},
in dem er die Position des säkularen westlichen Selbst im Projekt der
Moderne und den Problemkomplex von Macht und Wissen neu untersucht. Er
liefert darin letztlich eine globale Kritik der Wissenschaft an sich,
indem er \enquote{the depth of academia's lethal complicity in modern
forms of capitalism, colonialism, and hegemonic power}
aufzeigt.\footnote{Hallaq, Wael B.: Restating Orientalism: A Critique of
  Modern Knowledge, New York: Columbia University Press, 2018.}

Als Dienstleister und Teil der Wissenschaft sind folglich auch die
Bibliothek und die Bibliotheks- und Informationswissenschaft keine
neutralen Instanzen im (post)kolonialen Gefüge.

\hypertarget{uxfcberblick-uxfcber-derzeitige-entwicklungen-zu-postkolonialen-fragestellungen-im-bid-bereich}{%
\section{5. Überblick über derzeitige Entwicklungen zu
postkolonialen Fragestellungen im
BID-Bereich}\label{uxfcberblick-uxfcber-derzeitige-entwicklungen-zu-postkolonialen-fragestellungen-im-bid-bereich}}

\hypertarget{koloniale-artefakte-in-bibliotheken}{%
\subsection{5.1 Koloniale Artefakte in
Bibliotheken}\label{koloniale-artefakte-in-bibliotheken}}

Möchte man Spuren von kolonialem Erbe in deutschen Bibliotheken
ergründen, liegt es nahe, zunächst nach schriftlichen Artefakten
kolonialen Ursprungs zu fragen.

Ein seltener Fall eines als koloniales Sammlungsgut anerkannten Buches,
das restituiert wurde, ereignete sich im Februar 2019 im namibischen
Gibeon: Die Wissenschaftsministerin von Baden-Württemberg Theresia Bauer
gab dort in einem feierlichen Akt die \enquote{Witbooi-Bibel}, zusammen
mit der \enquote{Witbooi-Peitsche}, an den Staat Namibia
zurück.\footnote{Thiemeyer, Thomas/von Bernstorff, Jochen:
  Südwestdeutsch trifft Deutsch-Südwest. Baden-Württemberg gibt zwei
  kolonialzeitliche Objekte an Namibia zurück, Merkur, 2.5.2019.} Beide
Objekte gehörten Hendrik Witbooi, dem Anführer des Nama-Aufstands gegen
die deutschen von General von Trotha geführten \enquote{Schutztruppen}
im Jahre 1904.\footnote{von Bernstorff, Jochen / Schuler, Jakob:
  Restitution und Kolonialismus: Wem gehört die Witbooi-Bibel?,
  Verfassungsblog, 4.3.2019.} Er fiel dem Genozid der Deutschen an den
Herero und Nama zum Opfer. Sein Regierungssitz war schon vorher, im Jahr
1893, geplündert worden, und der \enquote{kommissarische Intendant für
die Schutztruppe und Chef der Finanzverwaltung} in
\enquote{Deutsch-Südwestafrika}, Hofrat von Wassmannsdorf, überließ die
geplünderten Gegenstände aus dem Besitz der Witbooi-Familie dem
Linden-Museum Stuttgart.\footnote{Ebenda; Pressemitteilung: 'Wichtiges
  Zeichen der Versöhnung', offizielle Internetseite des Landes
  Baden-Württemberg, 10.12.2018,
  \url{https://www.baden-wuerttemberg.de/de/service/presse/pressemitteilung/pid/wichtiges-zeichen-der-versoehnung/}
  (zugegriffen am 15.6.2019).} Die späte Rückgabe blieb bislang ein in
ihrer Symbolkraft und politischen Prominenz singuläres Ereignis.

Recherchen zu der vorliegenden Arbeit zu etwaigen schriftlichen
Artefakten im Bestand der Staatsbibliothek zu Berlin ergaben, dass sich
zumindest im Hauptbestand keine solche Objekte befinden. Da sich jedoch
in den Sonderabteilungen eine Vielzahl bisher unerschlossener
Materialien befindet, ist nicht auszuschließen, dass auch Artefakte aus
kolonialen Kontexten darunter sind. Diese unklare, noch unerforschte
Situation dürfte auch für andere Häuser und Sammlungen gelten.

Der bereits erwähnte Savoy/Sarr-Bericht nennt ebenfalls Bibliotheken als
Lagerorte für schriftliche Artefakte aus kolonialen
Kontexten.\footnote{Am angegebenen Ort.} So fanden sich beispielsweise
wertvolle Handschriften, die Frankreich während seiner
Kolonialherrschaft in Mali geraubt hatte, in der Bibliothèque nationale
de France in Paris.\footnote{Dramiga, Joe: Die
  Sankoré-Schriften,\,SciLogs, 19.8.2010.} Der Verbleib einer Vielzahl
von Manuskript-Schätzen in der alten Handelsstadt und dem
mittelalterlichen Gelehrtenzentrum Timbuktu ist dem Umstand zu
verdanken, dass Notabeln und Gelehrte im 17. Jahrhundert unter
marokkanischer Besatzung und im 20. Jahrhundert unter französischer
Okkupation ihre Bibliotheken versteckten oder sie zum Schutz vor
Schändung und Raub einmauerten.\footnote{Dramiga, Joe: Ebenda; zum
  Erhalt der wertvollen Schriften gibt es mehrere Projekte
  internationaler Kooperation wie das Timbuktu Manuscripts Project,
  \url{https://web.archive.org/web/20060505134134/http://www.sum.uio.no/research/mali/timbuktu/project/timanus.pdf}}

Dass es auch Fälle gibt, in denen schriftliche Artefakte \emph{nach} dem
Ende der Kolonialzeit auf zweifelhaftem oder heute gar bekanntermaßen
illegalem Wege in europäische Sammlungen gekommen sind, bezeugt die
Existenz von Diebesgut des niederländischen Linguisten Jan Knappert in
der Sammlung der Londoner School of Oriental and African Studies (SOAS).
Der Experte für Bantusprachen hatte bei Recherchen in Kenia in privaten
Literaturbeständen aus dem 17. und 18. Jahrhundert einzelne Seiten
entwendet, die jetzt -- nicht einmal mit einem entsprechenden Vermerk --
als Knappert-Nachlass präsentiert werden.\footnote{Biersteker,
  Ann/Plane, Mark: Swahili Manuscripts and the Study of Swahili
  Literature, in: Research in African Literatures, Vol. 20, No.~3
  (Autumn, 1989), S. 449--472, 455; Allen, James de Vere: Review of Four
  Centuries of Swahili Verse: A Literary History and Anthology by Jan
  Knappert, in: Research in African Literatures, Vol. 13, No. 1, Special
  Issue on Nigerian Literature (Spring, 1982), S. 141--148, 142: Vom
  Namen Knappert wurde der Swahili-Begriff \enquote{kinaperti}
  abgeleitet, \enquote{meaning someone who asks to borrow something and
  then goes round telling everyone it is his and even selling it for his
  own gain}, \url{https://digital.soas.ac.uk/knappert}} Dies ist auch
insofern frappierend, als es sich bei diesen unrechtmäßigen Aneignungen
um Vorgänge aus dem späten 20. Jahrhundert handelt.

Auch im Zuge des 2004 von der British Library ins Leben gerufenen
\emph{Endangered Archives Programme} (EAP), das die Bewahrung bedrohten
schriftlichen Kulturguts durch Digitalisierung und
Open-Access-Veröffentlichung auf dem EAP-Repositorium zum Ziel
hat,\footnote{\url{https://eap.bl.uk}} sollen laut informierten
Kreisen\footnote{Dies erfuhr die Autorin in einem zur Vorbereitung der
  Recherchen zur vorliegenden Untersuchung geführten
  Expert*inneninterview.} historische Originale abhanden gekommen sein.
Die Manuskripte waren für die Nutzung in Projekten im Rahmen des
Programms nach London verschickt worden, um dort von Expert*innen
bearbeitet zu werden. Aufgrund von \enquote*{verschwundenen} Originalen
und den daraus resultierenden Verwerfungen und dem Reputations- sowie
Vertrauensverlust wurden die Digitalisierungsmodalitäten geändert: Die
Original-Archivalien verlassen nunmehr nicht ihr Herkunftsland, sondern
es werden Kameras und Scanner dorthin geschickt, für deren sachgemäße
Bedienung lokales Personal geschult wird.\footnote{\url{https://eap.bl.uk/about}}

\hypertarget{umgang-mit-altbestuxe4nden-aus-der-kolonialzeit-digitalisierungsprojekte}{%
\subsection{5.2 Umgang mit Altbeständen aus der Kolonialzeit --
Digitalisierungsprojekte}\label{umgang-mit-altbestuxe4nden-aus-der-kolonialzeit-digitalisierungsprojekte}}

Was Altbestände aus der Kolonialzeit in deutschen BID-Einrichtungen
betrifft, ist im Hinblick auf die staatliche Ebene das Bundesarchiv mit
seiner Fülle von Material und Aspekten zu nennen. So hat es zum einen
unter dem Titel \enquote{Grenzexpedition und Völkermord -- Quellen zur
Kolonialgeschichte} ein Portal digitalisierter Quellen
eingerichtet.\footnote{Bundesarchiv, \enquote{Grenzexpedition und
  Völkermord -- Quellen zur Kolonialgeschichte},
  \url{https://www.bundesarchiv.de/DE/Navigation/Entdecken/Kolonialgeschichte/kolonialgeschichte.html}}
Zum anderen engagiert es sich in einem Kooperationsprojekt mit dem
Goethe-Institut Kamerun um die Sicherung und Nutzbarmachung von Akten
der deutschen Kolonialverwaltung.\footnote{Bundesarchiv,
  \enquote{Zwischen Bestandserhaltung und Bühnennebel - Deutsche
  Kolonialakten in Kamerun},
  \url{https://www.bundesarchiv.de/DE/Content/Artikel/Ueber-uns/Aus-unserer-Arbeit/Textsammlung-Kamerun/kamerun.html?chapterId=38386}}

Das Pressearchiv des Hamburgischen Welt-Wirtschafts-Archivs (HWWA) ist
keine staatliche, für das Untersuchungsfeld aber ebenso relevante
Einrichtung. Seine Geschichte begann 1908 mit der Errichtung des
Hamburgischen Kolonialinstituts, für dessen Pressedokumentation es
zuständig war.\footnote{\url{http://webopac.hwwa.de/digiview/docs/hwwa.cfm}}
Die Pressearchive von HWWA sowie dem 1914 ebenfalls in Hamburg
gegründeten Institut für Weltwirtschaft wurden im Rahmen des Projekts
\enquote{Pressemappe 20. Jahrhundert} digitalisiert und sind online
verfügbar.\footnote{\url{http://webopac.hwwa.de/digiview/docs/forschungs.cfm}}

Zu den digitalisierten Materialien zählen auch die Bestände der 1934 vom
Hamburgischen Landesverband des Volksbundes für das Deutschtum im
Ausland (VDA) eingerichteten und 1945 aufgelösten Forschungsstelle für
das Übersee-Deutschtum, darunter ein Zeitungsarchiv, eine
Handbibliothek, Forschungs- und Prüfungsarbeiten und Vorträge.

Im Bereich der Universitätsbibliotheken lässt sich die abgeschlossene
Digitalisierung einer Kollektion kolonialer Bildpostkarten an der
Universitäts- und Stadtbibliothek Köln anführen, die auf einem dafür
eingerichteten Sammlungsportal für Forschungszwecke online zur Verfügung
steht.\footnote{Kolonialismus und afrikanische Diaspora auf
  Bildpostkarten, Sammlungsportal der Universitäts- und Stadtbibliothek
  Köln,
  \url{https://www.ub.uni-koeln.de/sammlungen/bildpostkarten/index_ger.html}}

Ein aufgrund seines beachtlichen Umfangs und seiner vielseitigen
Bestandteile besonders bedeutsames Projekt ist die im Frühjahr 2019
abgeschlossene \enquote{Digitale Sammlung Deutscher Kolonialismus}
(DSDK). Das Vorhaben umfasste die Digitalisierung und elektronische
Erschließung von über 1.000 Quellen zur deutschen Kolonialgeschichte aus
Beständen der Staats- und Universitätsbibliothek (SuUB) Bremen und der
Universitätsbibliothek (UB) Frankfurt. Die Quellen sind jetzt als
Volltexte im Open Access verfügbar und aufgrund einer Kooperation mit
der Berlin-Brandenburgischen Akademie der Wissenschaften (BBAW) zudem in
die virtuelle Forschungsinfrastruktur CLARIN-D integriert.\footnote{\url{https://www.suub.uni-bremen.de/ueber-uns/projekte/dsdk/}}
Die Auswahl der digitalisierten Quellen wurde auf Grundlage des --
ebenfalls digitalisierten -- historischen Bandkatalogs
\enquote{Kolonialwesen} der SuUB Bremen\footnote{Bandkatalog
  "Kolonialwesen" der SuUB Bremen, 1906 bis ca. 1940,
  \url{http://brema.suub.uni-bremen.de/urn/urn:nbn:de:gbv:46:1-8837}
  (zugegriffen am 15.6.2019).} getroffen, der in einer weiterführenden
Publikation\footnote{Müller, Maria Elisabeth/Schmidt-Brücken, Daniel
  (Hrsg.): Der Bremer Bandkatalog \enquote{Kolonialwesen}: Edition,
  Sprachwissenschaftliche und bibliotheksgeschichtliche Kommentierung,
  Berlin: De Gruyter, 2017.} wissenschaftlich aufgearbeitet wurde.

Für den Sammlungsbereich Afrika pflegt die Vereinigung für
Afrikawissenschaften in Deutschland eine Liste von Institutionen mit
entsprechenden Materialkollektionen im Schwerpunkt. Allerdings befinden
sich dort nicht nur Referenzen für Archive und Bibliotheken mit
ausschließlich \emph{kolonial}historischen Quellen.\footnote{\url{https://vad-ev.de/afrika-archive-und-bibliotheken/}}

Als erwähnenswertes Beispiel für ein kleineres Projekt zu Altbeständen
aus kolonialem Kontext, das zwar primär an einer Bibliothek angesiedelt
ist, aber doch mit Dokumenten aus einer Universitätsbibliothek arbeitet,
kann das neuzeitliche Geschichtsseminar \enquote{Kiel, Kamerun und
zurück: Ein Projekt zur Digitalisierung kolonialer Reiseberichte}
gelten, das im Sommersemester 2019 an der
Christian-Albrechts-Universität stattfand und sich mit Quellen aus der
UB Kiel auseinandersetzte .\footnote{Unter der Leitung der Dozentinnen
  Dr.~Swantje Piotrowski, M.A., Carolin Liebisch-Gümüs, M.A.}

\hypertarget{bid-praxis-gremien-zu-kolonialer-provenienzforschung-und-fragestellungen-fuxfcr-die-bibliothekarische-praxis}{%
\subsection{5.3 BID-Praxis: Gremien zu kolonialer Provenienzforschung und
Fragestellungen für die bibliothekarische
Praxis}\label{bid-praxis-gremien-zu-kolonialer-provenienzforschung-und-fragestellungen-fuxfcr-die-bibliothekarische-praxis}}

Ähnlich wie in Museen ist das Thema Provenienzforschung auch für
Bibliothekar*innen schon seit geraumer Zeit relevant. So stand beim
ersten von dem Arbeitskreis \emph{Provenienzforschung e.V.} --
hauptsächlich eine Interessenvertretung der Provenienzforscher*innen in
den Museen -- organisierten \enquote{Tag der Provenienzforschung} am 10.
April 2019 auch das Thema koloniale Kontexte auf der
Tagesordnung.\footnote{\url{https://www.arbeitskreis-provenienzforschung.org/tag-der-provenienzforschung-2021/tag-der-provenienzforschung-2019/};
  beim Arbeitskreis Provenienzforschung e.V. gibt es mehrere
  Arbeitsgruppen, darunter auch eine Arbeitsgruppe \emph{Koloniale
  Provenienzen}, diese existiert seit 2017 ins Leben gerufen wurde und
  ist aus der oben erwähnten Münchner Tagung zur Provenienzforschung
  hervorgegangen ist.} Beim Deutschen Bibliotheksverband (dbv) ist die
Kommission \emph{Provenienzforschung und Provenienzerschließung}
Ansprechpartnerin für alle Fragen rund um die Herkunft von
Bibliotheksbeständen -- bislang vornehmlich für den Kontext NS-Raubgut,
darüber hinaus auch für kriegsbedingt verlagerten Kulturgüter (Beutegut)
und Kulturgutverluste während der sowjetischen Besatzung und in der DDR
(zum Beispiel durch die Bodenreform). Sie vertritt diesbezügliche
bibliothekarische Belange nach außen.\footnote{\url{https://www.bibliotheksverband.de/kommissionen\#Provenienzforschung\%20und\%20Provenienzerschliessung}
  ihren Vorsitz hat derzeit Michaela Scheibe von der Staatsbibliothek zu
  Berlin inne, ebenda.} Außerdem fungiert die Kommission als
Geschäftsstelle des Arbeitskreises \emph{Provenienzforschung und
Restitution -- Bibliotheken} (APR-Bib). Letzterer, ein Zusammenschluss
von im bibliothekarischen Bereich tätigen Provenienzforscher*innen,
trifft sich regelmäßig zum Erfahrungsaustausch und erarbeitet Vorschläge
für die Umsetzung bibliothekarischer Anliegen~ im Zusammenhang mit der
Recherche nach NS-Raubgut.\footnote{\url{https://www.bibliotheksverband.de/provenienzforschung-und-provenienzerschliessung}}
Das Themenfeld Koloniale Provenienz ist relativ neu auf der Agenda des
dbv.

Ein Positionspapier zum Umgang mit Kulturgut aus kolonialen Kontexten in
Bibliotheken, das dem \emph{Leitfaden Professionell Arbeiten im Museum}
des Deutschen Museumsbundes,\footnote{Deutscher Museumsbund:
  Professionell arbeiten im Museum, Berlin: Deutscher Museumsbund, 2019,
  \url{https://www.museumsbund.de/wp-content/uploads/2020/01/dmb-leitfaden-professionell-arbeiten-online.pdf}}
in welchem unter anderem die Provenienzforschung im Bezug auf koloniale
Bestände geregelt wird, vergleichbar ist, gibt es bislang nicht. Ein
innerverbandlicher Prozess seitens des dbv ist jedoch bereits im
Gang.\footnote{Sammlungsgut aus Kolonialen Kontexten -- Stellungnahme des
  Deutschen Kulturrates, 20.2.2019, S. 6
  \url{https://www.kulturrat.de/wp-content/uploads/2019/03/Dossier-Kolonialismus.pdf}
  (zugegriffen am 15.6.2019); es gab im Frühjahr 2019 eine erste Umfrage
  zu Kulturgut aus kolonialen Kontexten in Bibliotheken, die von
  Prof.~Dr.~Thomas Bürger als Mitglied im Ausschuss Kulturerbe des
  Deutschen Kulturrats initiiert wurde und in Kooperation mit dem dbv
  durchgeführt wurde und in der es zunächst um Erkenntnisse geht, in
  welcher Form und in welchem Umfang derartige Bestände vorhanden und
  bekannt sind.} Zur Beantragung der Fördermittel, die das DZK nach
seinen neuen Förderrichtlinien seit Januar 2019 erstmals für Projekte
zur Verfügung stellt, die sich der Aufarbeitung von Provenienzen von
Kulturgut aus kolonialen Kontexten widmen, sind neben Museen auch
Bibliotheken und Archive in Deutschland berechtigt.\footnote{Dies wurde
  auch in der Bibliotheks-Community kommuniziert, siehe zum Beispiel
  \url{https://bibliotheksportal.de/2019/02/05/foerdermoeglichkeiten-2019/};
  \url{https://web.archive.org/web/20210506195011/https://www.bibliotheksverband.de/datensaetze/newsletter-national/dbv-newsletter-nr-139-2019-07-februar.html}}
Projekte zu Kulturgut aus kolonialen Kontexten in Bibliotheken gibt es
momentan noch nicht,\footnote{Auskunft von Michaela Scheibe, Vorsitzende
  der dbv-Kommission Provenienzforschung.} aber es ist anzunehmen, dass
in absehbarer Zukunft Förderanträge gestellt, bearbeitet und
entsprechende Mittel bewilligt werden, so dass es nur eine Frage der
Zeit ist, bis auch bibliothekarische Provenzienforschungsprojekte
anlaufen.

Als weiterer Aspekt der Beschäftigung einer eventuell kolonial geprägten
Herkunft von Sammlungsmaterial ist die Markierung der Provenienz in
Bibliothekskatalogen und sonstigen -daten\-banken festzuhalten. Laut
Michaela Scheibe, der Vorsitzenden der dbv-Kommission
Provenienzforschung, \enquote{{[}erlaubt{]} die etablierte Form der
Provenienzverzeichnung für Buchbestände auch die Verzeichnung der
Exemplargeschichte bei kolonialen Kontexten (sowohl in den
Katalogdatenbanken als auch im von ihr federführend unterhaltenen
ProvenienzWiki). Damit ist der jeweilige Einzelfall abgedeckt, ohne dass
eine pauschalisierte Zuordnung zum kolonialen Kontext erfolgt. Was
derzeit noch diskutiert werden muss, ist die Aufnahme eines
entsprechenden Deskriptors in den Thesaurus der Provenienzbegriffe --
T-PRO (analog zu NS-Raubgut, Enteignung, Bodenreform\ldots). In diesem
Zusammenhang muss auch definiert werden, was unter diesen Begriff im
Rahmen der Bibliotheksbestände fallen soll.}\footnote{Ebenda, Auskunft
  per E-Mail, ebenso wie die Information, dass die Aufnahme eines
  Deskriptors in den T-PRO Ende August 2019 auf der darauf folgenden
  Sitzung der dbv-Kommission Provenienzforschung und
  Provenienzerschließung diskutiert werden sollte. {[}Stand aus der
  Masterarbeit, auf der dieser Text basiert.{]}}

\hypertarget{bid-theorie-veruxf6ffentlichungen-zu-postkolonialen-themen-im-lis-bereich}{%
\subsection{5.4 BID-Theorie: Veröffentlichungen zu postkolonialen Themen
im
LIS-Bereich}\label{bid-theorie-veruxf6ffentlichungen-zu-postkolonialen-themen-im-lis-bereich}}

Zu einer Bestandsaufnahme zu postkolonialen Fragestellungen im
Bibliotheksbereich gehört auch ein Blick auf das Feld der aktuellen
Veröffentlichungen zu entsprechenden Themen in den Bibliotheks- und
Informationswissenschaft. Die Suche nach deutschsprachigen
LIS-Publikati\-onen mit einem postkolonial kritischen Ansatz bestätigt den
Befund einer bislang eher geringen Auseinandersetzung mit dem
Themenfeld. Dies betrifft den Bereich der Monographien, aber auch eine
diesbezügliche Recherche in Fachzeitschriften\footnote{Exemplarisch
  durchsucht wurden für die vorliegende Untersuchung die Archive der
  Zeitschriften \emph{Bibliothek: Forschung und Praxis und BuB -- Forum
  Bibliothek und Information.}} liefert kaum nennenswerte Treffer.
Erwähnenswert ist für den deutschsprachigen Bereich insbesondere der
Sammelband \emph{Koloniale Spuren in den Archiven der
Leibniz-Gemeinschaft},\footnote{Brogiato, Heinz Peter/Röschner, Matthias
  (Hrsg.): Koloniale Spuren in den Archiven der Leibniz-Gemeinschaft,
  Halle: Mitteldeutscher Verlag 2020.} in dem Archivmitarbeiter*innen
markante Beispiele an Schrift- und Bildquellen aus ihren Beständen aus
kolonialen Kontexten oder mit kolonialen Bezügen vorstellen.

Ebenso werden die englischsprachigen Titel hierzulande lediglich
punktuell rezipiert. Am ehesten werden Bücher vom US-amerikanischen
Verlag Library Juice Press, der seinen Schwerpunkt auf kritische Werke
in Bibliotheks- und Informationswissenschaft gelegt hat, in Deutschland
von der Zeitschrift LIBREAS. Library Ideas rezensiert.\footnote{\url{https://libreas.eu/}}
Dabei ist auch die Anzahl der englischsprachigen Veröffentlichungen
überschaubar: In der Reihe \emph{Series on Critical Race Studies and
Multiculturalism in LIS} von Library Juice Press sind aktuell vier Titel
erschienen, drei weitere sind in Vorbereitung.\footnote{\url{https://litwinbooks.com/series-on-critical-race-studies-and-multiculturalism-in-lis/}}
Vom in der Serie publizierten Buch \emph{Teaching for Justice --
Implementing Social Justice in the LIS Classroom}\footnote{Cooke, Nicole
  A./Sweeney, Miriam E. (Hrsg.): Teaching for Justice: Implementing
  Social Justice in the LIS Classroom, Sacramento: Library Juice Press,
  2017; immerhin verweist die Suche nach der Person der erstgenannten
  Herausgeberin auf eine Ausgabe zu Race and Ethnicity in Library and
  Information Science: An Update der Zeitschrift Library Trends 67 (1)
  Summer 2018, die wiederum an eine dort mit der Ausgabe Ethnic
  Diversity in Library and Information Science, herausgegeben von
  Kathleen de la Peña McCook, Library Trends 49 (1) Summer 2000,
  begonnene Debatte anschließt.} existieren derzeit (Stand 21.12.2021)
laut dem Karlsruher Virtuellen Katalog (KVK) ein Exemplar in
Bibliotheken in Deutschland. Das Werk \emph{Topographies of Whiteness --
Mapping Whiteness in Library and Information Science}\footnote{Schlesselman-Tarango,
  Gina (Hrsg.): Topographies of Whiteness: Mapping Whiteness in Library
  and Information Science\emph{,} Sacramento: Library Juice Press, 2017.}
ist gedruckt im Bestand von drei Bibliotheken nachgewiesen. Schließlich
ist der Titel \emph{Pushing the Margins: Women of Color and
Intersectionality in LIS}\footnote{Chou, Rose/Pho, Annie (Hrsg.):
  Pushing the Margins: Women of Color and Intersectionality in LIS,
  Sacramento: Library Juice Press, 2018.} laut KVK dreimal in deutschen
Bibliotheken im Printformat verfügbar. Der Titel \emph{Borders and
Belonging: Critical Examinations of Library Approachestoward
Immigrants}\footnote{Ndumu, Ana: Borders and Belonging: Critical
  Examinations of Library Approaches toward Immigrants, Sacramento:
  Library Juice Press, 2021.} ist bislang an fünf Einrichtungen
verfügbar. Im Frühling 2022 soll in der Serie der Band \emph{Everywhere
and Nowhere - Understanding Diaspora in the Library}\footnote{Moreno,
  Teresa Helena: Everywhere and Nowhere: Understanding Diaspora in the
  Library, Sacramento: Library Juice Press, 2022 (im Erscheinen).}
erscheinen. Ebenfalls im laufenden Jahr ist bei MIT Press der im Open
Access verfügbare Sammelband \emph{Knowledge Justice: Disrupting Library
and Information Studies through Critical Race Theory}\footnote{Leung,
  Sofia Y./López-McKnight, Jorge R.: Knowledge Justice: Disrupting
  Library and Information Studies through Critical Race Theory,
  Cambridge, Massachusetts: The MIT Press, 2021.} erschienen, der seine
Agenda bereits im Titel auf seine Fahnen geschrieben hat. Ebenfalls im
Open Access verfügbar ist die noch nicht bei einem Verlag erschienene
englischsprachige MA (LIS)-Dissertation der deutschen Autorin Nora
Schmidt mit dem Titel \emph{The privilege to select: global research
system, European academic library collections, and
decolonisation}.\footnote{Schmidt, Nora: The privilege to select: global
  research system, European academic library collections, and
  decolonisation, 2020,
  \url{https://portal.research.lu.se/ws/files/83315048/diss_master_print_final.pdf}}

Auch wenn man sich die Curricula von LIS-Studiengängen in Deutschland
anschaut, gelangt man zu der Feststellung: Die Vermittlung einer
postkolonial kritischen Perspektive auf die Disziplin, bibliothekarische
Berufe und die Rolle der Bibliotheken in Gesellschaft und
Wissenschaftssystem ist derzeit nicht vorgesehen.

\hypertarget{befunde-der-bibliotheksumschau}{%
\section{6. Befunde der
Bibliotheksumschau}\label{befunde-der-bibliotheksumschau}}

Im Rahmen der vorangegangenen Bibliotheksumschau zu kolonialen Kontexten
wurde deutlich, dass die Sensibilisierung für das deutsche koloniale
Erbe in der BID-Welt bei Weitem noch nicht so hoch ist wie bei den
Museen. Dass postkolonialen Fragestellungen in Bibliotheken
vergleichsweise geringere Aufmerksamkeit zuteilwird, liegt vermutlich in
erster Linie daran, dass es hier weniger -- und weniger prominente --
kolonial sensible Artefakte gibt. Das Problem präsentiert sich also
weniger akut. Jedoch umfasst der Problembereich nicht nur Objekte und
Provenienzen, sondern auch eine ganze Bandbreite von Wissenspraxen,
Regelwerken und der bibliothekarischen Arbeit selbst.

Auch dahingehend zeigt die vorliegende Untersuchung, dass auf diesem
historisch kritischen Feld noch Entwicklungspotenzial und -bedarf liegt,
um kolonial konnotierte Reflexionen und -- zum Teil unsichtbare und
unbewusste -- eurozentristische Perspektiven aufzudecken, zu benennen,
infrage zu stellen und zu dekonstruieren. Dies gilt auf den
verschiedensten Ebenen bibliothekarischen Handelns -- vom Erwerben und
Sammeln übers Erschließen bis hin zur Präsentation und Vermittlung von
Beständen und Wissen.

Mit der aktuell auch und gerade in der Bibliothekswelt einsetzenden
Sensibilisierung für die gesellschaftspolitische Bedeutung und
historische Verantwortung, die mit Deutschlands kolonialem Erbe
verbunden ist, steht zu erwarten, dass -- überfällige --
Kolonialprovenienzforschungsprojekte durchgeführt werden und wichtige
neue Erkenntnisse hinsichtlich kolonialer Ursprünge und fragwürdiger
Erwerbungskontexte von Bibliotheksbeständen hervorbringen werden.
Darüber hinaus ist davon auszugehen, dass mit der voranschreitenden
Digitalisierung weiteres Material erschlossen und sichtbar gemacht
werden wird, welches kritisch erforscht werden kann.

Da BID-Einrichtungen sowohl mit ihren Sammlungen und mit ihren
Katalogsystemen sowie mit ihrer systematisch-hierarchisierten und
sprachlichen Wissensrepräsentation ein Abbild der gesellschaftlichen
Vorstellungen und kollektiven Identität ihrer Zeit darstellen, spiegeln
sie auch die herrschenden Auffassungen der Wissenschaft und damit
Machtverhältnisse bezüglich Wertigkeiten sowie wissenschaftlicher
Relevanz von Perspektiven, Fragestellungen und Methoden, bezüglich
Deutungshoheit und Definitionsmacht wider.

Gleichzeitig prägen sie in ihrer Funktion als Gedächtnisspeicher in
einer Wechselwirkung diese gesellschaftlichen Vorstellungen und wirken
identitätsstiftend\footnote{Heber, am angegebenen Ort, S. 8.} und haben
so das Potenzial, sozio-kulturelle Transformationsprozesse mitzusteuern
und eine wichtige Rolle bei gesellschaftlichen Entwicklungen als
\enquote{aktive Glieder zwischen Vergangenheit und Gegenwart für die
Zukunft}\footnote{Mittler, am angegebenen Ort, S. 39.} zu spielen.

\hypertarget{ausblick}{%
\section{7. Ausblick}\label{ausblick}}

Die vorliegende Untersuchung präsentiert gleichsam wie ein
Makro-Panorama mit mehreren \enquote{Zooms} in die Mikroebene eine
Bestandsaufnahme über koloniale und postkoloniale Reflexionen in
deutschen BID-Einrichtungen und im deutschen BID-Wesen. Auch wenn
postkoloniale Fragestellungen und Perspektiven weit davon entfernt sind,
in der Mitte der bibliothekarischen Gesellschaft angekommen zu sein,
lassen sich Chancen für eine gesteigerte Sensibilität und einen
bewussteren Umgang mit dieser Thematik registrieren.

Was die Strukturierung und Hierarchisierung von Wissen in
bibliothekarischen Klassifikationen anbelangt, ist als Alternative zu
den herkömmlichen hierarchischen, vielfach auf Binarität beruhenden,
autoritären Standardisierungsformen eine rhizomartige\footnote{Siehe
  Schulz, Bastienne: Die Karibik zwischen Enracinement und Errance:
  Neobarocke Identitätsentwürfe bei Édouard Glissant und Patrick
  Chamoiseau, Berlin: Ed. Tranvia, 2014, wo sie mit Bezug zu
  Deleuze/Guattari erläutert: ``Die Prinzipien der Konnexion und
  Heterogenität besagen, dass im Gegensatz zum Strukturbaum das Rhizom
  nicht aus binären Verbindungen besteht. Es verbindet vielmehr mittels
  unterschiedlicher Linien einen beliebigen Punkt mit einem anderen.
  Treffen diese Linien aufeinander, entsteht ein Bruch, an dem das
  Rhizom weiter wuchert. Es entsteht eine \enquote*{Anti-Genealogie}
  (Prinzip des asignifikanten Bruchs), Gilles Deleuze and Félix
  Guattari, Mille Plateaux (Éditions de minuit, 1980, S. 10ff.).} und
partizipative patron-driven Relationierung von Bestandstiteln denkar.
Ansätze und auch praktische Modelle hierfür existieren bereits, man
denke nur an die Stichworte \enquote{Bibliothek 2.0} mit ihren
\enquote{Folksonomies}, social tagging und schließlich das semantic web.

Die Digitalisierung gerade von Material aus postkolonialen Kontexten
(ebenso natürlich aus präkolonialen Kontexten, sofern es sich um
nicht-europäische/nicht-westliche Quellen handelt) hat ebenso das
Potenzial, mehr nicht-weiße Perspektiven sichtbar zu machen, wie der
bewusste vermehrte Erwerb von Materialien diverser Autor*innenschaft. So
kann, in einem graduellen Prozess, der dominante unbenannte, da von der
weißen Mehrheitsgesellschaft als \enquote{normal} angenommene,
eurozentristische Blick -- ganz nach der Forderung
\enquote{Provincialize Europe} von Dipesh Chakrabarty\footnote{Dipesh
  Chakrabarty, Dipesh: Provincializing Europe: Postcolonial Thought and
  Historical Difference, Princeton: 2000.
  \url{https://books.google.de/books/about/Provincializing_Europe.html?id=QqDa4tGENvYC\&pgis=1}}
-- mit anderen Positionen und Blickweisen in Frage gestellt und
relativiert werden. Gleichzeitig besteht auch im Zuge der
Digitalisierung nach wie vor die Gefahr, die sich bereits im Bereich des
World Wide Web teilweise realisiert hat, dass der eurozentrische Blick
der große Profiteur von der Digitalisierung ist. Insofern eine
Re-Kolonisierung anstatt einer De-Kolonisierung stattfindet.\footnote{Stingl,
  Alexander I. The Digital Coloniality of Power: Epistemic Disobedience
  in the Social Sciences and the Legitimacy of the Digital Age, Lanham:
  Lexington Books, 201.}

Eine weitere und tiefergehende Vernetzung nicht nur von nicht
eurozentristischen beziehungsweise postkolonial kritischen Quellen
innerhalb eines Bestands, sondern auch von Institutionen und ihren
Beständen und Sammlungen, Repositorien sowie Katalogen in Deutschland
wäre eine vorstellbare und wünschenswerte Herangehensweise. So wäre ein
gemeinsamer Verbundkatalog zum deutschen Kolonialismus denkbar,
vergleichbar mit dem gemeinsamen Internet-Katalog der
Arbeitsgemeinschaft der Gedenkstättenbibliotheken (AGGB).\footnote{\url{https://www.topographie.de/aggb/online-katalog/}}
Im Hinblick auf die Funktion von BID-Einrichtungen als
Gedächtnisspeicher wäre ein solcher Verbundkatalog ein
informationsstruktureller Beitrag zur Erinnerungskultur -- gleichsam ein
allen im Internet zugängliches \enquote{virtuelles Denkmal}.\footnote{Thein,
  Helen: Gedenkstättenbibliotheken: Zur Bestimmung eines Bibliothekstyps
  (Humboldt-Universität zu Berlin, 2017), S. 45,
  \url{http://dx.doi.org/https://doi.org/10.18452/18443}} Wenn auch
nicht sogleich als Grundstein für einen entsprechenden Verbundkatalog,
so weist das vom FID Afrikastudien an der Universitätsbibliothek der
Goethe-Universität Frankfurt am Main bei der DFG beantragte Projekt zum
Aufbau eines \enquote{Themenportals (Deutscher)
Kolonialismus}\footnote{Welches die Sammlungen von (1)
  Kolonial-Bibliothek der UB Frankfurt/Main, (2) Kolonialem Bildarchiv
  der UB Frankfurt, (3) das oben bereits vorgestellte Projekt
  \enquote{DSDK} sowie (4) die Digitalisierte Zeitschriften des FID
  Sozial- \& Kulturanthropologie (SKA) in ein Nachweissystem
  zusammenführen will. Ebenso ist die Einbindung des
  \enquote{Archivführers Kolonialismus} der FH Potsdam in das Projekt
  angedacht.} in genau diese Richtung.

Nicht zuletzt bedürfen solche Transformationsprozesse im deutschen
BID-Wesen einer kritischen Selbstwahrnehmung der Bibliothekar*innen und
sonstigen Entscheidungsträger*innen.\footnote{Hier noch einmal der
  Hinweis auf meine Selbstverortung als Weiße und meine aus dieser
  Sozialisierung geprägten Gedanken und Haltungen. Ich kann nicht für
  Schwarze und People of Colour sprechen, aber darauf hinweisen, dass es
  emanzipatorische Bibliothekseinrichtungen für eine von Rassismus in
  Deutschland betroffene Klientel durchaus gibt, z.B. die beim
  Referent\_innenrat der HU Berlin angesiedelte Stelle Amo Books, auch
  als A.W. Amo Books bekannt, versteht sich als selbstkritische
  Menschenrechts- und Entkolonialisierungsquelle für die kritische
  afrikanische Diaspora in Deutschland, die sich gegen Rassismus, sowie
  für Menschenwürde und Selbstbestimmung einsetzt. Amo Books umfasst
  unter anderem eine Bibliothek, einen Info-Stand, eine Buchhandlung und
  einen Verlag: \url{http://www.refrat.de/amo/wb/?js=1\&lang=de}} Es ist
meine Überzeugung, dass sie als Akteur*innen im Bildungs- und
Wissenschaftsbereich eine besondere Verantwortung haben, der
historischen Verantwortung Deutschlands nicht nur an der Shoah, sondern
auch an der Maafa\footnote{Das von Marimba Ani geprägte Wort bedeutet in
  Kiswahili: Katastrophe, große Tragödie, schreckliches Ereignis' und
  bezeichnet die komplexe interdependente Gemengelage von Sklaverei,
  Imperialismus, Kolonialismus, Invasion, Unterdrückung,
  Entmenschlichung und Ausbeutung, siehe dazu den Beitrag von Nadja
  Ofuatey-Rahal zu \enquote{Maafa} in: Susan Arndt, Wie Rassismus aus
  Wörtern spricht\,: (K)Erben des Kolonialismus im Wissensarchiv
  deutscher Sprache\,; Ein kritisches Nachschlagewerk (Unrast-Verl,
  2011), S. 594.} gerecht zu werden.

In diesem Sinne darf die vorliegende Arbeit als einer der ersten
Beiträge zu einer \enquote{bibliothekarischen
Kolonialvergangenheitsbewältigung} verstanden werden.

%autor
\begin{center}\rule{0.5\linewidth}{0.5pt}\end{center}

\textbf{Dr.~Antonia Paula Herm} (LL.M., Maître en droit, M.A.~(LIS)) hat
Rechtswissenschaft in Potsdam, Paris und Aberdeen und berufsbegleitend
Bibliotheks- und Informationswissenschaft in Berlin studiert. Sie ist
derzeit als wissenschaftliche Mitarbeiterin für den
Fachinformationsdienst für internationale und interdisziplinäre
Rechtsforschung an der Staatsbibliothek zu Berlin tätig.

\end{document}
