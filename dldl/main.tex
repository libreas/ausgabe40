\documentclass[a4paper,
fontsize=11pt,
%headings=small,
oneside,
numbers=noperiodatend,
parskip=half-,
bibliography=totoc,
final
]{scrartcl}

\usepackage[babel]{csquotes}
\usepackage{synttree}
\usepackage{graphicx}
\setkeys{Gin}{width=.4\textwidth} %default pics size

\graphicspath{{./plots/}}
\usepackage[ngerman]{babel}
\usepackage[T1]{fontenc}
%\usepackage{amsmath}
\usepackage[utf8x]{inputenc}
\usepackage [hyphens]{url}
\usepackage{booktabs} 
\usepackage[left=2.4cm,right=2.4cm,top=2.3cm,bottom=2cm,includeheadfoot]{geometry}
\usepackage{eurosym}
\usepackage{multirow}
\usepackage[ngerman]{varioref}
\setcapindent{1em}
\renewcommand{\labelitemi}{--}
\usepackage{paralist}
\usepackage{pdfpages}
\usepackage{lscape}
\usepackage{float}
\usepackage{acronym}
\usepackage{eurosym}
\usepackage{longtable,lscape}
\usepackage{mathpazo}
\usepackage[normalem]{ulem} %emphasize weiterhin kursiv
\usepackage[flushmargin,ragged]{footmisc} % left align footnote
\usepackage{ccicons} 
\setcapindent{0pt} % no indentation in captions


\usepackage{CJKutf8}

%%%% fancy LIBREAS URL color 
\usepackage{xcolor}
\definecolor{libreas}{RGB}{112,0,0}

\usepackage{listings}

\urlstyle{same}  % don't use monospace font for urls

\usepackage[fleqn]{amsmath}

%adjust fontsize for part

\usepackage{sectsty}
\partfont{\large}

%Das BibTeX-Zeichen mit \BibTeX setzen:
\def\symbol#1{\char #1\relax}
\def\bsl{{\tt\symbol{'134}}}
\def\BibTeX{{\rm B\kern-.05em{\sc i\kern-.025em b}\kern-.08em
    T\kern-.1667em\lower.7ex\hbox{E}\kern-.125emX}}

\usepackage{fancyhdr}
\fancyhf{}
\pagestyle{fancyplain}
\fancyhead[R]{\thepage}

% make sure bookmarks are created eventough sections are not numbered!
% uncommend if sections are numbered (bookmarks created by default)
\makeatletter
\renewcommand\@seccntformat[1]{}
\makeatother

% typo setup
\clubpenalty = 10000
\widowpenalty = 10000
\displaywidowpenalty = 10000

\usepackage{hyperxmp}
\usepackage[colorlinks, linkcolor=black,citecolor=black, urlcolor=libreas,
breaklinks= true,bookmarks=true,bookmarksopen=true]{hyperref}
\usepackage{breakurl}

%meta
\expandafter\def\expandafter\UrlBreaks\expandafter{\UrlBreaks%  save the current one
  \do\a\do\b\do\c\do\d\do\e\do\f\do\g\do\h\do\i\do\j%
  \do\k\do\l\do\m\do\n\do\o\do\p\do\q\do\r\do\s\do\t%
  \do\u\do\v\do\w\do\x\do\y\do\z\do\A\do\B\do\C\do\D%
  \do\E\do\F\do\G\do\H\do\I\do\J\do\K\do\L\do\M\do\N%
  \do\O\do\P\do\Q\do\R\do\S\do\T\do\U\do\V\do\W\do\X%
  \do\Y\do\Z}
%meta

\fancyhead[L]{Redaktion LIBREAS\\ %author
LIBREAS. Library Ideas, 40 (2021). % journal, issue, volume.
\href{https://doi.org/10.18452/23811}{\color{black}https://doi.org/10.18452/23811}
{}} % doi 
\fancyhead[R]{\thepage} %page number
\fancyfoot[L] {\ccLogo \ccAttribution\ \href{https://creativecommons.org/licenses/by/4.0/}{\color{black}Creative Commons BY 4.0}}  %licence
\fancyfoot[R] {ISSN: 1860-7950}

\title{\LARGE{Das liest die LIBREAS, Nummer \#9 (Herbst / Winter 2021)}}% title
\author{Redaktion LIBREAS} % author

\setcounter{page}{1}

\hypersetup{%
      pdftitle={Das liest die LIBREAS, Nummer \#9 (Herbst / Winter 2021)},
      pdfauthor={Redaktion LIBREAS},
      pdfcopyright={CC BY 4.0 International},
      pdfsubject={LIBREAS. Library Ideas, 40 (2021)},
      pdfkeywords={Literaturübersicht, Bibliothekswissenschaft, Informationswissenschaft, Bibliothekswesen, Rezension, literature overview, library science, information science, library sector, review},
      pdflicenseurl={https://creativecommons.org/licenses/by/4.0/},
      pdfcontacturl={http://libreas.eu},
      baseurl={https://doi.org/10.18452/23811},
      pdflang={de},
      pdfmetalang={de}
     }



\date{}
\begin{document}

\maketitle
\thispagestyle{fancyplain} 

%abstracts

%body
Beiträge von Ben Kaden (bk), Karsten Schuldt (ks), Michaela Voigt (mv),
Viola Voß (vv), Sara Juen (sj)

\hypertarget{zur-kolumne}{%
\section{1. Zur Kolumne}\label{zur-kolumne}}

Ziel dieser Kolumne ist es, eine Übersicht über die in der letzten Zeit
erschienene bibliothekarische, informations- und
bibliothekswissenschaftliche sowie für diesen Bereich interessante
Literatur zu geben. Enthalten sind Beiträge, die der LIBREAS-Redaktion
oder anderen Beitragenden als relevant erschienen.

Themenvielfalt sowie ein Nebeneinander von wissenschaftlichen und
nicht-wissenschaftlichen Ansätzen wird angestrebt und auch in der Form
sollen traditionelle Publikationen ebenso erwähnt werden wie
Blogbeiträge oder Videos beziehungsweise TV-Beiträge.

Gerne gesehen sind Hinweise auf erschienene Literatur oder Beiträge in
anderen Formaten. Diese bitte an die Redaktion richten. (Siehe
\href{http://libreas.eu/about/}{Impressum}, Mailkontakt für diese
Kolumne ist
\href{mailto:zeitschriftenschau@libreas.eu}{\nolinkurl{zeitschriftenschau@libreas.eu}}.)
Die Koordination der Kolumne liegt bei Karsten Schuldt, verantwortlich
für die Inhalte sind die jeweiligen Beitragenden. Die Kolumne
unterstützt den Vereinszweck des LIBREAS-Vereins zur Förderung der
bibliotheks- und informationswissenschaftlichen Kommunikation.

LIBREAS liest gern und viel Open-Access-Veröffentlichungen. Wenn sich
Beiträge dennoch hinter eine Bezahlschranke verbergen, werden diese
durch \enquote{{[}Paywall{]}} gekennzeichnet. Zwar macht das Plugin
\href{http://unpaywall.org/}{Unpaywall} das Finden von legalen
Open-Access-Versionen sehr viel einfacher. Als Service an der
Leserschaft verlinken wir OA-Versionen, die wir vorab finden konnten,
jedoch auch direkt. Für alle Beiträge, die dann immer noch nicht frei
zugänglich sind, empfiehlt die Redaktion Werkzeuge wie den
\href{https://openaccessbutton.org/}{Open Access Button} oder
\href{https://core.ac.uk/services/discovery/}{CORE} zu nutzen oder auf
Twitter mit
\href{https://twitter.com/hashtag/icanhazpdf?src=hash}{\#icanhazpdf} um
Hilfe bei der legalen Dokumentenbeschaffung zu bitten.

\hypertarget{artikel-und-zeitschriftenausgaben}{%
\section{2. Artikel und
Zeitschriftenausgaben}\label{artikel-und-zeitschriftenausgaben}}

\hypertarget{vermischte-themen}{%
\subsection{2.1 Vermischte Themen}\label{vermischte-themen}}

{[}Schwerpunktthema dieser Ausgabe{]} Baumann, Daniela Yvonne:
{[}Rezension zu{]} Jennifer Bajorek, Unfixed: \emph{Photography and
Decolonial Imagination in West Africa}. In: Camera Austria 153 (2021);
S. 91. {[}gedruckt{]}

Ausführlich würdigt die Rezensentin Jennifer Bajoreks die 2020 bei Duke
University Press (\url{https://doi.org/10.1515/9781478004585})
erschienene Studie zur Fotografiegeschichte in West-Afrika, in der sie
nicht nur Fotograf*innen und ihre Arbeiten vorstellt, sondern der Frage
nachgeht, inwieweit ein eigenständiges genuin afrikanisches
Publikationsformat für Fotografie -- hier am Beispiel der ab 1953 von
dem senegalesischen Schriftsteller und Politiker Ousmane Socé Diop
herausgegeben Zeitschrift BINGO, L'illustré africain Revue mensuelle de
l'activité noire -- und die darin transportierten Fotografien nicht
allein postkoloniale Veränderungen dokumentieren, sondern zugleich auch
prägen. Es geht folglich um Fotografie als mediales und politisches
Verfahren der Dekolonisierung. Bemerkenswert scheint der Ansatz, über
eine methodische Breite eurozentristische Projektionen auf den
Gegenstand bewusst zu machen und soweit wie möglich zu vermeiden. Durch
eine Annäherung in der Tiefe, bei der sie vor allem über Interviews mit
Fotografierenden einerseits die Entstehungsbedingungen und andererseits
die Wirkungen von Fotografie in Beziehung zu ihrem geografischen und
geschichtlichen Setting über die Bildanalyse stellt, wird die Rolle der
Fotografie für die Herausbildungen postkolonialer Identität,
Vorstellungen und visuellen Baustein gesellschaftlichen Wandels
herausgehoben. Diese Annäherung ist auch allgemein für eine dekolonial
gerichtete Auseinandersetzung mit Kulturzeugnissen und -praxen und die
dazu Forschenden relevant. Erstens, die Betroffenen sprechen lassen und
ihre Perspektiven sichtbar machen und zweitens eine umfassende
Kontextualisierung in geschichtliche Zeit und politischen Raum scheinen
sich als zwei elementare Grundsätze dekolonialer Forschungspraxis
anzubieten. (bk)

\begin{center}\rule{0.5\linewidth}{0.5pt}\end{center}

Kodua-Ntima, Kwame ; Akussahb, Harry ; Adjeib, Emmanuel (2021).
\emph{Managing stress among library staff in public university libraries
in Ghana}. In: The Journal of Academic Librarianship 47 (2021) 4,
102362, \url{https://doi.org/10.1016/j.acalib.2021.102362} {[}Paywall{]}
{[}OA-Version: \url{http://ugspace.ug.edu.gh/handle/123456789/36339}{]}

Der Text berichtet von einer Umfrage unter Bibliothekar*innen an drei
Universitäten in Ghana über die Level von Stress, welche diese im
Berufsalltag erfahren und über Coping-Strategies, welche sie gegen
diesen Stress anwenden. Sicherlich sind die Ergebnisse lokal geprägt.
(Die Befragten melden ein moderates, aber kontinuierliches Level an
Stress, mit Unterschieden zwischen den Universitäten und auch
unterschiedlichen Schwerpunkten. Ihre Coping-Strategien sind vor allem
auf persönlicher Ebene angesiedelt, beispielsweise indem Tage frei
genommen oder starke zwischenmenschliche Kontakte etabliert werden.)
Dennoch zeigt die Studie auf, dass Bibliotheken kein stressfreier
Arbeitsplatz sind. Sie gibt ein Modell einer Umfrage vor, welche auch in
anderen Bibliotheken durchgeführt werden könnte und gibt auch einige
Hinweise dazu, was von Bibliotheksleitungen unternommen werden könnte,
um den konstanten Stress ihrer Mitarbeitenden ausgleichen zu helfen.
(ks)

\begin{center}\rule{0.5\linewidth}{0.5pt}\end{center}

Oberlies, Mary K. ; J. Kirker, Maoria ; Mattson, Janna ; Byrd, Jason
(2021). \emph{Epistemology of Teaching Librarians: Examining the
Translation of Beliefs to Practice}. In: College \& Research Libraries
82 (2021) 4, \url{https://doi.org/10.5860/crl.82.4.513}

In dieser Studie (Umfrage und strukturierte Interviews) beschäftigen
sich die Autor*innen damit, wie Wissenschaftlichen Bibliothekar*innen,
die mit dem Unterricht von Informationskompetenzen beschäftigt sind,
ihre eigene Lehre reflektieren. Die Basis sind dabei US-amerikanische
Dokumente, aber die grundsätzlichen Ergebnisse werden sich wohl ähnlich
auch in anderen Ländern wiederfinden. Bibliothekar*innen haben einen
gewissen Bias dazu, den eigenen Lernstil als allgemeinen Lernstil zu
vermuten. Sie haben wenig Zeit und Praxis darin, ihre Lehre -- inklusive
der eigenen Annahmen dazu, welchen Lerntheorien sie implizit folgen --
zu reflektieren, aber wenn, dann tun sie dies vor allem zusammen mit
Kolleg*innen mit gleichen Aufgaben. Es gibt einen merklichen Unterschied
zwischen dem, was die Bibliothekar*innen gerne inhaltlich unterrichten
würden auf der einen Seite und dem, was von ihnen erwartet wird, das sie
unterrichten auf der anderen. Nicht zuletzt lernen viele das
Unterrichten nicht in der Ausbildung, sondern direkt in der Praxis. Dies
führt dazu, dass sie sich oft Gedanken darum machen, ob sie ausreichend
Autorität gegenüber den Studierenden vermitteln. Die Studie zeigt auch,
dass diese Lehrpraxis in ständiger Veränderung ist und deshalb nicht
einfach einmal festgeschrieben werden kann. Ein kontinuierliches
Nachdenken darüber, was man eigentlich tut und warum, ist also für diese
Kolleg*innen eigentlich notwendig, muss aber im Arbeitsalltag erst
einmal ermöglicht werden. (ks)

\begin{center}\rule{0.5\linewidth}{0.5pt}\end{center}

Bridges, Laurie M. ; Llebot, Clara (2021). \emph{Librarians as Wikimedia
Movement Organizers in Spain: An interpretive inquiry exploring
activities and motivations}. In: First Monday 26 (2021) 6--7,
\url{https://doi.org/10.5210/fm.v26i3.11482}

Diese Interviewstudie zeigt einerseits, was Bibliotheken in Spanien mit
der Wikipedia für Veranstaltungen und Angebote organisieren, wie sie an
ihr mitarbeiten und versucht andererseits zu klären, warum sie das tun.
Damit liefert der Text eine gute Übersicht über die Breite dieser
Angebote und liefert damit einen (weiteren) Einstieg für Bibliotheken,
die sich dafür interessieren.

Gleichzeitig zeigt er auch, dass der Grossteil der Motivation für die
Arbeit der befragten Bibliothekar*innen mit der Wikipedia und ihren
Schwesterprojekten weder intrinsisch motiviert ist, noch direkt aus der
Wikipedia-Community entstammt. Auffällig war für die Autorinnen, dass
die Bibliothekar*innen zwar Kontakte in diese Community haben, aber
nicht selber Teil davon sind, auch wenn sie erfolgreiche Veranstaltungen
für / mit der Wikipedia organisierten. Vielmehr wurden sie getrieben vom
Interesse, lokale Sprachen und Kultur zu erhalten und zu dokumentieren,
sowie Wissen an sich frei zu teilen. Dies widerspricht nicht den Zielen
der Wikipedia, aber es zeigte sich trotzdem, dass Bibliotheken und
Wikipedia zwei unterschiedliche Entitäten darstellen. (ks)

\begin{center}\rule{0.5\linewidth}{0.5pt}\end{center}

O'Neill, Brittany (2021). \emph{Do they know it when they see it?:
Natural language preferences of undergraduate students for library
resources}. In: College \& Undergraduate Libraries (Latest Articles),
\url{https://doi.org/10.1080/10691316.2021.1920535} {[}Paywall{]}

Regelmässig wird vermutet, dass Nutzer*innen den
\enquote{Bibliotheksjargon} nicht verstehen würden und deshalb
Schwierigkeiten damit hätten, Erklärungen von Bibliothekar*innen zum
Beispiel bei Bibliothekseinführungen, oder aber auch den Aufbau von
Bibliothekswebsiten zu verstehen. Diese Studie untersucht dies vor dem
Hintergrund, dass es immer wieder die Idee gibt, dieses Problem
anzugehen, indem stattdessen eine \enquote{natürliche Sprache} benutzt
wird.

Konkret wurden Studierende gebeten, eine Umfrage auszufüllen, bei der
ihnen beispielsweise eine Datenbank gezeigt wurde und sie dann sagen
sollten, wie sie diese nennen. Untersucht wurde auch, ob sich die
Antworten unterscheiden, wenn die Datenbank als Screenshot oder wenn
eine schriftliche Definition gezeigt wurde. Das Ergebnis war
ernüchternd: Viele Antworten waren \enquote{falsch} in dem Sinne, dass
die Angebote anders benannt wurden, als sie von Bibliothekar*innen im
\enquote{Bibliotheksjargon} genannt werden, gleichzeitig gab es auch
keine Übereinstimmung in diesen \enquote{falschen} Antworten. Ob die
Studierenden schon eine Bibliothekseinführung besucht hatten oder nicht,
hatte auch keinen erkennbaren Einfluss auf ihre Antworten. Es gibt
offenbar keine eindeutige Sprache, um Angebote von Bibliotheken zu
benennen, sondern sie werden von Menschen immer wieder anders benannt
und damit auch anders verstanden. Es ist wohl kein Fehler des
\enquote{Bibliotheksjargons}, es gibt einfache keine \enquote{natürliche
Sprache}, die man besser benutzen könnte. Die Zahl der befragten
Studierenden in dieser Studie war klein, die Autorin verortet die
Ergebnisse aber auch in der weiteren (erstaunlich umfangreichen)
Forschung zu dieser Frage. (ks)

\begin{center}\rule{0.5\linewidth}{0.5pt}\end{center}

Durrant, Summer (2021). \emph{Using an Evaluation Grid to Holistically
Assess Library Databases}. In: Collection Management (Latest Articles),
\url{https://doi.org/10.1080/01462679.2021.1958723} {[}Paywall{]}
{[}OA-Version: \url{https://scholar.umw.edu/administrative/14}{]}

Es wird -- wie im Titel angekündigt -- eine Form beschrieben, wie eine
Bibliothek (University of Mary Washington, Virginia) Datenbanken
bewertet, um Entscheidungen darüber zu treffen, ob sie (weiterhin)
abonniert werden sollten oder nicht. In diesem Fall werden für eine
Anzahl von Kriterien Punkte vergeben (diese sind im Anhang angegeben und
lassen sich so theoretisch auch nachnutzen).

Interessanter als die konkrete Umsetzung ist die Darstellung des Status
Quo. Die Autorin stellt selber fest, dass sich in vielen
Wissenschaftlichen Bibliotheken die Frage stellt, wie Datenbanken und
ähnliche Angebote besser bewertet werden können als mit reinen
Kosten-Nutzen-Analysen. Dieser Wunsch nach \enquote{ganzheitlichen}
Analysen hat verschiedene Lösungen hervorgebracht, die in Bibliotheken
genutzt werden. Nur einige davon scheinen auch publiziert worden sein
(zu den publizierten liefert der Artikel eine Übersicht). Wieder einmal
zeigt sich hier ein Problem, das von vielen Bibliotheken offenbar
alleine angegangen wurde, obwohl es sich praktisch allen stellt. Eine
Zusammenarbeit und offene Diskussion zwischen Bibliotheken wäre
hilfreich. (ks)

\begin{center}\rule{0.5\linewidth}{0.5pt}\end{center}

Garnar, Martin ; Tonyan, Joel (2021). \emph{Library as place:
Understanding contradicting user expectations.} In: The Journal of
Academic Librarianship 47 (2021) 102391,
\url{https://doi.org/10.1016/j.acalib.2021.102391} {[}Paywall{]}

Die Autoren kritisieren in diesem Text zuerst, dass die meisten
Bibliotheken sich damit zufrieden geben (oder sich nicht anderes
zutrauen), Umfragen durchzuführen, wenn es darum geht, zu verstehen, wie
Nutzer*innen die Bibliothek sehen und was sie von ihnen fordern.
Vielmehr müssten mehr Methoden genutzt und vor allem mit Nutzer*innen
direkt geredet werden, wenn man verstehen will, wie diese Bibliotheken
wahrnehmen.

Anschliessend berichten sie von einer Studie in der Bibliothek der
University of Colorado, in welcher sie versuchen, diesem Anspruch
gerecht zu werden. Sie nutzen neben der Analyse der Bibliothek selber
undeiner Umfrage auch Fokusgruppen und Interviews. Die Erkenntnis dieses
Vorgehens ist, dass die Nutzer*innen den Raum Bibliothek als ruhigen
Raum schätzen, in dem sie konzentriert arbeiten können, dass sie Zugang
zu gedruckten und elektronischen Medien haben wollen und die Arbeit der
Bibliothekar*innen positiv hervorheben. Ausserdem wollen sie einfach
zugängliche Räume, die sie ohne grosse Einführungen nutzen können. Es
zeigt sich also, wieder einmal, dass die grossen Veränderungen in der
Nutzung, von denen Bibliotheken ausgehen, sich auch in dieser Bibliothek
nicht wirklich zeigen. (ks)

\begin{center}\rule{0.5\linewidth}{0.5pt}\end{center}

Owens, E. (2021). \emph{Impostor Phenomenon and Skills Confidence among
Scholarly Communications Librarians in the United States}. In College \&
Research Libraries 82 (2021) 4, 490--512.
\url{https://doi.org/10.5860/crl.82.4.490}

Twitter: Peter Suber (@petersuber),
\url{https://twitter.com/petersuber/status/1443645606468145153}

Der Term \enquote{Imposter-Syndrom} benennt eine kognitive Verzerrung in
Bezug auf die eigenen Fähigkeiten und Erfolge -- nämlich eine verzerrte
Selbstwahrnehmung, welche dazu führt, dass Betroffene sich selbst und
ihre Leistungen regelmäßig unterschätzen und Sorge haben, dass andere
sie als als unfähig (und ähnliches) entlarven könnten. Owens hat in
einer Umfrage untersucht, wie verbreitet das Imposter-Syndrom unter
Bibliotheksbeschäftigten ist, die im Bereich Publikationsunterstützung
tätig sind. (Zum Umfeld \enquote{scholarly communication} zählt Owens
dabei Personen, die sich mit den folgenden Themenfeldern beschäftigen:
Repositorien, Wissenschaftliches Publizieren, Urheberrecht,
(Forschungs-)Datenmanagement, Forschungsevaluation.) Owens hat im
Zeitraum Mitte Februar bis Ende März 2020 Beschäftigte in
wissenschaftlichen Bibliotheken in den USA online befragt; von 206
begonnenen Antworten konnte Owens 149 einzelne, abgeschlossene
Umfrageergebnisse auswerten. Im Ergebnis stellt sie fest, dass das
Imposter-Syndrom vergleichsweise häufig bei den Bibliotheksbeschäftigten
verbreitet ist, die im Umfeld \enquote{scholarly communication} tätig
sind. Sie führt selbst verschiedene Aspekte auf, die die Aussagekraft
ihrer Umfrageergebnisse beschränken -- unter anderem
\enquote{self-selection bias} und dem allgemeinen Problem von mangelnder
Objektivität bei Selbsteinschätzungen. Der Artikel ist -- vielleicht vor
allem anekdotisch -- interessant, da er auch mögliche Strategien
benennt, um dem Imposter-Syndrom im Arbeitsalltag zu begegnen.\\
Peter Suber macht auf Twitter auf diesen Artikel zum Imposter-Syndrom
von Erin Owens aufmerksam: Die Ergebnisse bestätigen seine persönlichen
Erfahrungen. In dem Twitter-Thread liefert er weitere Erklärungsansätze
dafür, dass das Imposter-Syndrom vergleichsweise häufiger bei den
Bibliotheksbeschäftigten verbreitet ist, die im Umfeld
\enquote{scholarly communication} tätig sind: Gründe sind seiner Meinung
nach unter anderem die Neuigkeit und Schnelllebigkeit der Themenfelder,
welche in der Regel unter \enquote{scholarly communication} subsumiert
werden, die mangelnde Abdeckung dieser Themenfelder in der
Berufsausbildung sowie die Vielzahl und Vielfältigkeit der Themen, die
in einer Institution durch eine Person oder sehr kleine Personengruppe
abzudecken sind. (mv)

\hypertarget{kritik-der-aktuellen-versuche-von-bibliotheken-und-bibliothekswissenschaft-diverser-zu-werden}{%
\subsection{2.2 Kritik der aktuellen Versuche von Bibliotheken und
Bibliothekswissenschaft, diverser zu
werden}\label{kritik-der-aktuellen-versuche-von-bibliotheken-und-bibliothekswissenschaft-diverser-zu-werden}}

{[}Schwerpunktthema dieser Ausgabe{]} Mehra, Bharat (2021). \emph{Enough
Crocodile Tears!: Libraries Moving beyond Performative Antiracist
Politics.} In: Library Quarterly 91 (2021) 2, 137--149,
\url{https://doi.org/10.1086/713046} {[}Paywall{]}

{[}Schwerpunktthema dieser Ausgabe{]} Cooke, Nicole ; Colón-Aguirre,
Mónica (2021). \emph{\enquote{Killing It from the Inside}: Acknowledging
and Valuing Black, Indigenous, and People of Color as LIS Faculty.} In:
Library Quarterly 91 (2021) 3, 243--249,
\url{https://doi.org/10.1086/714324} {[}Paywall{]}

{[}Schwerpunktthema dieser Ausgabe{]} Wiegand, Wayne A. (2021).
\emph{Race and School Librarianship in the Jim Crow South, 1954-1970:
The Untold Story of Carrie Coleman Robinson as a Case Study.} In:
Library Quarterly 91 (2021) 3, 254--268,
\url{https://doi.org/10.1086/714314} {[}Paywall{]}

Gleich zwei aktuelle Editorials -- wobei es in dieser Zeitschrift oft
mehr als ein Editorial pro Ausgabe gibt -- der \emph{Library Quarterly}
(Mehra 2021 und Cooke \& Colón-Aguirre 2021) thematisieren die
Orientierung des US-amerikanischen Bibliothekswesens hin zu mehr
Diversity im Anschluss an die Black-Lives-Matter-Proteste 2020. Dies
aber kritisch, aus der Sicht von nicht-weissen Kolleg*innen. In beiden
wird postuliert, dass die Themen an sich nicht neu wären: Rassismus gäbe
es schon weit länger, Polizeigewalt auch. Mehra eher wütend, Cooke und
Colón-Aguirre eher diplomatisch, aber trotzdem meinungsfest, halten dem
Bibliothekswesen vor, dass es nicht einfach das Thema besetzen könne,
ohne sich selber zu verändern. Es gäbe die Tendenz, einfach so zu tun,
als wären Bibliotheken, Bibliotheksverbände und die bibliothekarische
Ausbildung an sich schon antirassistisch, obwohl sich immer wieder das
Gegenteil zeigt: Bei Redebeiträgen auf Konferenzen, bei den Ergebnissen
von Programmen, die eigentlich Diversität im Bibliothekswesen verbessern
sollten, in der Bibliothekspraxis oder im Umgang mit Kolleg*innen, die
nicht zu den bestimmenden Gruppen der Gesellschaft gehören. Dies könne
sich nur ändern, wenn das Bibliothekswesen die eigenen Strukturen
analysiert, als veränderungswürdig begreift und dann tatsächlich
verändert, also nicht nur darüber rede.

Dazu passt der auch in der Library Quarterly veröffentlichte Text von
Wiegand über Carrie Coleman Robinson, die in den späten 1960er Jahren
als Schulbibliothekarin in Alabama arbeitete und -- vertreten durch die
Teacher Association -- den Bundesstaat Alabama wegen Diskriminierung
aufgrund ihrer \enquote{race} verklagte. Wiegand stellt dar, wie die
bibliothekarischen Verbände sich weigerten, diesen Prozess überhaupt
wahrzunehmen oder gar ihr eigenes Mitglied, das Coleman Robinson war, zu
unterstützen. Insbesondere die School Library Association akzeptierte
stattdessen lieber segregierte Verbände in den Südstaaten, als sich
überhaupt mit dem Thema Rassismus zu befassen. Wiegand, der auch sonst
zur Bibliotheksgeschichte publiziert, reflektiert am Ende seines
Artikels, dass Bibliotheken ihre eigene Geschichte als positiv sehen
wollen, sich selber als progressive Institutionen und deshalb die Teile
ihrer eigenen Geschichte ignorieren, welche zeigen, dass dieses
Selbstbild historisch nicht stimmt. Damit vergäben sie sich auch die
Chance, über ihre eigenen inhärenten Strukturen zu lernen, welche in den
genannten Editorials kritisiert werden. (ks)

\begin{center}\rule{0.5\linewidth}{0.5pt}\end{center}

{[}Schwerpunktthema dieser Ausgabe{]} Cooke, Nicole A. ; Kitzie, Vanessa
L. (2021). \emph{Outsiders-within-Library and Information Science:
Reprioritizing the marginalized in critical sociocultural work.} In:
JASIST - Journal of the Association for Information Science and
Technology 72 (2021) 10: 1285--1294,
\url{https://doi.org/10.1002/asi.24449} {[}Paywall{]}

Im Rahmen einer Schwerpunktausgabe, die sich damit auseinandersetzt, ob
und wie es einen Paradigmenwechsel in der Library and Information
Science (LIS) hin zu einem umfassenderen, von gesellschaftlichen Fragen
geprägten Verständnis von Information und Informationsnutzung geben
sollte, kritisieren Cooke und Kitzie die LIS grundsätzlich. Die Frage
sei nicht, ob ein solcher Paradigmenwechsel notwendig sei -- auch wenn
sie ihn explizit nicht ablehnen --, sondern ob er überhaupt möglich
wäre. Die in der Ausschreibung zur Schwerpunktausgabe eingeforderte
Forschung zur Informationsnutzung, welche die Sichtweisen
marginalisierter Gruppen mit einbezieht, andere Blickwinkel einnimmt und
andere Forschungsparadigmen anwendet, als die in der LIS verbreiteten,
würden schon existieren. Sie würden aber weder gesehen noch als
vollwertig akzeptiert.

Das Feld der LIS sei geprägt von weissen Forschenden, welche auch in
ihrer Forschung den eigenen Standpunkt nicht reflektierten und vor allem
auch mittelständische Interesse und Blickwinkel einbringen. Forschende
aus marginalisierten Gruppen würden nicht als gleichwertig, ihre
Arbeiten als mehr kritikwürdig als andere behandelt. Wenn, dann würden
sie als \enquote{Token} behandelt, welche zum Beispiel auf Panels eine
Diversität repräsentieren sollen, die es in der konkreten Forschung aber
nicht gibt. Es gäbe teilweise Abwehrprozesse gegen diese, auch weil sie
dem Selbstbild einer \enquote{neutralen Forschung} widersprächen.

Cooke und Kitzie vertreten explizit die Meinung, dass sich die LIS
grundsätzlich ändern muss und dabei die eigenen Grundannahmen benennen
und reflektieren, aber auch Forschende aus marginalisierten Gruppen
direkt unterstützen muss, beispielsweise indem deren Forschungsansätze
und Forschungen in den Kanon aufgenommen und nicht als trendy oder
irrelevant abgewehrt werden, wenn sie das Ziel erreichen will, diverser
zu werden. Es gehe nicht um eine reine Ausweitung der
Forschungsinteressen, sondern um eine Veränderung der LIS selber. (ks)

\hypertarget{veruxe4nderung-der-aufgaben-und-der-arbeit-in-wissenschaftlichen-bibliotheken}{%
\subsection{2.3 Veränderung der Aufgaben und der Arbeit in
Wissenschaftlichen
Bibliotheken}\label{veruxe4nderung-der-aufgaben-und-der-arbeit-in-wissenschaftlichen-bibliotheken}}

Glaser, Timo (2021). \emph{Digital Humanities aus dem Fachreferat
heraus}. In: 027.7 Zeitschrift für Bibliothekskultur / Journal for
Library Culture 8(1). \url{https://doi.org/10.21428/1bfadeb6.3daa6c49}

Einen speziellen Bereich neuer Fachreferatsaufgaben prüft Timo Glaser:
Die Digital Humanities (DH) als
Informationskompetenz-Vermittlungsaufgabe und als Forschungsaufgabe.
Denn: \enquote{Es finden sich {[}\ldots{]} an Universitäten kaum Orte,
an denen Studierende der Geisteswissenschaften diese Techniken lernen
können. In einzelnen Studiengängen gibt es freilich fachspezifische
Angebote, in reguläre Module oder Lehrveranstaltungen eingebettete
Workshops oder Ähnliches werden allerdings eher selten von fachfremden
Studierenden wahrgenommen. Fachübergreifende Workshops gibt es
gelegentlich in Graduiertenschulen, diese stehen dann allerdings den
meisten Studierenden nicht offen.}

Aus diesen Überlegungen heraus ist an der UB Marburg das
\enquote{Digital Humanities Learning Lab} entstanden
(\url{https://www.uni-marburg.de/de/ub/lernen/kurse-beratung/dll}).
Dieses Lab hat nicht nur die Sichtbarkeit der Bibliothek als
Ansprechpartnerin für DH-Themen in unterschiedlichen Kontexten erhöht,
sondern die aufgebauten Kompetenzen zu digitalen Techniken konnten auch
\enquote{in house} profitabel eingesetzt werden, zum Beispiel bei der
Informationsextraktion aus Texten und Dateien oder bei statistischen
Auswertungen. Weitergeführt unter dem Blickwinkel \enquote{DH als
Forschungsaufgabe} eröffnen sich dann weitere Möglichkeiten,
\enquote{den Forschungsalltag {[}zu{]} erleben} und \enquote{die
Kommunikationsfähigkeit mit den Wissenschaftler:innen} zu erhöhen. Aber:
\enquote{Inwieweit diese Tendenz auch im deutschsprachigen
Bibliotheksbereich um sich greifen wird, bleibt abzuwarten.} (vv)

\begin{center}\rule{0.5\linewidth}{0.5pt}\end{center}

Madhusudhan, Margam ; Lamba, Manika (2021). \emph{The changing roles of
librarians: Managing emerging technologies in libraries}. In: Singh,
Sanjay Kumar ; Sarma, Kishor (Hrsg.). Quality Library Services in the
New Era. Guwahati: EBH. S. 88--96. Preprint:
\url{https://doi.org/10.5281/zenodo.4721411}

Mit der Frage der sich wandelnden Aufgaben von Subject Librarians
befassen sich Margam Madhusudhan und Manika Lamba vom Department of
Library and Information Studies der Uni Delhi. Sie haben 14
\enquote{traditionelle} und 16 aktuelle Rollen beziehungsweise
Tätigkeiten identifiziert, zu denen sie ergänzend 24 (!) neue Rollen
oder Tätigkeiten vorschlagen, von klein (zum Beispiel die Verwendung von
QR-Codes auf der Bibliothekswebseite) bis groß (zum Beispiel der Einsatz
von Blockchain-Technologien in verschiedenen bibliothekarischen
Bereichen). Diese 24 sind nur \enquote{\emph{{[}s{]}ome} {[}Hervorhebung
V.V.{]} of the new roles which liaison librarians should perform in
addition to the traditional and current roles}. Man könnte ja mal
überlegen, was einem nach Durchsicht aller 54 Punkte noch als fehlend
auffällt -- und was man selbst unter Fachreferat beziehungsweise
Subject/Liaison Librarian verbuchen oder vielleicht eher in anderen
Abteilungen sehen würde.

Um all diese Rollen ausfüllen zu können, benötigt ein*e
\enquote{Librarian 4.0} Kompetenzen und Fähigkeiten aus neun Bereichen,
die die Autorinnen zusammengestellt haben. Da überrascht es nicht, dass
sie zu dem Schluss kommen: \enquote{The role of a subject librarian is
dynamic, broad, and intensive in nature.} Ob Helmut Oehling mit solch
einer Intensität gerechnet hat, als er 1998 seine \enquote{Thesen zur
Zukunft des Fachreferenten} schloss mit These 12: \enquote{Der
Fachreferent 2000 ist unverzichtbar für Wissenschaft und Lehre und damit
frei von allen Legitimationsproblemen seines Berufsstandes. Er erreicht
Akzeptanz durch Kompetenz.}?

(Oehling, Helmut (1998). \emph{Wissenschaftlicher Bibliothekar 2000 --
quo vadis? 12 Thesen zur Zukunft des Fachreferenten}. In:
Bibliotheksdienst 32 (1998) 2: 247--254.
\url{https://doi.org/10.1515/bd.1998.32.2.247}. {[}Paywall{]}
{[}OA-Version, Preprint:
\url{https://www.tuhh.de/b/hapke/agfnthes.html}{]}) (vv)

\begin{center}\rule{0.5\linewidth}{0.5pt}\end{center}

Li, Xiang ; Li, Tang (2021). \emph{The Evolving Responsibilities, Roles,
and Competencies of East Asian Studies Librarians: A Content Analysis of
Job Postings from 2008 to 2019}. In: College \& Research Libraries 82
(2021) 4, \url{https://doi.org/10.5860/crl.82.4.474}

Mit der schon etablierten Methode der Inhaltsanalyse von
Stellenausschreibungen untersuchen Li \& Li, wie sich die Aufgaben von
Fachreferent*innen im Bereich Ostasien-Studien in den USA in den letzten
Jahren verändert haben. Bedenkenswertes Ergebnis -- insbesondere, wenn
es sich auch für andere Fachgebiete und Länder zeigen sollte -- ist,
dass zwar der Aufgabenbereich um Kommunikation und Netzwerkarbeit
innerhalb der jeweiligen Bibliothek und Universität sowie des gesamten
bibliothekarischen Feldes gewachsen ist, aber dass bei allen
Veränderungen weiterhin der Hauptfokus auf dem Bestandsmanagement liegt.
Dies hat sich seit Jahrzehnten nicht verändert. (ks)

\begin{center}\rule{0.5\linewidth}{0.5pt}\end{center}

Morales, Jessica M. ; Beis, Christina A. (2021). \emph{Communication
across the electronic resources lifecycle: a survey of academic
libraries}. In: Journal of Electronic Resources Librarianship 33 (2021)
2: 75--91, \url{https://doi.org/10.1080/1941126X.2021.1913841}
{[}Paywall{]} {[}OA-Version:
\url{https://ecommons.udayton.edu/roesch_fac/71}{]}

In diesem Umfrage (in den USA) wurden Wissenschaftliche
Bibliothekar*innen befragt, zu welchen Themen, mit welcher selbst
eingeschätzten Zufriedenheit und auf welchen Wegen sie mit Nutzer*innen
kommunizieren. Grundsätzlich sind viele damit zufrieden, wie diese
Kommunikation funktioniert. Bevorzugt werden Mailkontakte und
persönliche Treffen (allerdings wurde die Umfrage vor der
Covid-19-Pandemie durchgeführt). Andere Formen von Kommunikation wie
Ticketsysteme oder Online-Konsultationen kommen vor, aber in viel
weniger Fällen. Was die Umfrage vor allem zeigt, ist die Bedeutung,
welche Kommunikation mit Nutzer*innen (auch) in Wissenschaftlichen
Bibliotheken im Arbeitsalltag der Bibliothekar*innen hat. Es ist zu
erwarten, dass eine Umfrage im DACH-Raum zu ähnlichen Ergebnissen kommen
würde. Dies scheint aber weder in der bibliothekarischen Literatur noch
der Aus- und Weiterbildung reflektiert zu sein. (ks)

\begin{center}\rule{0.5\linewidth}{0.5pt}\end{center}

Schuster, Janice G. (2021). \emph{Collection, organization, analysis,
and application of usage data to inform an academic library's annual
electronic resource renewal decisions}. In: Journal of Electronic
Resources Librarianship, 33 (2021) 2: 130--135,
\url{https://doi.org/10.1080/1941126X.2021.1912555} {[}Paywall{]}
{[}OA-Version: \url{http://works.bepress.com/janice_schuster/40/}{]}

Die Kollegin, von der dieser Text stammt, gibt einen kurzen Einblick,
wie in ihrer Bibliothek (an einer kleinen katholischen Universität)
COUNTER-Daten gesammelt, aufbereitet und dann für Entscheidungen über
die Bestandsentwicklung genutzt werden. Sie stellt kurz die steigende
Professionalisierung in den letzten Jahren dar, aber auch, wie viel
immer noch von spezifischen Entscheidungen und Handlungsabläufen in der
lokalen Universität geprägt ist. Interessant wären mehr solche
Hands-on-Berichte, da diese Arbeit selbstverständlich in allen
Bibliotheken durchgeführt wird, aber offenbar immer wieder lokal neu
aufgesetzt wird. Ein weiter Austausch zum Vorgehen und den Erfahrungen
damit, könnte zu einem professionelleren Bestandsmanagement im gesamten
Bibliothekswesen führen. (ks)

\hypertarget{covid-19-und-die-bibliotheken-dritte-welle}{%
\subsection{2.4 COVID-19 und die Bibliotheken, Dritte
Welle}\label{covid-19-und-die-bibliotheken-dritte-welle}}

Kou, Yin ; Chen, Ping ; Pan, Jie-Xing (2021). \emph{The Service
Experiences of Public Libraries during the COVID-19 Emergency in Wuhan:
Three Case Studies}. In: Journal of the Australian Library and
Information Association, 70 (2021) 3, 287--300,
\url{https://doi.org/10.1080/24750158.2021.1960251} {[}Paywall{]}

Die Stadt Wuhan ist mit der Covid-19-Pandemie eng verbunden, da hier
bekanntlich das betreffende Virus das erste Mal nachgewiesen wurde.
Gleichzeitig war es die erste Grossstadt, die in einen Lockdown ging.
Der Text stellt die Arbeit von drei Bibliotheken (Hubei Provincial
Library, Wuhan {[}Public{]} Library, Jianghan District Library) im
Frühjahr 2020 vor, was an sich schon interessant ist, weil es auch die
ersten Bibliotheken waren, die mit der Pandemie umzugehen lernen
mussten.

Es zeigen sich darüber hinaus aber tatsächlich Unterschiede zwischen
diesen chinesischen Bibliotheken und Berichten über die Arbeit von
Bibliotheken in anderen Ländern, welche in früheren Ausgaben dieser
Kolumne besprochen wurden. Die Bibliotheken in Wuhan stellten, wie
anderswo auch, schnell auf digitale Angebote um, entwickelten aber auch
eigene Ausstellungen und digitale Sammlungen. In den temporären
Krankenhäusern, die im Frühjahr 2020 eingerichtet wurden -- und das war
in anderen Berichten nicht zu lesen -- übernahmen sie
biblio-therapeutische Funktionen, indem sie zusammen fast hundert
Leseecken für Personal und Kranke einrichteten und bestückten. Zudem
wird thematisiert, dass alle Bibliotheken ihr Personal aufforderten,
freiwillig andere gesellschaftliche Aufgaben bei der Pandemie-Bekämpfung
zu übernehmen. (Wie freiwillig dies wirklich erfolgte, ist nicht Thema
des Textes.) Die Autor*innen betonen zudem, dass die Pandemie dazu
geführt hat, dass die Bibliotheken ihre Zusammenarbeit untereinander
verstärkten und jetzt als Netzwerk agieren. (ks)

\begin{center}\rule{0.5\linewidth}{0.5pt}\end{center}

Kehnemuyi, Kaitlin (2021). \emph{Effects of COVID-19 on Disaster
Planning in Academic Libraries}. In: Journal of Library Administration,
61 (2021) 5: 507--529,
\url{https://doi.org/10.1080/01930826.2021.1924530} {[}Paywall{]}

Dieser Text berichtet über die Ergebnisse einer Umfrage unter
Wissenschaftlichen Bibliotheken an der US-amerikanischen Ostküste, ob es
Katastrophenpläne in ihren Einrichtungen gibt und wenn ja, wie sie sich
in der COVID-19 Pandemie bewährt haben. Die Umfrage wurde im Sommer 2020
durchgeführt, also -- wie die Autorin selber bemerkt -- eigentlich zu
früh, da die Pandemie nicht, wie erhofft, im Herbst 2020 endete.

Allerdings ist der Titel des Artikels auch etwas irreführend. Weit mehr
als die Hälfte bezieht sich gar nicht auf die Umfrage selber, sondern
stellt dar, was Katastrophenpläne für Bibliotheken sein können, warum
sie notwendig sind, wie sie erstellt und aktualisiert werden. Damit
liefert die Autorin einen umfassenden Überblick, gerade wenn sich neu
mit dem Thema beschäftigt wird (was, so ist der Autorin zuzustimmen, in
jeder Bibliothek passieren sollte).

Die Ergebnisse der Umfrage sind dagegen eher mittelmässig interessant:
Die meisten der antwortenden Bibliotheken hatten einen Katastrophenplan,
aber das mag durch die Selbstauswahl der Einrichtungen bedingt sein. Die
ohne Plan haben vielleicht gar nicht erst geantwortet. Die meisten Pläne
waren auf lokal zu erwartende Ereignisse abgestimmt (Überflutung, Feuer,
und, wohl US-spezifisch, \enquote{active shooter}). Nur eine Bibliothek
hatte im Laufe einer früheren Pandemie (H1N1-Pandemie 2009/2010) einen
Plan erstellt, der während der COVID-19-Pandemie eingesetzt werden
konnte. Alle anderen waren eher von dieser Katastrophe überrascht.
Allerdings arbeiteten auch schon im Sommer 2020 viele Bibliotheken
daran, Pandemien in ihre Katastrophenplanung aufzunehmen. (ks)

\begin{center}\rule{0.5\linewidth}{0.5pt}\end{center}

Ameen, Kanwal (2021). \emph{COVID-19 pandemic and role of libraries}.
In: Library Management, 42 (2021) 4/5: 302--304,
\url{https://doi.org/10.1108/LM-01-2021-0008} {[}Paywall{]}

Dieser Text über die Situation in Pakistan erinnert daran, dass die
Pandemie für Länder des globalen Südens noch lange nicht zu Ende ist,
sondern aufgrund fehlender Impfstoffe auf lange Sicht anhalten wird.
Dies gilt auch für Bibliotheken, die weiterhin mit ihr planen müssen.
Gleichzeitig betont die Autorin, dass die Pandemie gezeigt hat, dass dem
pakistanischen Bibliothekswesen weithin die Kompetenzen fehlen,
elektronische Dienstleistungen zugänglich zu machen. Allerdings sieht
sie dies nicht nur negativ. Die Gesellschaft hätte erkannt, wie
bedeutsam Digital Equality wäre, deshalb wäre es in Zukunft wohl
einfacher, diese anzustreben. (ks)

\begin{center}\rule{0.5\linewidth}{0.5pt}\end{center}

Wilson, Maree (2021). \emph{Australian Public Library Staff Living
through a Pandemic: Personal Experience of Serving the Community}. In:
Journal of the Australian Library and Information Association, 70:3,
322--334, \url{https://doi.org/10.1080/24750158.2021.1955436}
{[}Paywall{]}

Der Artikel basiert auf einer Auswertung von acht Interviews mit
Bibliothekar*innen in australischen Public Libraries. Die Autorin
achtete darauf, dass diese in verschiedenen Bundesstaaten und in
unterschiedlichen Hierarchiepositionen arbeiten, um mit ihnen eine
umfassendere Übersicht zu bieten, als dies Berichte aus einzelnen
Bibliotheken alleine tun können. Deshalb ist der Methoden- und
Theorieteil des Textes etwas ausufernd. Anzumerken ist auch, dass
Australien zwar nicht so erfolgreich im Umgang mit der Covid-Pandemie
war beziehungsweise ist wie das Nachbarland Aotearoa Neuseeland, aber
doch viel besser als europäische Staaten. Insoweit waren die
Auswirkungen auf die Bibliotheken vielleicht nicht so gross wie im
DACH-Raum.

Was sich zeigte, war erstens, dass Öffentliche Bibliotheken auch in
Australien schnell auf die Herausforderungen reagierten und zum Beispiel
digitale Dienstleistungen aufbauten. Auf die geleistete Arbeit waren die
befragten Kolleg*innen stolz. Überrascht waren sie davon, wie sehr sich
zeigte, welche Bedeutung die Bibliotheken für Personen in den jeweiligen
Gemeinden hatten. Sie berichteten davon, überrascht zu sein, wie lange
die Schlangen bei der Wiedereröffnung waren oder wie positiv die
Rückmeldungen. Offenbar haben die Kolleg*innen die bibliothekarische
Propaganda davon, welche Bedeutung Öffentliche Bibliotheken in der
jeweiligen Gemeinde spielen, die selbstverständlich auch in Australien
verbreitet wird, selber nicht geglaubt, bis sie die tatsächlichen
Reaktionen gesehen haben.

Eine Kritik ist am Text zu leisten: Am Ende behauptet die Autorin, dass
dies alles ein Hinweis darauf wäre, wie sehr Bibliotheken tatsächlich
Dritte Orte werden. Aber davon war in den Ergebnissen, welche die
Autorin berichtet, gar nicht die Rede. Egal, wie Dritte Orte definiert
werden, geht es dort ja immer darum, dass Nutzer*innen miteinander
kommunizieren und Communities bilden. Aber die Interviews berichten von
Kommunikation zwischen Nutzer*innen und Bibliothekar*innen. Das ist
etwas anderes. Hier scheint der Wunsch nach einer bestimmten Entwicklung
die realen Ergebnisse überdeckt zu haben. (ks)

\begin{center}\rule{0.5\linewidth}{0.5pt}\end{center}

Heady, Christina ; Vossler, Joshua ; Weber, Millicent (2021). \emph{Risk
and ARL Academic Library Policies in Response to COVID-19}. In: Journal
of Library Administration, 61:7, 735--757,
\url{https://doi.org/10.1080/01930826.2021.1972725} {[}Paywall{]}

Dieser Text ist aus zwei Gründen hervorzuheben. Er ist unter anderem
eine Studie dazu, ob sich der Reichtum der Trägereinrichtung, deren
Status (privat oder öffentlich) oder die politische Kultur des
Bundesstaates, in dem eine Bibliothek situiert ist, darauf auswirkt, wie
viel oder wenig risikoreich mit der COVID-19 Pandemie (bis Frühjahr
2021) umgegangen wurde. Oder anders gesagt: Ob Bibliothekar*innen und
Nutzer*innen in reicheren Einrichtungen besser geschützt wurden als in
ärmeren. Das alles fokussiert auf Hochschulbibliotheken, die Mitglied in
der Association of Research Libraries (ARL) (alle aus den USA oder
Kanada) sind. Die Ergebnisse sind also nicht einfach übertragbar, aber
doch interessant.

Methodisch wurden verschiedene Sicherheitsmassnahmen, die während der
Pandemie in Bibliotheken eingesetzt wurden, gesucht, diese dann mit
Punkten bewertet und die Ergebnisse dieser Bewertung statistisch mit
Werten wie dem Status der jeweiligen Universität, der politische
Affiliation der Gouverneur*innen und dem Etat der Trägereinrichtung in
Beziehung gesetzt. Es zeigte sich, dass alle Bibliotheken während der
Pandemie Massnahmen ergriffen haben und dass sich die untersuchten Werte
praktisch nicht darauf auswirkten, wie viele und welche dies waren.
(Einzuschränken ist allerdings, wie die Autor*innen vermerken, dass die
Mitgliedsbibliotheken der ARL alle einen relativ hohen Etat haben.)
Einzig bei der Frage, ob die Gouverneur*innen republikanisch oder
demokratisch waren (nicht für die kanadische Bibliotheken gültig),
zeigte sich, dass es tendenziell wahrscheinlicher war, dass eine
Bibliothek weniger Massnahmen ergriff, wenn der Bundesstaat während der
Pandemie \enquote{republikanisch} war. Es waren also wohl politisch
beeinflusste Entscheidungen. Hier wäre es selbstverständlich wieder
interessant zu schauen, ob sich diese Ergebnisse auch im DACH-Raum
reproduzieren lassen.

Der zweite Grund, den Text hervorzuheben, ist die Rahmung. Am Anfang und
am Ende erinnern die Autor*innen zurecht daran, dass Bibliotheken aus
der Influenza-Pandemie vor hundert Jahren hätten lernen können, wie auf
weitere Pandemien zu reagieren wäre, es aber kaum getan haben. Es wäre
also einfach, die einmal gemachten Erfahrungen wieder zu vergessen.
Vielmehr sei es nötig, COVID-19 zum Anlass zu nehmen, die Reaktionen von
Bibliotheken in zukünftigen Pandemien besser voraus zuplanen. (ks)

\begin{center}\rule{0.5\linewidth}{0.5pt}\end{center}

Cohn, Sarah ; Hyams, Rebecca (2021). \emph{Our Year of Remote Reference:
COVID19's Impact on Reference Services and Librarians}. In: Internet
Reference Services Quarterly, \hspace{0pt}\hspace{0pt}25 (2021) 4:
127--144, \url{https://doi.org/10.1080/10875301.2021.1978031}
{[}Paywall{]} {[}OA-Version:
\url{https://academicworks.cuny.edu/cc_pubs/847/}{]}

Es ist selbstverständlich ein nicht ganz so guter Witz, von
verschiedenen Wellen von Texten zum Thema COVID-19 und die Bibliotheken
zu sprechen, wie das in dieser Kolumne seit drei Ausgaben gemacht wird.
Aber wenn man diese Bezeichnung auf Inhalte anwendet, kann man schon
feststellen, dass in den ersten Monaten der Pandemie eine bestimmte Form
von Texten erschien, die vor allem berichtete, was einzelne Bibliotheken
getan haben. Diese Texte hatten immer wieder ähnliche Aussagen, blieben
ein wenig im Allgemeinen (beispielsweise wurde oft gesagt, dass digitale
Angebote ausgebaut wurden, aber kaum spezifiziert welche) und erscheinen
heute auch kaum noch.

Wenn das die inhaltlich erste Welle war, stammt der Text von Cohn und
Hyams Teil aus einer zweiten inhaltlichen Welle, in welcher mehr auf die
konkreten Veränderungen eingegangen und sich auch mit möglichen
Konsequenzen auseinandergesetzt wird. Auch das hat seine Grenzen, schon
weil die Pandemie ja noch nicht vorbei ist. Aber die Aussagen sind
konkreter geworden.

Cohn und Hyams führten im Februar 2021 eine Online-Umfrage zu den
Erfahrungen mit der Beratung von Nutzer*innen per Chat im
Bibliothekssystem der City University of New York (mit 25
Hochschulstandorten und 23 Bibliotheken) durch. Es geht also um einen
fokussierten Bereich. Die Antworten zeigen, dass vor der Pandemie Chats
für diese Aufgabe nur zum Teil verbreitet waren, jetzt aber zum
Arbeitsalltag aller, mit einer Ausnahme, Bibliotheken gehören. Die
Ausbildung dafür war eher zufällig (einige Kolleg*innen hatten eine an
früheren Arbeitsorten erhalten, einige bei der Einführung der
Chatsysteme an ihrer Einrichtung vor der Pandemie, viele mussten den
Umgang mit Chat aber im laufenden Betrieb lernen). Die meisten
Kolleg*innen fanden diese Form der Beratung schwieriger als die Beratung
direkt vor Ort, auch weil sie sich umstellen mussten. Einige beklagten
sich darüber, dass die Nutzer*innen fordernder und unhöflicher seien als
bei der Beratung vor Ort. Darüber, ob die Arbeitslast gestiegen oder
gesunken ist, gibt es keinen Konsens. (ks)

\hypertarget{uxf6ffentliche-bibliotheken}{%
\subsection{2.5 Öffentliche
Bibliotheken}\label{uxf6ffentliche-bibliotheken}}

Stejskal, Jan ; Hajek, Petr ; Cerny, Pavel (2021). \emph{A novel
methodology for surveying children for designing library services: A
case study of the Municipal Library of Prague}. In: Journal of
Librarianship and Information Science 53 (2021) 2,
\url{https://doi.org/10.1177/0961000620948568} {[}Paywall{]}

Das Problem, welches in dieser Studie angegangen wurde, ist, wie man
Daten von Kindern über deren Bibliotheksnutzung erheben kann. Dem stehen
immer wieder unterschiedliche Barrieren im Weg. Gelöst wird dies hier
erstens durch die Verbindung von ethnologisch orientierten Beobachtungen
und zweitens einer angepassten Umfrage. Diese Umfrage wird von den
Kindern beantwortet, aber mit Hilfe der jeweiligen erwachsenen
Begleitperson. Hiermit wird unter anderem die Frage, wie man überhaupt
Einverständnis für die Teilnahme von unter 18-jährigen erlangen kann,
gelöst und gleichzeitig die Frage, wie man eine vertrauensvolle
Atmosphäre für die Kinder schafft. Zudem ist die Umfrage grafisch als
Labyrinth gestaltet, bei dem jede Frage einen Punkt auf dem Weg durch
das Labyrinth darstellt, was dem Layout von Kinderbüchern angepasst ist.
Die Studie zeigt also, dass eine solche Befragung möglich ist.

Die konkreten Ergebnisse ergeben, dass zumindest in Prag der Hauptgrund
für den Besuch einer Öffentlichen Bibliothek für Kinder die Ausleihe von
Büchern ist, gefolgt vom Spielen in der Bibliothek. Auch
Bibliotheksbesuche aus anderen Gründen werden von Kinder für die
Buchleihe benutzt. (ks)

\begin{center}\rule{0.5\linewidth}{0.5pt}\end{center}

Rodger, Joanne ; Erickson, Norene (2021). \emph{The Emotional Labour of
Public Library Work}. In: Partnership 16 (2021) 1,
\url{https://doi.org/10.21083/partnership.v16i1.6189}

Eine weitere Umfrage, diesmal mit offenen Antwortmöglichkeiten und unter
Personal in kanadischen Öffentlichen Bibliotheken, zeigt, dass
\enquote*{emotional labor} insbesondere bei der Arbeit mit Nutzer*innen
nicht nur einen wichtigen Teil der Arbeit des Personals ausmacht,
sondern einen, der massiven Einfluss auf sowohl die positive als auch
die negative Bewertung dieser Arbeit hat. Es ist für viele
Bibliothekar*innen wichtig, direkten Kontakt zu Nutzer*innen zu haben,
aber gleichzeitig lösen Konflikte mit einigen Nutzer*innen auch die
grösste psychische Belastung für die meisten Befragten aus. Der Beitrag
weist darauf hin, dass bislang in Kanada wenige Bibliotheken dies
konkret angehen und zum Beispiel ihrem Personal aktiv Unterstützung bei
der Bewältigung solcher Probleme und Belastungen bieten. Erwähnt werden
vor allem einige Fortbildungen zum Umgang mit \enquote{schwierigen
Personen}, aber wenig darüber hinaus. In einigen Bibliotheken gibt es
informelle Unterstützung des Personals untereinander. Die Autorinnen
fordern -- mit Verweis auf die Krankenpflege, die dies besser
organisiert hätte -- \enquote*{emotional labor} ernster zu nehmen. (ks)

\begin{center}\rule{0.5\linewidth}{0.5pt}\end{center}

Li, Xiaofeng (2021). \emph{Young people's information practices in
library makerspaces}. In: JASIST 72 (2021) 6: 744--758,
\url{https://doi.org/10.1002/asi.24442} {[}Paywall{]}

Erstaunlich unterrepräsentiert in der bibliothekarischen Literatur ist,
was in den Makerspaces, welche Öffentliche Bibliotheken in den letzten
Jahren eingerichtet haben, tatsächlich passiert. Es gibt viele
Versprechen, unter anderem, dass sie informelles Lernen nach
konstruktivistischen Prinzipien fördern würden, aber überprüft werden
diese kaum. Die Studie von Li ist eine Ausnahme. Hier wurden die
Interaktionen von Jugendlichen in zwei Makerspaces (eine in einer
Öffentlichen Bibliothek, eine in einer Schulbibliothek, beide im eher
ländlichen Raum in New Jersey) beobachtet und analysiert. Was sich
zeigt, ist, dass die Jugendlichen, welche diese Makerspaces besuchen,
wirklich auf unterschiedliche Weise mit Informationen interagieren,
diese untereinander austauschen oder im Bedarfsfall suchen. Zudem
arbeiten sie tatsächlich in Projekten, bei denen sie auch bei Fehlern
weitermachen, bis sie diese gemeistert haben. (Ob sie de facto dabei
etwas lernen und auch besser, als in der formalen Bildung, wie dies
vorhergesagt wird, wurde nicht untersucht.) Voraussetzung dafür ist aber
eine grosse Vertrautheit der Jugendlichen miteinander und auch mit den
jeweiligen Bibliothekar*innen und anderen Personen (zum Beispiel
Lehrpersonen in der Schule) sowie regelmässige Besuche des jeweiligen
Makerspaces. (ks)

\begin{center}\rule{0.5\linewidth}{0.5pt}\end{center}

Van Melik, Rianne ; Merry, Michael S. (2021). \emph{Retooling the public
library as social infrastructure: a Dutch illustration.} In: Social \&
Cultural Geography {[}Latest Articles{]},
\url{https://doi.org/10.1080/14649365.2021.1965195}

In dieser Studie wird über einen soziologischen Zugang versucht zu
verstehen, wie Öffentliche Bibliotheken als \enquote{soziale
Infrastrukturen} wirken. Dabei wird eine kleine niederländische
Einrichtung als Case Study untersucht, in welcher eine der Autorinnen
ehrenamtlich tätig ist. Die Forschung fand über Beobachtungen, formelle
und informelle Interviews sowie der Auswertung offizieller Dokumente
statt. Es ist also eine sehr niederländische oder gar spezifisch lokale
Situation, die hier berichtet wird.

Die Autorinnen betonen, dass es eine Veränderung der Bibliothek hin zu
einer Einrichtung gegeben hätte, welche Treffen zwischen Menschen
ermöglicht und damit eine soziale Funktion einnimmt. Allerdings wäre es
falsch, die Möglichkeit von Treffen alleine schon als wirksam zu
begreifen. Nur, weil sich Menschen treffen könnten, hiesse dies noch
nicht, dass Menschen auch miteinander kommunizieren oder längerfristige
Verbindungen aufbauen. Dies müsste erst, beispielsweise mit gesonderten
Veranstaltungen, vorangetrieben werden. Ansonsten sei oft zu beobachten,
dass Menschen zwar gemeinsam den gleichen Ort Bibliothek nutzen, aber
ohne miteinander zu interagieren. Die Autorinnen zeigen, dass es solche
wirkungsvolle Treffen geben kann und dass Bibliothekar*innen die Aufgabe
übernehmen, Veranstaltungen zu organisieren, welche diese ermöglichen.
Aber sie warnen auch davor, in eine Art Sozialromantik zu verfallen und
zu behaupteten, dass solche Treffen ständig und erfolgreich stattfinden
würden.

Zu kritisieren ist, dass -- wie so oft auch in der bibliothekarischen
Literatur -- einfach behauptet wird, dass es eine Veränderung in der
Arbeit der Bibliothek und der Nutzung gegeben hätte, ohne dies weiter
nachzuweisen. (ks)

\hypertarget{wissenschaftliche-bibliotheken-research-data-management-fair-prinzipien-open-science-open-access}{%
\subsection{2.6 Wissenschaftliche Bibliotheken: Research Data Management,
FAIR-Prinzipien, Open Science, Open
Access}\label{wissenschaftliche-bibliotheken-research-data-management-fair-prinzipien-open-science-open-access}}

Stille, Wolfgang, Farrenkopf, Stefan, Hermann, Sibylle, Jagusch, Gerald,
Leiß, Caroline, \& Strauch-Davey, Anette (2021).
\emph{Forschungsunterstützung an Bibliotheken: Positionspapier der
Kommission für forschungsnahe Dienste des VDB}. O-Bib. Das offene
Bibliotheksjournal / Herausgeber VDB, 8(2), 1--19.
\url{https://doi.org/10.5282/o-bib/5718}

Der Term \enquote{Research Support} oder \enquote{Research Support
Services} scheint andernorts längst fest etabliert als (neue) Domäne
wissenschaftlicher Bibliotheken; in der deutschen Bibliothekswelt
scheint der Term bisher aber noch recht neu. Das Positionspapier der
VDB-Kommission für forschungsnahe Dienste (gegründet 2018) gibt eine
Einführung, was unter \enquote{forschungsnahen Diensten} zu verstehen
ist (unter anderem Beratung und Services in den Bereichen
Forschungsdatenmanagement, Publikationsdienste,
Autor*innenidentifikation, Szientometrie, Systematic Reviews) und gibt
einen Einblick in das mögliche Aufgabenspektrum in diesen Teilbereichen.
Zu letzterem liefert das Papier vor allem auch Argumente, warum
Bibliotheken derartige Service einführen beziehungsweise ausbauen
sollten. (mv)

\begin{center}\rule{0.5\linewidth}{0.5pt}\end{center}

Journal of eScience Librarianship (2021). \emph{Data Curation in
Practice}. In: Journal of eScience Librarianship 10 (2021) 3,
\url{https://escholarship.umassmed.edu/jeslib/vol10/iss3/}

In dieser Ausgabe des Journal of eScience Librarianship sind Artikel aus
US-amerikanischen und kanadischen Universitätsbibliotheken versammelt,
welche ihre Praktiken im Forschungsdatenmanagement präsentieren. Zu
lernen ist aus ihnen, dass sich die Herausforderungen und Lösungsansätze
gleichen. Insoweit ist es auch richtig, sie in einer Ausgabe zu
versammeln, aber es stellt sich die Frage, ob dann nicht dazu
übergegangen werden sollte, auch gemeinsame Lösungen zu finden.

Das gesamte Forschungsdatenmanagement, welches hier besprochen wird, ist
so aufgebaut, dass ein Grossteil der Arbeit in der \enquote{Curation}
von Bibliotheken und nicht von Forschenden geleistet wird, auch wenn es
immer wieder den Hinweis gibt, dass Forschende ein Interesse daran haben
müssten, weil es ihnen von Förderern vorgeschrieben wird. Ein wenig
drängt sich der Eindruck auf, als ob Bibliotheken die Aufgabe ernster
nehmen. Als Lösungen werden verschiedene Workflows und Handbücher, aber
auch einzelne Tools präsentiert. (ks)

\begin{center}\rule{0.5\linewidth}{0.5pt}\end{center}

{[}Schwerpunktthema dieser Ausgabe{]} Sengupta, Papia. (2021).
\emph{Open access publication: Academic colonialism or knowledge
philanthropy?}. In: Geoforum. 118 (2021), 203--206.
\url{https://doi.org/10.1016/j.geoforum.2020.04.001} {[}Paywall{]}

Papia Sengupta kritisiert und differenziert das Argument von Open Access
als egalitäres Verfahren der wissenschaftlichen Kommunikation, indem sie
auf die Fortschreibung von Exklusions- und Benachteilungsmechanismen
hinweist. Es spielt für die Teilhabe, so das Argument, in der Praxis
durchaus eine erhebliche Rolle, wo die Autor*innen beheimatet sind. Dazu
kommen ökonomische, politische und Gender-spezifische Aspekte, die
darauf einwirken, dass beispielsweise die Wissenschaft im globalen Süden
auch bei Open Access benachteiligt bleibt. Die Autorin schreibt daher
auch von neo-kolonialistischen Effekten beziehungsweise im Anschluss an
Walter Mignolo einem akademischen Kolonialismus (\enquote{academic
colonialism}). Abstrakt bedeutet dies, dass Forschende in der westlichen
Welt besseren Zugang zu Forschungsmöglichkeiten, -förderungen und
-netzwerken haben und daher bis hin zum Setzen der Forschungsagenda
sichtbarer und einflussreicher sind. So werden Publikationen aus dem
globalen Süden bereits seltener beispielsweise im Web of Science oder
Scopus verzeichnet, unter anderem weil dort häufig nicht digital und
nicht auf Englisch publiziert wird. Dies wird dadurch noch verschärft,
dass die großen Wissenschaftsverlage, die Publikationsstandards,
Ranking-Verfahren und die Festlegung von Impact-Faktoren prägen, sich in
Europa und den USA befinden und weitgehend von den dort üblichen
Rahmenbedingungen und Konventionen für wissenschaftliche Arbeit
ausgehen. Entsprechend reproduziert sich auch bei Open Access, wie es
bislang weitgehend praktiziert wird, die benannte Sichtbarkeits- und
damit Wirksamkeitshürde. Papia Sengupta schlägt drei Lösungen vor, um
diese Effekte abzufedern: Erstens sollen relative APCs eingeführt
werden, die Abstufungen enthalten, bei denen Autor*innen des globalen
Nordens und Forschungsförderung höhere APCs zahlen und Autor*innen aus
dem globalen Süden und ohne gesonderte Förderung frei von APCs
publizieren können. Zweitens sollen Ko-Autor*innenschaften mit
Autor*innen des globalen Südens gezielt angeregt und unterstützt werden.
Drittens sollten die großen Wissenschaftsverlage gezielt Zweigstellen
und Publikationen in den Ländern des globalen Südens aufsetzen. Und
schließlich ruft die Autorin dazu auf, die Vorstellung von Wissenschaft
zu differenzieren und anzuerkennen, dass Prozesse der Wissensproduktion
immer kontextabhängig sind. Bei einem Ideal der gleichen Teilhabe durch
Open Access sind zudem angesichts fortwirkender Effekte eines
\enquote{akademischen Kolonialismus} entsprechend bewusste
Ausgleichsbemühungen notwendig. (bk)

\hypertarget{monographien-und-buchkapitel}{%
\section{3. Monographien und
Buchkapitel}\label{monographien-und-buchkapitel}}

\hypertarget{vermischte-themen-1}{%
\subsection{3.1 Vermischte Themen}\label{vermischte-themen-1}}

Eckert, Rainer (2019). \emph{Archivare als Geheimpolizisten: Das
Zentrale Staatsarchiv der DDR in Potsdam und das Ministerium für
Staatssicherheit}. Leipzig: Leipziger Universitätsverlag, 2019
{[}gedruckt{]}

In dieser Studie wird, hauptsächlich auf der Basis von überlieferten
Quellen des Ministerium für Staatssicherheit (Stasi) selber, die
Überwachung des Zentralen Staatsarchivs der DDR durch diesen
Geheimdienst rekonstruiert. Eine Einordnung in Kontext und
Forschungsstand findet vor allem über die Fussnoten statt, aber der
Fokus des Textes liegt bei einzelnen Inoffiziellen Mitarbeiter*innen
(IM), bei der Führungskraft innerhalb des Geheimdienstes, welche den
Grossteil der Überwachung koordinierte und bei der Überwachung zweier
Direktoren des Archivs, darunter Karl Schirdewan, welcher zuvor aus dem
Zentralkomitee der SED ausgeschlossen wurde. Das Archiv wurde von der
Stasi als \enquote{sicherheitsrelevant} angesehen, weil es Quellen zur
deutschen Geschichte enthielt, insbesondere aus dem Nationalsozialismus,
und weil es auch von Forschenden aus dem \enquote{nicht-sozialistischen
Ausland} genutzt wurde. Gezeigt wird in der Studie, dass der
Geheimdienst durchgängig Informationen über die Arbeit im Archiv
sammelte und dafür IMs betreute, aber wenig aktiv eingriff.

Der Autor war, wie er in der Einführung erwähnt, selber als Student der
Archivwissenschaft und Geschichte von einer Relegation betroffen, die im
Zusammenhang mit IMs stand, welche später auch im Zentralen Staatsarchiv
tätig waren. Deshalb sei er an diesem Thema interessiert gewesen.
Erfreulicherweise schlägt sich das nicht in einer subjektiven Sicht
nieder (die verständlich gewesen wäre), Auseinandersetzungen werden nur
wenige und dann in den Fussnoten geführt. Man merkt, dass der Autor
Historiker ist. Allerdings verbleibt er stattdessen sehr eng an den
Quellen selber und zitiert durchgängig direkt aus den Berichten der
Stasi. Im Nachwort vermerkt er, dass solche Quellen einer intensiven
Kritik unterzogen werden müssten, aber in der Studie vermittelt er den
Eindruck, als könne man die internen Berichte des Geheimdienstes für
bare Münze nehmen.

Vorgestellt wird diese Publikation hier aber aus einem anderen Grund:
Richtig bemerkt der Autor, dass für bestimmte Bereiche der
DDR-Gesellschaft der Einfluss der Stasi schon tiefgehend untersucht
wurde, aber nicht für das Archivwesen. Dies kann man ebenso für das
Bibliothekswesen vermerken. Die Frage, wie sehr und mit welcher Wirkung
die Stasi Archive \enquote{bearbeitete}, ist bislang ebenso ungeklärt
wie für Bibliotheken. (ks)

\begin{center}\rule{0.5\linewidth}{0.5pt}\end{center}

Babb, Sarah (2020). \emph{Regulating Human Research: IRBs From Peer
Review to Compliance Bureaucracy}. Stanford: Stanford University Press,
2020 {[}gedruckt{]}

Diese Studie bietet Denkanregungen dafür, wie und wieso sich Strukturen
nahe bei der Forschung etablieren, welche Ansprüche, die von ausserhalb
der Forschung selbst kommen, umsetzen helfen. Bezogen auf das
Bibliothekswesen ist dabei an Infrastrukturen für Open Access, Open
Science und (dabei besonders) Forschungsdatenmanagement zu denken. Babb
untersucht soziologisch eine ähnliche Struktur: Die Einrichtungen,
welche ethische Regeln in der US-amerikanischen Forschung durchsetzen
sollen. Selbstverständlich gibt es Unterschiede zu Open-Access-Büros
oder Forschungsdatenmanagement-Abteilungen in Europa, aber doch auch
viele strukturelle Gemeinsamkeiten. Babb zeigt, wie innerhalb recht
kurzer Zeit aufgrund politischen Drucks, der nach einer Anzahl von
Skandalen in der medizinischen Forschung aufgebaut wurde, sich
Strukturen entwickelten, die sich bürokratisierten und
professionalisierten. Zuerst als Komitees an einzelnen Hochschulen und
Forschungseinrichtungen organisiert, die von Forschenden selber bestückt
wurden und die Forschungsvorhaben vorgängig auf ethische Fragen hin
beurteilten, entstanden bald personell immer stärker ausgestattete
Strukturen, welche erst diesen Komitees zuarbeiteten, sich dann aber
auch untereinander vernetzten, Tools (vor allem Formulare), ein eigenes
Wissen über ihre Aufgaben und eigene Prozeduren entwickelten. Daran
anschliessend etablierten sich \enquote{unabhängige} Boards, die sich
als Firmen konstituierten und solche Bewertungen gegen Gebühren
übernehmen. Es etablierte sich so schnell, dass die \enquote{Compliance}
mit ethischen Vorgaben der staatlichen Forschungsförderer Kosten
verursacht, die von Forschungseinrichtungen und -- über die
Forschungsförderung -- von den Förderern übernommen werden.

Ähnliche Entwicklungen lassen sich heute im DACH-Raum, wie gesagt, bei
Open Access oder dem Forschungsdatenmanagement beobachten: Zuerst
undenkbar, ist heute, mit steigenden Forderungen durch die
Forschungsförderer und die Forschungspolitik, die Kostenübernahme zur
Normalität geworden. Die Arbeit in diesen Bereichen wurde bürokratisiert
und professionalisiert -- beispielsweise in Open-Access-Büros und
regelmässigen Treffen wie den Open-Access-Tagen. Bislang sind diese
Strukturen an Bibliotheken angegliedert, aber die Studie zeigt auch,
dass dies nicht alternativlos ist. (Babb diskutiert selber am Ende, dass
die US-amerikanische \enquote{Lösung} nicht die einzige ist.) Was Babb
mit ihrer kurzgehaltenen und leicht verständlichen Beschreibung
ermöglicht, ist, die Entwicklungen nicht als gegeben, sondern als
gestaltbar zu verstehen. (ks)

\begin{center}\rule{0.5\linewidth}{0.5pt}\end{center}

McGuinness, Claire (2021). \emph{The Academic Teaching Librarian's
Handbook}. London: Facet Publishing, 2021 {[}gedruckt{]}

Der Begriff Handbuch wird auf sehr unterschiedliche Formen von Büchern
angewandt, so auch hier. Richtiger benannt wäre das Buch als
Selbst-Lehrbuch für Wissenschaftliche Bibliothekar*innen mit
Lehrverpflichtungen im Bereich Informationskompetenz. Es ist darauf
ausgerichtet, wichtige Konzepte sowohl zur Informationskompetenz als
auch zur Lehre und Integration in die bibliothekarische Arbeit zu
vermitteln, inklusive regelmässig eingefügten Aufgaben für
Bibliothekar*innen, welche diese Aufgaben übernehmen werden.

Selbstverständlich ist es auf den britischen und US-amerikanischen
Kontext ausgerichtet, insoweit werden vor allem dort verbreitete
Konzepte und Standards vermittelt oder zum Beispiel mit dem Thema
\enquote{learning analytics} Dinge besprochen, die im DACH-Raum keine
grosse Relevanz haben. Dennoch kann es seine Aufgabe, das selbstständige
Einarbeiten in diese Aufgaben, erfüllen, wenn es in dieser Weise -- also
inklusive des Erfüllens der Aufgaben -- genutzt wird. Bestimmte Punkte
müssen dann wohl übersprungen oder Standards aus dem DACH-Raum selber
ergänzt werden. Sichtbar wird, dass die Autorin den Unterricht in der
Bibliothek als eine ständig zu planende, zu reflektierende und neu
anzupassende Aufgabe betrachtet und sich am
Professionalisierungsverständnis von Lehrer*innen orientiert. Nicht nur
bei den Lernenden, sondern auch bei den Lehrenden soll mit diesem Buch
ein Verständnis von Wissen und Informationen als kontextualisiert und
ständig im Wandel etabliert werden. (ks)

\begin{center}\rule{0.5\linewidth}{0.5pt}\end{center}

Priestner, Andy (2021). \emph{A handbook of user experience research \&
design in libraries}. Lincolnshire: UX in Libraries, 2021 {[}gedruckt{]}

Berater*innen, auch im Bibliotheksbereich, prägen oft für ihre Angebote
möglichst einprägsame Namen. Andy Priestner hat als solcher in
Grossbritannien den Begriff \enquote{UX in Libraries} geprägt, unter dem
er auch Workshops und Consulting anbietet. Grundsätzlich ist dies genau
dasselbe, was andere Berater*innen im DACH-Raum unter dem Begriff
\enquote{Design Thinking} anbieten: Es geht darum, Veränderungen in
einer Bibliothek mit den immer gleichen Schritten (Discover, Define,
Develop, Deliver) anzugehen und am Ende des Prozesses ein einsetzbares
Produkt, zum Beispiel ein neues Angebot oder einen neuen
Bibliotheksraum, zu liefern. Austauschbar sind dann die Methoden, welche
in den einzelnen Schritten eingesetzt werden. Teil des Prozesses ist es,
dass Bibliothekar*innen direkt an diesen beteiligt werden und
beispielsweise im Rahmen des Schrittes \enquote{Define} selber
Interviews mit Nutzer*innen führen.

Was dieses Handbuch liefert, ist eine ausführliche Darstellung möglicher
Methoden -- wobei die meisten nicht neu sind -- immer sehr
praxisorientiert geschrieben. Grundsätzlich kann mit diesem Buch jede
Bibliothek selber ihre \enquote{UX in Libraries}-Prozesse (oder halt
Design-Thinking-Prozesse) aufsetzen.

Die Probleme des Buches sind die, welche solche Herangehensweise immer
prägen: Es wird einfach behauptet, dass Design ein sinnvolles Vorbild
für die Entwicklung von Angeboten von Bibliotheken wäre, obwohl das
empirisch und theoretisch schwerlich haltbar ist. Zudem wird durchgehend
behauptet, dass die Sicht der Nutzer*innen eingenommen werden müsse, um
sinnvolle Produkte erstellen zu können, aber gleichzeitig wird zum
Beispiel einfach darüber hinweggegangen, dass im Prozess dieser Blick
immer von den Bibliothekar*innen interpretiert wird, egal, wie viele
Umfragen durchgeführt werden. Auch in diesem Buch wird Theorie gering
geachtet und als hinderlich dargestellt. Und, was unter all dem Text im
Buch untergeht, ist, dass nicht geklärt wird, wie sich Bibliotheken
überhaupt verändern. Es wird die Illusion aufgebaut, dass es reicht, den
definierten Prozess durchzuführen, um zu einer Veränderung zu kommen. Ob
das Buch zu empfehlen ist, hängt sehr davon ab, welche Position man
überhaupt zu solchen Prozessen wie UX oder Design Thinking für
Bibliotheken einnimmt. Es repräsentiert aber sehr gut, was der Autor als
Berater Bibliotheken anbietet. (ks)

\begin{center}\rule{0.5\linewidth}{0.5pt}\end{center}

Chen, Amy Hildreth (2020). \emph{Placing Papers: The American Literary
Archives Market.} (Studies in Print Culture and the History of the Book)
Amherst ; Boston: University of Massachusetts Press, 2020 {[}gedruckt{]}

Der Rezensent unterrichtet seit Jahren das Fach
\enquote{Bestandsmanagement}. In diesem will er den Studierenden unter
anderem vermitteln, dass Bibliotheken durch ihren stetigen Etat auch
Marktteilnehmer sind und sich deshalb um sie herum Firmen etabliert
haben, welche ihre Geschäftsmodelle explizit auf die Geldmittel
ausgerichtet haben, welche Bibliotheken regelmässig mobilisieren können.
Dabei geht es ihm nicht darum, das abzulehnen, sondern die reale
Situation zu zeigen, in der die Studierenden in ihrem Arbeitsleben
agieren werden -- inklusive der Möglichkeit, dass sie nach dem Studium
bei solchen Firmen arbeiten könnten.

Das Beispiel, welches der Rezensent dafür oft heranzieht, ist der Markt
mit Vor- und Nachlässen von Schriftsteller*innen in
Universitätsbibliotheken, welcher sich in den USA entwickelt hat. Das
Buch von Chen stellt diesen Markt in einer leicht zugänglichen Sprache
anhand der Positionen aller Beteiligten vor (Autor*innen und deren
Familien beziehungsweise Nachlassverwalter*innen, Bibliotheken,
Universitäten und Stiftungen, Literaturagent*innen und -händler*innen,
die Personen in den Bibliotheken, die konkret an der Auswahl,
Beschreibung und Aufbereitung der Nachlässe beteiligt sind sowie
Forschende, die mit den Nachlässen wissenschaftlich arbeiten). Sie
beschreibt die Rollen, die alle in diesem System \enquote{spielen},
inklusive der Entwicklungen seit den 1950er-Jahren und der Diskurse, die
in den verschiedenen Sphären (also beispielsweise zwischen Bibliotheken
selber) geführt werden. Dabei zeigt sie, dass es zwar überall Diskurse
gibt, welche diesem System zuschreiben wollen, dass es den Beteiligten
um etwas anderes geht als um Geld, aber, dass es am Ende doch ein Markt
ist wie andere, der auch nach den gleichen Marktprinzipien funktioniert.
Das Vorgehen erinnert wohltuend an die Arbeiten von Sarah Thornton
(\emph{Seven Days in the Art World} (2008), \emph{33 Artists in 3 Acts}
(2014)), welche das Gleiche ähnlich zugänglich für den Kunstmarkt
zeigte.

Chen postuliert, dass die steigenden Preise in Zukunft dazu führen
werden, dass sich der Markt weiter in die Autor*innen, welche
erfolgreich ihre Nachlässe platzieren können und diejenigen, die damit
Probleme haben werden, aufteilen wird. Mit einer empirischen Auswertung
zeigt sie weiterhin, dass trotz einiger Veränderungen rassistische und
sexistische Strukturen in diesem Markt weiterhin existieren und damit
weisse Autoren es leichter haben werden, zu den Gewinnern zu zählen, als
andere Autor*innen. Gleichzeitig sieht sie eine Tendenz hin dazu, dass
die Nachlässe eher in eigenen Institutionen untergebracht werden, die
sich, von Stiftungen finanziert, als eigenständige Kultur-, Freizeit-
und Forschungseinrichtungen etablieren. Sammlungen in
Universitätsbibliotheken, die bislang den Markt getrieben hätten, würden
weniger relevant werden.

Der Rezensent wird dieses Buch ab jetzt als Einführung empfehlen. Es
zeigt gut die Position von Bibliotheken als Teil eines Marktes, auch
wenn vieles US-spezifisch ist und sich so nicht ganz im DACH-Raum
wiederfindet. (ks)

\begin{center}\rule{0.5\linewidth}{0.5pt}\end{center}

\pagebreak

Spiekermann, Sarah (2019): \emph{Digitale Ethik -- Ein Wertesystem für
das 21. Jahrhundert}. München: Droemer Verlag, 2019 {[}gedruckt{]}\\
Sarah Spiekermann beschreibt in ihrem Buch ihre Erfahrungen im Silicon
Valley als Wirtschaftsinformatikerin und wie sie während dieser Zeit
begann, ihre eigenen Werte und die Werte der Tech-Branche, aber auch die
Werte, die im Digitalen herrschen, zu hinterfragen. Sie stellt sich die
Frage, wie eine menschengerechte Digitalisierung aussehen sollte und
warum uns dabei der ewige Fortschrittsgedanke im Weg stehen könnte. Sie
entwirft ein neues Wertesystem, welches dabei helfen soll, das Wohl der
Menschen in der digitalen Welt nicht aus den Augen zu verlieren. Im
Kapitel \enquote{Digitale Ethik in der Praxis} gibt sie Tipps und teilt Ideen,
wie neue Werte integriert, gelebt und weitergegeben werden können. Das
Buch ist auf der einen Seite sehr philosophisch und auf der anderen sehr
praxisorientiert. Es liest sich gut und bietet viele interessante
Blickwinkel auf ein Thema, mit dem wir uns alle auseinandersetzen
sollten und welches uns wahrscheinlich in Zukunft mehr und mehr
beschäftigen wird. Es ist gerade auch für Menschen, die im
Bibliothekswesen und/oder der Informationswissenschaft verortet sind,
interessant, da die Digitalisierung und die digitale Information diese
Bereiche so grundlegend verändert. (sj)

\begin{center}\rule{0.5\linewidth}{0.5pt}\end{center}

Arns, Inke; Lechner, Marie (Hrsg.) (2021): \emph{Computer Grrrls -- HMKV
Ausstellungsmagazin 2021/01}. Bönen/Westfalen: Verlag Kettler, 2021
{[}gedruckt{]}\\
In diesem Ausstellungsmagazin kommen 23 Künstler*innen zu Wort, die das
Zusammenspiel von Technologie und Geschlecht aus verschiedenen
Blickwinkeln beleuchten und in neue Beziehungen zueinander setzen. Die
Beiträge beschäftigen sich mit dem Verhältnis zwischen Frauen und
Technik, historisch, künstlerisch und kritisch. Neben Fotos der
Ausstellung, Beschreibungen der einzelnen Künstler*innen und einer
\enquote{Computer-Grrrls-Timeline} befinden sich auch fünf Essays in dem
Buch. Diese setzen sich mit verschiedenen Aspekten der
Weiblich/Technik-Beziehung auseinander. In dem Essay von Claire L. Evans
mit dem Titel \emph{Computer Grrrls} wird die Geschichte der Computer
und des Internets aufgerollt und die weibliche Beteiligung daran in den
Vordergrund gerückt. Viel zu oft wird die wegweisende Arbeit vieler
Frauen in dieser Geschichte vergessen oder nicht erwähnt (soweit ich
mich erinnern kann, habe ich davon in meinem Studium auch nichts
gehört). Representation matters -- und genau deswegen ist es umso
schöner, in dem Text von Evans zu lesen: \enquote{[\ldots] bis hin zu den
Teams von Informationswissenschaftlerinnen und Bibliothekarinnen, die
dafür sorgten, dass die früheste Version des Internets, das ARPANET,
funktional, elegant und durchsuchbar blieb.} Ein Buch, das zum
Entdecken einlädt, auch wenn man die Ausstellung nicht besucht hat. (sj)

\begin{center}\rule{0.5\linewidth}{0.5pt}\end{center}

Wikimedia Deutschland (Hrsg.) (2021): \emph{Offen und gerecht! Anstöße
zur (selbst-)kritischen Reflexion von Biases im Wissenschaftssystem}.
\url{https://lernraumfreieswissen.de/lessons/offen-und-gerecht-anstoesse-zur-selbst-kritischen-reflexion-von-biases-im-wissenschaftssystem/}\\
Wikimedia Deutschland (Hrsg.) (2021): \emph{Open and Equitable! Impulses
for (self-)critical reflection on biases in the academic system}.
\url{https://lernraumfreieswissen.de/lessons/open-and-equitable-impulses-for-self-critical-reflection-on-biases-in-the-academic-system/}\\

Im Rahmen einer Reihe von Onlinekursen (\enquote{Lernraum Freies
Wissen}), die aktuell kollaborativ erarbeitet werden, gibt Wikimedia
Deutschland eine Broschüre, oder besser \enquote{Lerneinheit}, heraus,
die sich mit der Frage beschäftigt, wie offene Wissenschaft so gestaltet
werden kann, dass eine gleichberechtigte Mitwirkung und (digitale)
Teilhabe möglich ist. Um dies zu erreichen, müssen -- so die Botschaft
der Broschüre -- Ungleichheiten auf verschiedenen Ebenen überwunden
werden. Das sind namentlich die persönliche, epistemische, systemische
und strukturelle Ebene, für die jeweils eine Reihe von Fragen
präsentiert werden, anhand derer das eigene Denken und die eigene
Forschungstätigkeit geprüft werden kann, um etwa Vorurteile oder
Vorbehalte aufzudecken oder strukturelle Herausforderungen zu
identifizieren. Die Zusammenstellung wird ergänzt um
Literaturempfehlungen. Die Inhalte sind auf Deutsch und Englisch und
jeweils als PDF-Dokument sowie in maschinenlesbarer Form (txt-Format)
verfügbar. (mv)

\hypertarget{politik-und-bibliotheken}{%
\subsection{3.2 Politik und
Bibliotheken}\label{politik-und-bibliotheken}}

Hümmler, Lilian (2021). \emph{Wenn Rechte Reden: Die Bibliothek des
Konservatismus als (extrem) rechter Thinktank}. Hamburg: Marta Press,
2021 {[}gedruckt{]}

Die im Titel genannte Bibliothek des Konservatismus ist eine
Einrichtung, angesiedelt in Berlin, welche sich als
\enquote{meta-politische} Organisation (verstanden nach Carl Schmitt)
der radikalen Rechten begreift. Dies meint, dass sie sich vor allem mit
der Produktion und Verbreitung von Wissen sowie der Vernetzung von
Akteur*innen beschäftigen will, nicht mit der konkreten Politik in --
wie es im Duktus dieser Szene heissen würde -- \enquote{den Parlamenten
und auf der Strasse}. Oder anders gesagt versteht sie sich als
Eliteorganisation, welche den Diskurs der radikalen Rechten bestimmen
will. Die Arbeit analysiert diese Organisation und vor allem Vorträge,
welche in den Räumen derselben gehalten wurden. (Ausgewertet wurden
solche bis 2018, aufgeführt werden mehr.) Dabei wird sichtbar, dass die
Organisation sich als Schnittstelle zwischen rechts-konservativen und
rechts-radikalen Zusammenhängen versteht. Die Autorin zeigt, wie hier
alle zu erwartenden Themenbereiche in Veranstaltungen auftauchen:
Anti-modernistische und anti-demokratische Grundhaltungen, Rassismus,
Sexismus, Antisemitismus und Verschwörungsglauben, Anti-Feminismus und
die explizite Unterstützung christlich-fundamentalistischer
Abtreibungsgegner*innen. Das alles wird im Buch nachvollziehbar
dargestellt und in seinen Konsequenzen diskutiert.

Warum die Arbeit hier besprochen wird, ist aber selbstverständlich der
Fakt, dass sich diese Einrichtung als Bibliothek bezeichnet. Ist sie
das? Die Autorin ist keine Bibliothekarin und unternimmt es deshalb zum
Beispiel auch gar nicht, die Professionalität der bibliothekarischen
Arbeit zu bewerten. Vielmehr thematisiert sie, dass die Einrichtung das
Bild einer Bibliothek als \enquote{Geistesort} zu bedienen versucht. Im
von grossem Pathos geprägten Duktus der Szene wird sie als
\enquote{Bastion} beschrieben, an der sich eine \enquote{geistige Elite}
rüsten würde für einen angeblichen \enquote{Kulturkampf}. Dies geht
einher mit ständigen Überhöhungen nicht nur der Bibliotheken, sondern
zum Beispiel auch des Raumes, der -- obgleich eher beschränkt -- immer
wieder als eine Art Burg beschrieben wird. Gleichzeitig versucht sich
die Bibliothek aber nach aussen auch als ernstzunehmende
wissenschaftliche Einrichtung zu präsentieren. Die Autorin erwähnt zum
Beispiel, dass sie an der \enquote{Langen Nacht der Bibliotheken} 2013
teilnahm, um sich als eine Einrichtung unter vielen darzustellen. (Nicht
erwähnt, aber in Berlin bekannt, sind Versuche der Bibliothek, auch von
anderen Bibliotheken als professionelle Einrichtung akzeptiert zu
werden.) Die Einrichtung ist in Nähe der Technischen Universität und der
Universität der Künste situiert und nutzt -- obgleich in keiner
Verbindung zu den Hochschulen stehend -- akademische Terminologie wie
Semester oder Seminar, um als gleichwertig zu erscheinen. Das alles ist
Teil der gesamten Arbeit der Organisation, die -- so die Autorin --
daran arbeitet, die \enquote{Grenzen des Sagbaren} zu verschieben sowie
anti-humanistische und anti-demokratische Positionen in der
Öffentlichkeit als normale Positionen zu verankern. Der Fakt, dass sie
dabei eine Bibliothek ist -- was sie auch vor allem durch einen Nachlass
wurde -- ist dabei ein Nebenaspekt.

Grundsätzlich ist es dabei nichts Neues, dass politische Bewegungen
Bibliotheken gründen und dass diese Bibliotheken ihren Fokus auf andere
Aufgaben legen als zum Beispiel Universitätsbibliotheken. (Von solchen
Einrichtungen gäbe es alleine in Berlin selber mehrere Beispiele.) Das
Besondere ist hier allerdings, dass sich die Bibliothek des
Konservatismus als unbedenklich, neutral und professionell arbeitende
Einrichtung verstanden wissen will (was andere \enquote{politische
Bibliotheken} oft explizit nicht versuchen, obwohl sie faktisch oft
professioneller arbeiten) und dass sie Teil einer politischen Bewegung
ist, die gegen den humanistischen Minimalkonsens steht, auf den sich
Bibliothekar*innen bei allen Differenzen sonst gut einigen können. (ks)

\begin{center}\rule{0.5\linewidth}{0.5pt}\end{center}

Frances, Sherrin (2020). \emph{Libraries amid protest. Books,
organizing, and global activism.} Amherst ; Boston: University of
Massachusetts Press, 2020 {[}gedruckt{]}

Die Autorin -- keine Bibliothekarin, sondern Assistenz-Professorin in
einem Englisch-Departement -- untersucht \enquote{Protestbibliotheken}.
Das sind hier Bibliotheken, welche während Platzbesetzungen zwischen
2011 und 2016 betrieben wurden. Bekannt ist wohl vor allem die Occupy
Wall Street People's Library (New York), aber die Autorin integriert
auch Bibliotheken aus Madrid (BiblioSol), Oakland (Biblioteca Popular
Victor Martinez), Istanbul (Gezi Park Library), Kiev (Maidan Library),
Paris und Lyon (BiblioDebout) und Chicago (Freedom Square Library).
Ihrer Ansicht nach ist die Zeit dieser Platzbesetzungen aufgrund neuer
polizeilicher Regelungen jetzt vorbei.

Die Fragen, die sie hauptsächlich umtreiben, sind, was eine solche
\enquote{Protestbibliothek} eigentlich ist, warum und von wem sie
betrieben wurden und was sie über das Konzept \enquote{Bibliothek}
aussagen. Sie zeigt, dass alle diese Bibliotheken ihre regionalen und
nationalen Eigenheiten hatten, aber gleichzeitig auch viele
Gemeinsamkeiten. Am deutlichsten wird dies bei der Bibliothek in Kiev,
die -- im Gegensatz zu den anderen Bibliotheken, die eher mit Protesten
mit klassisch \enquote{linken} Themen wie Kapitalismuskritik,
Antirassismus, Proteste gegen Polizeigewalt oder Abwendung von
Gentrifizierung zu tun hatten -- im Rahmen eines Protestes betrieben
wurde, bei dem es praktisch um die Neudefinition der nationalen
Identität der Ukraine ging.

Fast alle Bibliotheken entstanden spontan. In New York sei zum Beispiel
ein Berg Bücher abgelegt worden, jemand schrieb dazu einen Zettel
\enquote{Library} und dann nahmen sich wieder andere Personen diesem
Berg Bücher an und begannen, eine Bibliothek einzurichten. Die Personen,
welche dies taten, entwickelten eine Identität als
\enquote{Bibliothekar*innen}, meist ohne bibliothekarische Ausbildung.
(Aber mit Ausnahmen. In Frankreich waren es eher ausgebildete
Bibliothekar*innen, in New York gab es auch viele Kolleg*innen, die sich
engagierten.) Die Bibliotheken entwickelten sich zu Leseräumen auf den
Plätzen und zu Buchumschlagplätzen. Was Menschen zu diesen Bibliotheken
zog, waren die physischen Bücher. Es gab immer ein System der
\enquote{Ausleihe}, aber nie eine Möglichkeit, die Medien
zurückzufordern. Zudem trafen immer wieder neue Buchspenden ein, von
Privatpersonen, Autor*innen und Verlagen. Die Bibliotheken erreichten
schnell einen Bestand von mehreren tausend Büchern. Systematik und
Kataloge gab es, wenn überhaupt, nur rudimentär. Die meisten
Bibliotheken überlebten die Räumung der jeweiligen Plätze und mussten
sich der Frage stellen, wie sie permanent weiter existieren sollten.
Einige schafften dies, zum Beispiel ist die Bibliothek in Madrid heute
Teil eines sozialen Zentrums. Andere nicht.

Die Autorin stellt die Protestbibliotheken zudem immer wieder in
Kontrast zu Public Libraries in den USA und fragt -- auch weil es die
Kritik gab, dass das \enquote{keine richtigen Bibliotheken} wären --,
was sie über die Möglichkeiten von Bibliotheken aussagen. Ihre Analyse
orientiert sich am Konzept der \enquote{Prefiguration}, welches eher aus
dem anarchistischen Denken stammt und vom Ethnologen David Graeber bei
der Analyse von Occupy Wall Street stark gemacht wurde: Die
Platzbesetzungen seien ein Vorgriff auf die Gesellschaft gewesen, in der
die Protestierenden leben wollten, praktisch ein Versuch, schon in der
besseren Zukunft zu leben. Die Bibliotheken, wie sie existierten, wären
dadurch auch ein Versuch gewesen, auf die Zukunft zuzugreifen. Als Idee
seien sie freier gewesen, als es die Public Libraries, die eingebunden
sind in Politik und Geschichte der USA, je sein könnten. Deshalb hätten
sie bestimmte Versprechen wie Demokratisierung des Zugangs zu Wissen,
aber auch, einen Raum anzubieten, um zur Ruhe zu kommen, aus Interesse
zu lesen und die Zeit zu verlangsamen, besser umsetzen können, als es
Public Libraries je tun könnten. (ks)

\hypertarget{mental-illness}{%
\subsection{3.3 Mental Illness}\label{mental-illness}}

Brown, Robin ; Sheidlower, Scott (2021). \emph{Seeking to Understand: A
Journey into Disability Studies and Libraries}. Sacramento: Library
Juice Press, 2021 {[}gedruckt{]}

Die persönliche Motivation der beiden Autor*innen ist, dass sie es
selber als Bibliothekar*innen mit Behinderungen wertvoll fanden, sich
forschend mit ihrer Situation auseinanderzusetzen. Dadurch schafften sie
sich Wissen, mit dem sie ihre eigene Situation besser verstehen konnten.
Und gleichzeitig, über eine Umfrage und Interviews, lernten sie, dass
sie nicht allein sind, sondern dass es eine ganze Anzahl von
Kolleg*innen mit sichtbaren und nicht-sichtbaren Behinderungen gibt,
welche in verschiedenen Bibliotheken tätig sind.

In diesem sehr kurzen Buch führen sie in die Literatur zu Disability
Studies und verwandten Themen -- beispielsweise critical race studies
oder feministische Forschung -- ein, stellen die Ergebnisse der
genannten Umfrage und Interviews vor und diskutieren Möglichkeiten des
kritischen Engagements für Bibliothekar*innen mit Behinderungen in
Bibliotheken. Eine Stärke des Buches ist das letzte Kapitel
\enquote{Life Writing}, in dem solche Kolleg*innen aufgeschrieben haben,
was sie ihren eigenen Kolleg*innen und vor allem Manager*innen über ihre
Situation mitteilen wollen. Das liefert bedenkenswerte Einblicke.

Zu kritisieren ist allerdings, dass das Buch unbenannt und unreflektiert
nur die US-amerikanische Situation thematisiert. Diese wird einfach als
gegeben vorausgesetzt. Dies beginnt beim Vorwort und den Beispielen, die
herangezogen werden, aber zeigt sich auch in den vorgeschlagenen
Aktivitäten, die alle auf Etablierung und Durchsetzung legaler Rahmen
hinauslaufen. Andere Politik- und Lösungsansätze, beispielsweise die
autonomen Ansätze der Enthinderungsbewegungen im DACH-Raum, werden gar
nicht erst angedacht. (ks)

\begin{center}\rule{0.5\linewidth}{0.5pt}\end{center}

Dubem, Miranda ; Wade Carrie (edit.) (2021). \emph{LIS Interrupted:
Intersections of Mental Illness and Library Work.} Sacramento: Library
Juice Press, 2021 {[}gedruckt{]}

Dieses Buch und das gerade besprochene können gut aufeinander bezogen
gelesen werden. Beide erschienen im selben Jahr und Verlag. Auch dieses
behandelt, allerdings als Herausgeberinnenwerk in Artikeln verschiedener
Autor*innen, den Komplex Behinderungen und Arbeit in Bibliotheken, wobei
hier auch Themen wie Drogenmissbrauch, Traumaverarbeitung oder
Suizidversuche thematisiert werden. Das Buch ist emotional, in weiten
Teilen schwer zu lesen, weil hier vor allem Kolleg*innen aus
Bibliotheken und einem Archiv sehr persönlich vom Leben mit ihren
Behinderungen, Traumata und vergleichbaren Erfahrungen berichten. Nur
ein kleiner Teil der Beiträge behandelt zum Themenfeld zugehörige Punkte
wie die in der Bibliothekspraxis genutzte Sprache oder Systematiken mit
einem persönlichen Abstand. Dem Grossteil der Beiträge ist aber jeweils
notwendigerweise eine Triggerwarnung vorangestellt.

In den meisten Beiträgen schildern Kolleg*innen nicht nur, wie die
jeweiligen Behinderungen und Traumata (beispielsweise das Überleben des
Anschlags auf den Boston Marathon 2013 oder die archivalische Arbeit mit
Oral History Interviews von Holocaustüberlebenden) ihr Leben und ihre
Arbeit in Bibliotheken beeinflussen, sondern reflektieren auch das
Verhalten von anderen Kolleg*innen, dem Management und, seltener, von
Nutzer*innen. Dabei herrscht sehr klar das in einem frühen Beitrag
diskutierte \enquote{soziale Modell} von Behinderung vor, welches davon
ausgeht, dass es vor allem gesellschaftliche Strukturen sind, welche
bestimmen, ob physische oder psychologische Eigenheiten von Menschen
überhaupt behindernd wirken und wenn ja, wie. Zudem gibt es auch immer
wieder Vorschläge, entweder für ebenfalls Betroffene oder aber für das
Bibliothekswesen, um besser mit den geschilderten Situationen umzugehen.

Was das Buch vermittelt -- hoffentlich -- ist für Betroffene von
Behinderungen und Traumata, dass sie nicht alleine mit ihrer Situation
sind und dass es hilfreich sein kann, die Situation anzuerkennen und zu
benennen. Gleichzeitig zeigt es eine ganz erschreckende Bandbreite an
Behinderungen und Traumata auf, mit denen Kolleg*innen umzugehen haben.
Die Herausgeberinnen haben Beiträge für über 300 Seiten versammeln
können und man kann davon ausgehen, dass dies nur ein ganz kleiner Teil
der tatsächlich Betroffenen ist, die sich hier äussern. Gleichzeitig
zeigt das Buch, dass viele Bibliotheken als Arbeitsumfeld überhaupt
nicht angemessen auf diese Situation reagieren können oder wollen, es
aber gleichzeitig auch Einrichtungen gibt, in denen dies gelingt. Die
Hauptbelastung für die betroffenen Kolleg*innen stellen fast immer
andere Kolleg*innen, das Management oder die Arbeitsbedingungen selber
dar, kaum die Nutzer*innen. Es ist also auch eine Frage des Wollens.

Was dem Buch -- und darin gleicht es auch wieder dem gerade zuvor
besprochenen Werk -- anzumerken ist, ist die US-amerikanische Herkunft.
Nur in Ausnahmen wird die Situation in anderen (angloamerikanischen)
Ländern zum Vergleich herangezogen. Einige Traumata sind wohl im
DACH-Raum kulturell nicht so gravierend, beispielsweise die ständigen
Drills für den Fall eines Amoklaufs, die belastend sein können. Aber wir
wissen es nicht, da es kein vergleichbares Werk für die Situation in
Bibliotheken im deutschsprachigen Raum gibt. Es würde sich lohnen, wie
das Buch auch zeigt, Betroffene reden zu lassen und ihnen zuzuhören.
(ks)

\hypertarget{geschichte}{%
\subsection{3.4 Geschichte}\label{geschichte}}

Poulain, Martine (dir.) (2019). \emph{Où sont les bibliothèques
françaises spoliées par les nazis ?}. (Papiers) Villeurbanne: Presses de
l'Enssib, 2019, \url{https://doi.org/10.4000/books.pressesenssib.7814}

Dieser Band versammelt Beiträge aus einem 2017 durchgeführten Kolloquium
mit Bibliothekar*innen und Forscher*innen aus Frankreich, Deutschland,
Österreich und Weissrussland, das sich mit Fragen der
Provenienzforschung in Bezug auf Frankreich beschäftigte. Es geht dabei
sowohl um Buchbestände, die während der Nazizeit in Frankreich
geplündert wurden als auch um solche Bestände, die heute in
französischen Bibliotheken stehen. Die Beiträge schliessen an dem an,
was man aus der Provenienzforschung vor allem aus Deutschland und
Österreich seit den letzten 20 Jahren weiss. Ergänzt wird das durch zwei
Forschungsstränge: Anatole Stebouraka berichtet über Bücher, die erst
von den Nazis beschlagnahmt, dann am Ende des Zweiten Weltkrieges von
der Roten Armee als \enquote{Trophäen} -- die als Ersatz für
geplündertes Kulturgut in der Sowjetunion verstanden wurden --
requiriert wurden und die heute (noch) in den drei grossen Bibliotheken
in Minsk lagern. Eine ganze Anzahl von Beiträgen geht auch auf Bestände
ein, die aus unterschiedlichen, nicht immer ganz klaren Gründen, in
französischen Bibliotheken liegen, beispielsweise weil sie von den
nationalsozialistischen Behörden \enquote{zwischengelagert} wurden, weil
sie in grossen Rückgabeaktionen Ende der 1940er, Anfang der 1950er an
französische Einrichtungen übergeben, aber von diesen nicht an ihre
ursprünglichen Besitzer*innen oder Nachfolger*innen weitergegeben wurden
oder weil sie, wie die Universitätsbibliothek Strassburg,
nationalsozialistische Einrichtungen darstellen, die nach 1945 (wieder)
vom französischen Staat übernommen wurden.

Der Band ergänzt sehr gut das vorhandene Wissen über die Buchdiebstähle,
welche von den Nazis vorgenommen wurden, aber auch von der
\enquote{institutionellen Amnesie}, die oft nach 1945 in den einzelnen
Einrichtungen über die jeweiligen Bestände herrschte. (Der Band zeigt,
dass dies nicht nur für deutsche und österreichische, sondern auch
französische Bibliotheken galt und dass eine Gegenbewegung erst in den
letzten Jahrzehnten, wieder auch in Frankreich, aufkam.) Er zeigt noch
einmal eindrücklich, dass die Verbrechen der Nazis praktisch ganz Europa
(und weitere Räume) erfassten und deshalb die Aufarbeitung, auch in
Bibliotheken, auf internationaler Ebene erfolgen muss.

Nicht beantwortet wird allerdings die Frage, welche den Titel des Buches
bildet, nämlich, ob französische Bibliotheken von den Nazis geplündert
wurden. Es geht bei den Beispielen vor allem um private Buchsammlungen
und Sammlungen von jüdischen, freimaurerischen oder linken
Organisationen, nicht um die Bibliotheken der grossen staatlichen
Institutionen. (ks)

\begin{center}\rule{0.5\linewidth}{0.5pt}\end{center}

Speer, Andreas ; Reuke, Lars (Hrsg.) (2020). \emph{Die Bibliothek -- The
Library -- La Bibliothèque: Denkräume und Wissensordnungen}.
(Miscellanea Mediaevalia, 41) Berlin, Boston: Walter de Gruyter.
\url{https://doi.org/10.1515/9783110700503} {[}Paywall{]}

In seiner physischen Version erscheint dieser, mit allen Anhängen rund
900 Seiten dicke, Band mit seinem Titel zuerst wie ein Handbuch, das
endgültig alle relevanten Fragen zum Thema klären will. Das ist es aber
nicht. Vielmehr sind hier die Beiträge der 41. Kölner Mediaevistentagung
von 2018 versammelt. Die Masse an Text ist einfach ein Zeichen für die
fast schon beängstigende Produktivität der Mediävistik.

Die einzelnen Kapitel legen den Begriff der Bibliothek weit aus. Es geht
nicht nur um Räume und Institutionen, sondern zum Beispiel auch um die
Rekonstruktion der Bücher, auf die sich mittelalterliche Autoren oder
Autorengruppen für ihre Texte gestützt haben oder um die theologischen
Diskussionen über Möglichkeiten des Bücherbesitzes in Klöstern
bestimmter Orden. Wie zu erwarten, gibt es mit einzelnen Beiträgen auch
Ausnahmen, aber grundsätzlich geht es zeitlich und räumlich um das
europäische Mittelalter. Gegliedert ist der Band grundsätzlich nach
Bibliothekstypen, aber in Teilen anders, als dies in der
bibliothekarischen Bibliothekstypologie geschehen würde:
Klosterbibliotheken, Universitätsbibliotheken, Hofbibliotheken, Stadt-,
Privat- und Missionsbibliotheken werden Bibliothekar*innen bekannt
vorkommen, aber Abschnitte zu Karolingischen Bibliotheken, zur Bibel --
die als Buchsammlung und damit auch als Bibliothek interpretiert wird --
oder \enquote{bibliotheca mystica} weniger. Verwirrenderweise sind im
Abschnitt \enquote{Virtuelle Bibliotheken} Beiträge zu den genannten,
aus Texten rekonstruierten Buchsammlung, auf die die Autoren
zurückgegriffen haben, versammelt und keine Beiträge zu digitalen
Angeboten von Bibliotheken.

Was in dem Band zu sehen ist, sind vor allem die Fragen, welche in der
Mediävistik an Texte, Bücher und Bibliotheken gestellt werden und die
Methoden, mit denen dabei vorgegangen wird. Teilweise werden lange
Anhänge mitgeliefert, welche die Arbeit an einzelnen Texten
dokumentieren, teilweise wird die Arbeit mit -- immer wieder neuen --
Datenbanken vorgestellt. Es geht erwartungsgemäss viel um
Ordensgeschichte, auch um Wissensordnungen, die aus Texten,
Bibliothekskatalogen und noch erhaltenen Sammlungen rekonstruiert werden
können. Das ist immer wieder interessant, wenn auch teilweise sehr
kleinteilig. Für die Bibliothekswissenschaft und Bibliotheken ergibt
sich aus dem Band aber auch die Erkenntnis, wie weit die Fragestellungen
und Methoden, die in der bibliothekswissenschaftlichen Forschung
dominieren, sich von denen in der Mediävistik unterscheiden. Der
einleitende Beitrag des Bandes (von Andreas Speer) versucht zwar, eine
Brücke zu schlagen, auch, weil die Mediävistik selbstverständlich eine
Wissenschaft ist, bei der viel in Bibliotheken oder mit Dokumenten aus
Bibliotheken gearbeitet wird. Hier wird auch auf Veränderungen in
Universitätsbibliotheken eingegangen. Im restlichen Band findet sich das
aber kaum wieder. Wenn überhaupt, dann wird auf Digitalisate
mittelalterlicher Dokumente als Arbeitsgrundlage verwiesen. Fragen, die
sich in Bibliotheken in Bezug auf diese Dokumente stellen, finden sich
aber nicht. (ks)

\begin{center}\rule{0.5\linewidth}{0.5pt}\end{center}

Kempf, Charlotte Katharina (2020). \emph{Materialität und Präsenz von
Inkunabeln: Die deutschen Erstdrucker im französischsprachigen Raum bis
1550}. (Forum historische Forschung Mittelalter ; 1) Stuttgart: Verlag
W. Kohlhammer, 2020 {[}gedruckt{]}

Die Autorin dieser geschichtswissenschaftlichen Dissertation postuliert,
dass über die Betrachtung einer ausgewählten Gruppe von Frühdruckern --
denen aus den Gebieten des Heiligen Römischen Reiches -- in einer
geografisch abgegrenzten Region -- den französischsprachigen Teilen
Europas, die nicht nur Frankreich, sondern hier auch die Freigrafschaft
Burgund und Städte wie Genf umfasst, welche damals in den Grenzen des
Heiligen Römischen Reiches selber lagen -- mehr über die Transformation
der non-typografischen Gesellschaft, die auf Handschriften basierte, zur
typografischen Gesellschaft, für die Drucke zur Normalität gehörten,
herausgefunden werden könne. Ihr Ansatz sei, nicht nur bekannte und
schon oft untersuchte Frühdrucker darzustellen, sondern die Breite der
unterschiedlichen Wege, welche Frühdrucker und deren Erzeugnisse nahmen,
gesamthaft abzudecken. Was sie in der Arbeit dann zeigt, ist vor allem,
wie prekär die Lebenswege der Frühdrucker und die Medientransformation
selber war. Ein grosser Teil der Frühdrucker, die sie vorstellt, druckte
nur für kurze Zeit und verschwand dann wieder aus den Quellen. Ebenso
begründeten viele von ihnen keine Drucktradition an den einzelnen Orten,
vielmehr waren ihre Druckereien Episoden der jeweiligen Stadt- oder
Klostergeschichte. Ausserdem gab es während der rund 50 Jahre des
Frühdrucks sowohl Beispiele, bei denen Druckwerke als neue Form von
Medien begrüsst und benutzt sowie gleichzeitig Beispiele, bei denen
diese Druckwerke und die damit eintretenden Veränderungen kritisiert
wurden. Zuletzt zeigt sie auch, dass ein erstaunlich grosser Teil der
untersuchten Frühdrucker erst eine Zeitlang in Basel aktiv war und dann
-- zumeist wegen Schulden, was aber auch damit zu tun haben kann, dass
ein Grossteil der Quellen, die wir über sie haben, Gerichtsurteile sind
-- in den französischsprachigen Teil Europas wechselte. Offenbar
fungierte diese Stadt schon im Spätmittelalter als Übergangsort vom
einen in den anderen Sprachraum.

Der Charakter der Arbeit als Dissertation ist offensichtlich: Die
Autorin geht äusserst strukturiert vor, beispielsweise werden die
Frühdrucker und ihre Drucke immer in gleich aufgebauten Kapiteln
beschrieben, und arbeitet sehr genau, was sich in einem ausufernden
Fussnotenapperat niederschlägt. Im methodologischen Teil ihres Vorworts
referiert die Autorin auf den \enquote{material turn} in den
Geschichtswissenschaften, in dem sie auch ihre eigene Arbeit verortet.
Die Materialität der Objekte soll Teil der historischen Untersuchung
sein und Aussagen ermöglichen, die nicht aus den reinen Textquellen zu
generieren sind. In der Arbeit schlägt sich dies vor allem darin nieder,
dass nicht nur Textquellen referiert werden -- dies allerdings recht
traditionell --, sondern auch Eigenschaften wie die Anzahl der Seiten
von Drucken der Frühdrucker dargestellt werden. Was so sichtbar wird,
ist, dass eine Anzahl von Druckern immer wieder ähnliche Drucke
produzierte, anderen aber experimentierten, und dass Drucker ihre
Werkzeuge (Matrizen, Schriftsätze, Druckstöcke) wiederverwendeten,
verkauften und kombinierten.

Der Autor dieser Rezension ist kein Geschichtswissenschaftler, kann also
nicht unbedingt etwas dazu sagen, ob der Anspruch, zum \enquote{material
turn} beizutragen, eingehalten wurde. Aber ihm wurde nicht klar, welche
Neuerung die im Text immer wieder betonte Untersuchung der Materialität
erbracht hat. Vielleicht ist der Frühdruck einfach kein gutes Beispiel,
um den \enquote{material turn} zu exemplifizieren, weil solche
Untersuchungen an den konkreten Drucken schon lange zum Handwerkszeug
der Inkunabelkunde gehört (zumindest der im DACH-Raum, während die
Autorin sich auch auf die französische Forschungstradition der
\enquote{histoire du livre} beruft). (ks)

\begin{center}\rule{0.5\linewidth}{0.5pt}\end{center}

Walworth, Julia C. (2020). \emph{Merton College Library: An Illustrated
History}. Oxford: Bodleian Library, 2020 {[}gedruckt{]}

In Grossbritannien und Irland, aber auch anderswo im
anglo-amerikanischen Raum, ist es durch die neoliberale Politik der
letzten Jahrzehnte normal geworden, dass sich Bibliotheken unter anderem
als Tourismusattraktion vermarkten. Die Leser*innen von Libreas werden
dem Konzept des Bibliothekstourismus selber nicht abgeneigt sein, aber
die Organisation einiger Bibliotheken in den genannten Ländern --
insbesondere wenn sie historische Gebäude und Bestände vorzuweisen haben
-- ist auf einem anderen Level. Hier werden Besuche direkt von den
Bibliotheken selber geplant, eigene Ausstellungen und Rundgänge
organisiert, bestimmte Bilder werden proaktiv verbreitet, Tourist*innen
stehen teilweise in ordentlichen Reihen an, um Eintritt zu zahlen und
dann nicht nur durch die Ausstellung, sondern auch durch den jeweiligen
\enquote{Gift Shop} geschleust zu werden. (Der Rezensent ist auch nicht
darüber erhaben, solche Besuche zu machen. Die Library of Trinity
College, das Book of Kells, die National Library in Dublin und die Linen
Hall Library in Belfast waren selbstverständlich Teil von touristischen
Irland- und Nordirland-Reisen. Ganz so, wie es die jeweiligen
Tourismusbüros geplant haben. Vergleichbares ist im DACH-Raum seltener
zu finden. Selbst solche seltenen Beispiele wie die Stiftsbibliothek
St.~Gallen sind im Vergleich zum angloamerikanischen Raum sehr
zurückhaltend.)

In den Gift Shops dieser Bibliotheken werden neben allen möglichen
anderen Produkten auch Bücher angeboten, die sich mit der entsprechenden
Bibliothek beschäftigen. Das sind keine geschichtswissenschaftlichen
Werke, sondern solche, die genau auf den Verkauf in solchen Shops
ausgelegt sind: schnell und unterhaltsam erzählte Geschichte,
selbstverständlich historisch akkurat, aber doch fokussiert auf
bestimmte Themen (Gründung, Skandale, interessante Geschichten und
Persönlichkeiten, besondere Bestände, hervorhebenswerte Details der
Räume). Zudem immer reich bebildert. Dieses Buch ist eines solcher
Werke. Es erzählt, wie gefordert, sehr kurz und lebendig die Geschichte
der Ende des 13. Jahrhunderts gegründeten Merton College Library, Teil
der Universität Oxford. Dem Genre entsprechend lernt man einiges über
die Gründung und die Bibliotheksentwicklung durch die Jahrhunderte,
wobei der Fokus eher auf dem Raum und interessanten Beständen liegt. Das
Buch ist mit professionellen Bildern des Raumes, Details der heutigen
Regale, zum Teil Bildern aus vergleichbaren Bibliotheken (wenn keine
Originale mehr in Oxford vorhanden sind) und vielen Beispielen seltener
Handschriften, Drucke und anderer Bestände (Globen, astronomische
Instrumente und so weiter) ausgestattet. Das alles lässt sich in einem
Ruck, vielleicht zum Five o'Clock Tea, durchlesen, was kein verlorener
Nachmittag wäre, aber trotzdem bleibt das Buch stark mit einem
touristischen Blick auf die genannte Bibliothek verbunden. (ks)

\hypertarget{social-media}{%
\section{4. Social Media}\label{social-media}}

{[}Diesmal keine Beiträge{]}

\hypertarget{konferenzen-konferenzberichte}{%
\section{5. Konferenzen,
Konferenzberichte}\label{konferenzen-konferenzberichte}}

McGowan, Bethany ; Hart, Jennifer ; Hum, Karen (2021). \emph{Specialized
Regional Conferences Support the Professional Development Needs of
Subject Librarians: A 5-Year Analysis of the Great Lakes Science Boot
Camps for Librarians}. In: College \& Research Libraries 82 (2021) 4,
\url{https://doi.org/10.5860/crl.82.4.548}

In diesem Text werden die Rückmeldungen von Teilnehmer*innen an fünf
jährlichen Durchgängen einer spezialisierten Bibliothekskonferenz
ausgewertet. Die Konferenz ist auf IT und Learning-Tools ausgerichtet
und lokal im Gebiet der \enquote{Great Lakes} (allerdings nur der
US-amerikanischen Seite) verankert. Sie findet jeweils an einer
Universitätsbibliothek in dieser Gegend statt und lädt Forschende und
Personal aus der jeweiligen Universität ein, praxisnahe Veranstaltungen
anzubieten. Grundsätzlich sind die Rückmeldungen positiv.

Aber der Text ist nicht einfach eine Darstellung dieser Konferenz,
sondern in gewisser Weise Werbung dafür, solche kleineren, lokal
orientierten und auf spezielle Themen ausgerichtete Konferenzen zu
organisieren. Die Autor*innen argumentieren, dass für viele kleine,
schlecht mit Mitteln ausgestatteten Bibliotheken sich grössere
Konferenzen kaum als Weiterbildungsveranstaltungen eignen, da es zu
viele Barrieren gibt, die einer Teilnahme der jeweiligen
Bibliothekar*innen entgegenstehen. Viele dieser Barrieren --
beispielsweise Kosten, aber auch Reisezeiten und zu geringe
Spezialisierung -- liessen sich durch kleine Konferenzen senken. (ks)

\begin{center}\rule{0.5\linewidth}{0.5pt}\end{center}

Momeni, Fakhri, Mayr, Philipp, Fraser, Nicholas, \& Peters, Isabella
(2021, September 27): \emph{Impact of Publishing and Citing Open Access
Papers on Researchers' Scientific Success}. Open-Access-Tage 2021.
\url{https://doi.org/10.5281/zenodo.5529858}

In ihrer Präsentation berichten die Autor*innen über ihre Untersuchungen
zu eventuellen Auswirkungen des jeweiligen nationalen Einkommensniveaus
auf das Publikationsverhalten von Wissenschaftler*innen sowie den
Folgeeffekten (\enquote{scientific success of scholars}). Dazu
analysierten sie mittels Scopus zwei Jahrgänge von Artikeln, die in
SpringerNature-Journals erschienen sind. Bei den Folgeeffekten bestätigt
sich die Annahme, dass die Open-Access-Verfügbarkeit von Publikationen
zu mehr Zitierungen führt. Der Schritt zu Open Access ist für die
Publizierenden jedoch mit Kosten verbunden, die auch über
länderspezifische Discounts der Wissenschaftsverlage nicht aufgefangen
werden. Preisnachlässe für Publikationsgebühren stellen daher nur einen
begrenzten Anreiz dar. Allerdings weisen die Länder mit den niedrigsten
Einkommen die höchsten Open-Access-Publikationsraten auf. Weitere
ermittelte Auffälligkeiten waren, dass Autor*innen eher zu
Full-Open-Access-Publikationen neigen, dass interdisziplinäre Aufsätze
eher Open Access publiziert werden und dass die Naturwissenschaften den
höchsten, Sozialwissenschaften dagegen den niedrigsten
Open-Access-Anteil aufweisen. (bk)

\hypertarget{populuxe4re-medien-zeitungen-radio-tv-etc.}{%
\section{6. Populäre Medien (Zeitungen, Radio, TV
etc.)}\label{populuxe4re-medien-zeitungen-radio-tv-etc.}}

Russew, Georg-Stefan (2021): \emph{Tempelhof-Schöneberg
Bezirksbibliothek wird Ziel mutmaßlich rechtsgerichteter Attacke}. In:
rbb24.de. 12.08.21
\url{https://www.rbb24.de/panorama/beitrag/2021/08/berlin-tempelhof-zerstoerte-buecher-bibliothek-mutmasslich-rechte-attacke.html}

Inforadio (2021): \emph{Tempelhofer Bibliothek wird erneut Ziel von
mutmaßlichen rechtem Angriff.} In: rbb24.de 16.09.21
\url{https://www.rbb24.de/panorama/beitrag/2021/09/bibliothek-tempelhof-schoeneberg-buecher-zersoert.html}

Immer wieder werden Bibliotheken Ziele von rechten Attacken. Bekannt
sind Fälle, in denen Flyer mit rechtem Gedankengut in Büchern von
Bibliotheken hinterlegt werden. Jüngst wurde die Bezirksbibliothek
Tempelhof-Schöneberg (Berlin) mehrmals Ziel von Zerstörung. Eine oder
mehrere unbekannte Personen zerschnitten Bücher, die sich kritisch mit
rechtem Gedankengut auseinandersetzen. Der Leiter der Bibliothek,
Boryano Rickum, veröffentlichte die Titel und Autor*innennamen der
betroffenen Bücher. Außerdem sagte er dem rbb (Rundfunk
Berlin-Brandenburg), dass die \enquote{neutrale} Position der Bibliotheken
nicht mehr möglich wäre und ein klares Zeichen für die Demokratie
notwendig sei. Am 16.09.21 berichtete rbb24.de von erneuten
zerschnittenen Büchern in der genannten Bibliothek. Der Leiter zog die
Konsequenz daraus und stellt einen Wachschutz ein. Die zerstörten Bücher
werden als Mahnmal in einer Vitrine im Foyer präsentiert, daneben stehen
dieselben (vollständigen) Bücher zur Ausleihe bereit. (sj)

\begin{center}\rule{0.5\linewidth}{0.5pt}\end{center}

O'Connor, William (2021). \emph{The Strange Tale of the Beautiful
Library and the Town That Never Asked for It}. In: The Daily Beast, 19.
Juni 2021,
\url{https://www.thedailybeast.com/the-strange-tale-of-the-beautiful-library-and-the-town-that-never-asked-for-it?ref=home}

Im Artikel wird die Geschichte erzählt, wie die Kleinstadt Winchester in
Virginia zu ihrer Öffentlichen Bibliothek gekommen ist. Die --
überdimensioniert, in einem für Bibliotheken grösserer Städte, im frühen
20. Jahrhundert typischen palastähnlichen Bau -- die dortige Altstadt
prägt. Kurz zusammengefasst: wegen dem Erbe eines
Südstaaten-begeisterten Richters und Unternehmers aus Pennsylvania
(offensichtlich nicht in den Südstaaten der USA). Dieser überliess sein
Vermögen lieber einer Stadt im Süden als der Stadt, in der er immerhin
tätig war oder entfernten Verwandten. Der Bau der Bibliothek dauerte
dann seine Zeit und als sie eröffnet wurden, war sie ab 1913 für die
nächsten 50 Jahre nur für Weisse zugänglich.

Der Artikel ist Teil einer Serie \enquote{The world's most beautiful
libraries}
(\url{https://www.thedailybeast.com/keyword/the-worlds-most-beautiful-libraries}),
welche (zumeist einmal monatlich) in der Reiserubrik der Zeitschrift The
Daily Beast erscheint. Diese Artikel sind in der Regel kurzweilig, aber
leider -- ein wenig im Konflikt mit dem Titel -- immer nur mit der
jeweiligen Titelgrafik und nicht noch weiteren Fotos bebildert. (ks)

\begin{center}\rule{0.5\linewidth}{0.5pt}\end{center}

Schimek, Cornelia: \emph{Acht Stunden Arbeit, acht Gläser Sekt. Oder
Wein. Oder Schnaps.} In: KATAPULT Magazin für Kartografik und
Sozialwissenschaft. No.~22 Juli-Sept 2021, S. 78--83. {[}gedruckt{]}

In ihrer umfangreichen Aufarbeitung des Themas \enquote{Alkohol am
Arbeitsplatz} arbeitet die Autorin berufsgruppenspezifische
Neigungsverteilungen zum \enquote{riskanten Konsum} von Alkohol heraus.
Dabei wird erwähnt, dass Bibliotheksberufe empirisch ein vergleichsweise
geringes Risiko mit sich bringen. Dieser Fakt wird in der
Zwischenüberschrift \enquote{Bibliothekare weniger betroffen}
ausdrücklich hervorgehoben. Sie bezieht sich dabei auf eine Tabelle im
Dateianhang zu Andrew Thompson, Munir Pirmohamed: \emph{Associations
between occupation and heavy alcohol consumption in UK adults aged
40--69 years: a cross-sectional study using the UK Biobank}. BMC Public
Health 21, 190 (2021). \url{https://doi.org/10.1186/s12889-021-10208-x}.
Die Autoren der Studie selbst heben allerdings andere Berufsgruppen als
Niedrigrisikogruppen hervor: \enquote{Clergy {[}\ldots{]} physicists,
geologists and meteorologists {[}\ldots{]}; and medical practitioners
{[}\ldots{]} were least likely to be heavy drinkers}. (bk)

\begin{center}\rule{0.5\linewidth}{0.5pt}\end{center}

Owen, David: \emph{RECORD COLLECTION. Labor of Love Dept.} In: The NEW
YORKER, Aug.~23., 2021, S. 45.

Der in Bridgeport, Connecticut, USA, beheimatete Community-Radiosender
WPKN-FM entschloss sich bei seiner Gründung im Jahr 1963 für eine
Archivierung der eingehenden Schallplatten, nicht nach beispielsweise
Genre oder Alphabet, sondern nach laufender Nummer, beginnend mit einer
Pressung von Judy Garlands \enquote{A Star is Born}. Daraus resultiert
ein, wie David Owen schreibt, \enquote{quirkily dendrochronological
register of old and new music during the past six decades or so}. (bk)

\begin{center}\rule{0.5\linewidth}{0.5pt}\end{center}

Sema Çağlak: \emph{Call for children from \enquote*{Colourful Hopes}.}
In: JINNEWS, 12.10.2021.
\url{http://jinnews39.xyz/en/ALL-NEWS/content/view/173802}

Der Beitrag stellt die spendenbasierte \enquote{Spielzeug-Bibliothek}
(Toy Library) der Colourful Hopes Association im türkischen Diyarbakır
und die damit zusammenhängenden zivilgesellschaftlichen Aktivitäten vor.
Sie ist nach dem kurdischen Menschenrechtsanwalt Tahir Elçi benannt, der
2015 ermordet wurde. Der Hintergrund ist, dass viele Kinder in der
Region keinen Zugang zu Spielplätzen und aufgrund von Armut keine
eigenen Spielzeuge haben. Diese Kinder können sich jeweils ein Spielzeug
für eine Woche ausleihen. Das Angebot umfasst zusätzlich Bücher und
mittlerweile auch Fahrräder. Neben dem Effekt eines Zugangs zu diesen
Dingen erhoffen sich die Betreibenden auch Effekte bei der Herausbildung
eines Verantwortlichkeitsgefühls und zwar nicht nur bei den Kindern,
sondern auch auf gesellschaftlicher Ebene. Der Zugang und die
Verfügbarkeit von Spielzeug für Kinder wird als Grundrecht angesehen,
dessen Absicherung die Aufgabe der gesamten Gesellschaft ist. (bk)

\begin{center}\rule{0.5\linewidth}{0.5pt}\end{center}

\begin{CJK}{UTF8}{ipxm}
Nagata, Kazuaki: \emph{Library named after alum Haruki Murakami at
Tokyo's Waseda University}. In: The Japan Times. 23.09.2021.
\url{https://www.japantimes.co.jp/news/2021/09/23/national/murakami-library-waseda/}
\
An der Waseda University (早稲田大学) in Shinjuku, Tokyo, wurde zum
01.10.2021 eine Bibliothek zu Ehren von Haruki Murakami und damit auch
mit der Bezeichnung Haruki Murakami Library in Ergänzung zum Namen
Waseda International House of Literature eröffnet. Zur Inspiration wird
die Bibliothek auch eine Nachbildung des Arbeitszimmers von Haruki
Murakami enthalten. Gestiftet wurde sie vom Mode-Milliardär Tadashi
Yanai (unter anderem Uniqlo Co., Ltd.). (bk)

\end{CJK}

\begin{center}\rule{0.5\linewidth}{0.5pt}\end{center}

Sommer, Will (2021). \emph{Inside the Right's Plan to Rebrand Sex Ed as
\enquote*{Child Porn}: Their new tactics include harassing school boards
and calling the cops on librarians.} In: The Daily Beast, 05.10.2021,
\url{https://www.thedailybeast.com/inside-the-rights-plan-to-rebrand-sex-ed-as-child-porn}

Es scheint, als wäre in den USA nach Jahren der rechtsextremen
Mobilisierung vor und während der Präsidentschafts Trumps, den
paranoiden Vorstellungen im Bezug auf dessen Ende sowie die
COVID-19-Pandemie ein wenig die Luft raus und es würde sich von
rechtsextremen Akteur*innen wieder bekannteren Themen zugewandt. Auf der
einen Seite konzentrieren sich diese Netzwerke aktuell erneut darauf,
die reproduktiven Rechte von Frauen einzuschränken. Aber offenbar ist
auch das Verbieten von Büchern in Bibliotheksbeständen und Schulen in
diesen Kreisen wieder en vouge geworden.

Der Journalist Will Sommer berichtet viel über die Entwicklungen der
rechten und rechtsextremen Netzwerke in den USA. In diesem Artikel
thematisiert er den letztgenannten Umstand. Während das \enquote{book
challenging} schon fast zu einer klassischen Taktik dieser rechten
Netzwerke gehört -- inklusive Verbreitung von Informationen, wie dabei
vorzugehen ist oder welche Bücher aktuell als verbotswürdig gelten --
und Öffentliche Bibliotheken in den USA allesamt Strukturen entwickelt
haben, um auf diese zu reagieren -- beispielsweise klare Abläufe für die
Eingabe und Bearbeitung solcher Beschwerden, eine starke Lobbyarbeit des
Bibliotheksverbandes oder auch transparenter Informationen über die
Beschwerden --, gibt es aktuell offenbar wieder eine Zunahme dieser
Fälle und gleichzeitig eine neue Taktik. Vertreter*innen in School und
Library Boards, aber auch einzelnen Bibliothekar*innen und Lehrpersonen,
wird gedroht, sie wegen der angeblichen Verbreitung von
\enquote{Pornographie} anzuzeigen. Die Bücher, um die es geht, sind
erwartungsgemäss solche zur Sexualaufklärung und solche mit einem Fokus
auf die Darstellung sexueller Diversität.

Sommer schildert im Text, wie Bibliotheken und der Bibliotheksverband
auf diese Herausforderung reagieren, aber auch, dass die
Zivilgesellschaft oft eine klar die Schulen und Bibliotheken
unterstützende Position einnimmt. Zudem vermittelt der Artikel den
Eindruck, dass die rechten Netzwerke sich nur eine Zeit mit dieser
Kampagne aufhalten werden, bis sie ein neues Thema gefunden haben, das
sie in ihrem paranoiden Weltbild anzugehen glauben müssen. (ks)

\hypertarget{abschlussarbeiten}{%
\section{7. Abschlussarbeiten}\label{abschlussarbeiten}}

{[}Diesmal keine Beiträge{]}

\hypertarget{weitere-medien}{%
\section{8. Weitere Medien}\label{weitere-medien}}

Fachstelle für Öffentliche Bibliotheken NRW (2021).
\emph{Abschlussbericht Pilotprojekt \enquote{Sprachschatz -- Bibliothek
und KiTa Hand in Hand} 2017--2019}. Düsseldorf: Bezirksregierung
Düsseldorf, 2021,
\url{https://fachstelle-oeffentliche-bibliotheken.nrw/wp-content/uploads/2021/06/Sprachschatz_Abschlussbericht_21_06_08_Webversion_FINAL.pdf}

Das Projekt \enquote{Sprachschatz} liess in sechs Gemeinden, die sich
dafür beworben hatten, im ländlichen Raum Nordrhein-Westfalens
Öffentliche Bibliotheken, Kindertagesstätten und zum Teil auch andere
Einrichtungen vor Ort kooperieren, um Digitale Medien für die Förderung
von Sprache und Vielfältigkeit einzusetzen. Diese Kooperationen wurden
über zwei Jahre strukturiert, mit Workshops und Zeitplänen unterstützt
sowie evaluiert. Der Projektbericht umfasst alles: eine Beschreibung des
Projektes und der sechs konkreten Kooperationen, die Evaluation (die
allerdings in der Analyse sehr zurückhaltend ist und vor allem
Ergebnisse von Befragungen und Auswertungen von Sitzungsprotokollen
berichtet) sowie im Projekt erstellte Materialien, die bei der
Zusammenarbeit anderer Bibliotheken und Kindertagesstätten verwendet
werden können (und den grössten Teil der Publikation einnehmen).

Interessant ist, über das Thema Sprachschatz und Vielfältigkeit hinaus,
dass im Bericht dargestellt wird, was grundsätzlich für solche
erfolgreichen Kooperationen notwendig war: Die beteiligten Einrichtungen
mussten sich regelmässig treffen, gemeinsam Ziele definieren und erst
lernen, was die jeweils andere Einrichtung eigentlich konkret macht
sowie gemeinsam klare Aufgaben definieren. Es war notwendig, gemeinsam
überhaupt die jeweils vorhandenen Stärken der anderen kennen zu lernen.
Hauptproblem aller Teams war die notwendige Zeit und Konsistenz im
Arbeitsalltag: Erstaunlich oft war es nicht möglich, dass über zwei
Jahre immer die gleichen Personen in den Einrichtungen Verantwortungen
übernahmen und ebenso waren regelmässige Treffen nicht immer einfach
durchzuführen. Kooperation, so wird in diesem Teil des Berichtes
deutlich, ist auch im ländlichen Raum, wo sich angeblich alle kennen,
eine Aufgabe, die aktiv geplant und von allen Beteiligten getragen
werden muss, um erfolgreich zu sein.

Auffällig war zudem (was in der Evaluation am Rande angesprochen wird,
aber eigentlich mehr betont werden sollte), dass auch in diesem Projekt
sehr schnell bei den Kooperationen gemeinsam auf einfacher umzusetzende
Themen fokussiert wurde (eher Ausprobieren von Medien und Techniken) und
komplexere Themen (Vielfältigkeit) mehr oder minder ignoriert wurden. Im
Alltag setzt sich schnell das durch, was einfach an die schon normale
bibliothekarische Arbeit (und wohl auch die Arbeit in
Kindertagesstätten) anzuschliessen war. (ks)

\begin{center}\rule{0.5\linewidth}{0.5pt}\end{center}

Bayerische Staatsbibliothek (2021). \emph{Mönch, Rebell, Bibliothekar.
Eine virtuelle Ausstellung zum 250. Geburtstag von Martin Schrettinger}.
\url{https://www.bsb-muenchen.de/va/ausstellungen/moench-rebell-bibliothekar/}

{[}via aubib-Blog:
\url{https://www.aubib.de/blog/article/2021/10/05/moench-rebell-bibliothekar-eine-virtuelle-ausstellung-zum-250-geburtstag-von-martin-schrettinger/}{]}

Martin Schrettinger gilt gemeinhin als Begründer der
Bibliothekswissenschaft, auch wenn sein Fokus die Organisation von
Bibliotheken und ihrer Bestände war -- ein Thema, dass heute die
Bibliothekswissenschaft nicht mehr prägt. Unbestritten ist jedoch, dass
er es war, welcher vor über zweihundert Jahren den Begriff
\enquote*{Bibliothekswissenschaft} prägte und einer der Personen war,
welche die Grundlagen für eine systematische Beschäftigung mit
Bibliotheken legte.

Schrettinger war bekanntlich in der heutigen Bayerischen
Staatsbibliothek in München tätig. Zu seinem 250. Geburtstag
veranstaltet diese Bibliothek jetzt eine virtuelle Ausstellung über ihn.
In dieser ist zu lernen, wie der von der Aufklärung geprägte ehemalige
Mönch mehrfach Versuche unternahm, die Büchermassen, welche sich durch
die Auflösung der Klöster nach der Säkularisierung und der
Beschlagnahmung derer Wertgegenstände durch den bayerischen Staat in
München sammelten, zu ordnen und wie er die Überlegungen dazu in ein
System brachte, welches von ihm den Namen
\enquote*{Bibliothekswissenschaft} erhielt. Man erfährt aber auch von
seinen Zusammenbrüchen und erhält einen reich bebilderten Kontext.
Selbstverständlich ist den Schriften Schrettingers, inklusive dem
\enquote{Handbuch der Bibliothekswissenschaft} (1834), ein eigener Teil
gewidmet. (ks)

\begin{center}\rule{0.5\linewidth}{0.5pt}\end{center}

CC Deutschland: \emph{FAQ}, \url{https://de.creativecommons.net/faqs/}

Creative Commons Deutschland stellt seit Sommer 2021 ein sehr
umfangreiches FAQ auf Deutsch bereit. Die circa 130 Fragen und Antworten
sind den folgenden fünf Hauptkategorien zugeordnet: \enquote{1. Über
Creative Commons}, \enquote{2. Allgemeine Informationen über die
CC-Lizenzen}, \enquote{3. Inhalte unter eine CC-Lizenz stellen},
\enquote{4. CC-lizenzierte Inhalte nutzen} und \enquote{5. Datenbanken,
Daten und KI}. Die Zusammenstellung bietet Wissenswertes für
Einsteiger*innen und Fortgeschrittene. Die deutschen FAQ bauen auf der
englischen Version auf
(\url{https://creativecommons.org/faq/});
ergänzend werden Informationen zu Besonderheiten geliefert, die sich aus
dem deutschen beziehungsweise europäischen Recht ergeben. (mv)

%autor

\end{document}
