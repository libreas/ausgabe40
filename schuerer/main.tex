\documentclass[a4paper,
fontsize=11pt,
%headings=small,
oneside,
numbers=noperiodatend,
parskip=half-,
bibliography=totoc,
final
]{scrartcl}

\usepackage[babel]{csquotes}
\usepackage{synttree}
\usepackage{graphicx}
\setkeys{Gin}{width=.4\textwidth} %default pics size

\graphicspath{{./plots/}}
\usepackage[ngerman]{babel}
\usepackage[T1]{fontenc}
%\usepackage{amsmath}
\usepackage[utf8x]{inputenc}
\usepackage [hyphens]{url}
\usepackage{booktabs} 
\usepackage[left=2.4cm,right=2.4cm,top=2.3cm,bottom=2cm,includeheadfoot]{geometry}
\usepackage{eurosym}
\usepackage{multirow}
\usepackage[ngerman]{varioref}
\setcapindent{1em}
\renewcommand{\labelitemi}{--}
\usepackage{paralist}
\usepackage{pdfpages}
\usepackage{lscape}
\usepackage{float}
\usepackage{acronym}
\usepackage{eurosym}
\usepackage{longtable,lscape}
\usepackage{mathpazo}
\usepackage[normalem]{ulem} %emphasize weiterhin kursiv
\usepackage[flushmargin,ragged]{footmisc} % left align footnote
\usepackage{ccicons} 
\setcapindent{0pt} % no indentation in captions

%%%% fancy LIBREAS URL color 
\usepackage{xcolor}
\definecolor{libreas}{RGB}{112,0,0}

\usepackage{listings}

\urlstyle{same}  % don't use monospace font for urls

\usepackage[fleqn]{amsmath}

%adjust fontsize for part

\usepackage{sectsty}
\partfont{\large}

%Das BibTeX-Zeichen mit \BibTeX setzen:
\def\symbol#1{\char #1\relax}
\def\bsl{{\tt\symbol{'134}}}
\def\BibTeX{{\rm B\kern-.05em{\sc i\kern-.025em b}\kern-.08em
    T\kern-.1667em\lower.7ex\hbox{E}\kern-.125emX}}

\usepackage{fancyhdr}
\fancyhf{}
\pagestyle{fancyplain}
\fancyhead[R]{\thepage}

% make sure bookmarks are created eventough sections are not numbered!
% uncommend if sections are numbered (bookmarks created by default)
\makeatletter
\renewcommand\@seccntformat[1]{}
\makeatother

% typo setup
\clubpenalty = 10000
\widowpenalty = 10000
\displaywidowpenalty = 10000

\usepackage{hyperxmp}
\usepackage[colorlinks, linkcolor=black,citecolor=black, urlcolor=libreas,
breaklinks= true,bookmarks=true,bookmarksopen=true]{hyperref}
\usepackage{breakurl}

%meta
%meta

\fancyhead[L]{Y. Schürer\\ %author
LIBREAS. Library Ideas, 40 (2021). % journal, issue, volume.
\href{https://doi.org/10.18452/23809.2}{\color{black}https://doi.org/10.18452/23809.2}
{}} % doi 
\fancyhead[R]{\thepage} %page number
\fancyfoot[L] {\ccLogo \ccAttribution\ \href{https://creativecommons.org/licenses/by/4.0/}{\color{black}Creative Commons BY 4.0}}  %licence
\fancyfoot[R] {ISSN: 1860-7950}

\title{\LARGE{Warum beschäftige ich mich mit dem Dekolonialisieren von Bibliotheken?}}% title
\author{Yvonne Schürer} % author

\setcounter{page}{1}

\hypersetup{%
      pdftitle={Warum beschäftige ich mich mit dem Dekolonialisieren von Bibliotheken?},
      pdfauthor={Yvonne Schürer},
      pdfcopyright={CC BY 4.0 International},
      pdfsubject={LIBREAS. Library Ideas, 40 (2021)},
      pdfkeywords={Bibliothek, Dekolonisierung, Dekolonialisierung, library, decolonization},
      pdflicenseurl={https://creativecommons.org/licenses/by/4.0/},
      pdfcontacturl={http://libreas.eu},
      baseurl={https://doi.org/10.18452/23809.2},
      pdflang={de},
      pdfmetalang={de}
     }



\date{}
\begin{document}

\maketitle
\thispagestyle{fancyplain} 

%abstracts

%body
\begin{quote}
\enquote{Letztlich ist die Dekolonialisierung von Geist und Wissen der erste
Schritt zur Dekolonialisierung von allem anderen. Nicht nach den Regeln
des hegemonialen Systems zu denken oder die bestehenden Mittel des
kolonialistischen Denkens zu kritisieren, scheint also die grundlegende
Phase zu sein, um wirklich Gleichheit zu erreichen.}\footnote{Übersetzt
  aus dem Blogbeitrag von Nihan Albayrak (2018): Diversity helps but
  decolonisation is the key to equality in higher~education. In: Eden
  Centre for Education Enhancement. Contemporary Issues in Teaching and
  Learning. Siehe dazu
  \url{https://lsepgcertcitl.wordpress.com/2018/04/16/diversity-helps-but-decolonisation-is-the-key-to-equality-in-higher-education}
  (letzter Zugriff aller Links am 01.12.2021).}
\end{quote}
\begin{flushright}Nihan Albayrak-Aydemir\end{flushright}

\begin{center}\rule{0.5\linewidth}{0.5pt}\end{center}

Spontan ist meine Antwort auf diese Frage: Aus mehreren Gründen. Es gibt
kein \enquote{Aha-Erlebnis}, das mich mit dem Thema in Verbindung
gebracht hat. Aber es gab eine inspirierende, sensibilisierende
Veranstaltung. In der Rückschau weiß ich heute, dass die ARLIS-Konferenz
2019\footnote{Art Library Society UK \& Ireland -- siehe dazu
  \url{https://arlis.net}.} und insbesondere die Gespräche mit Tavian
Hunter, der Leiterin der Bibliothek des Londoner \emph{Institute of
International Visual Arts} und Beck Wonders, eine Doktorandin an der
\emph{Glasgow School of Art} und Mitbegründerin der \emph{Vancouver
Womens Library}, im Anschluss an diese Konferenz, zu einer Art
Erkenntnis führten. Dieser Austausch veränderte meine Wahrnehmung.

Da ich in einer ländlichen Region im Osten Deutschlands aufwuchs, blieb
die Konfrontation mit rassistischen Personen und rassistischem Verhalten
schon in jungen Jahren leider nicht aus. Das Studium in Leipzig
verbesserte daran wenig, wenngleich die Stadt als Sonder- oder
Glücksfall in einer Region mit enorm hoher AfD- und NPD-Wähler-Dichte
gelten könnte. Doch auch hier berichten BIPoC\footnote{Abkürzung für
  \enquote{Black, Indigenous and People of Color}.}-Freunde von
Alltagsrassismus und viele verlassen aufgrund einschneidender Erlebnisse
und dem Wunsch nach mehr Diversität im Alltag die Region.

Vor dem Horizont dieser Erfahrungen erkannte ich nach und nach meine
Privilegien als weiße Person, die ich zuvor nie realisiert hatte: Ich
werde nicht in der Straßenbahn angestarrt, nicht gefragt, warum ich so
gut Deutsch spreche und woher meine Eltern kommen, ob ich gut tanzen
oder Basketball spielen könne, meine Haare werden nicht ungefragt
angefasst, \ldots{}

Zugleich muss ich erkennen, dass auch ich mit Stereotypen und
Vorurteilen aufwuchs, diese verinnerlichte und sie daher weniger leicht
abschütteln kann, als ich es mir wünsche. Der Prozess der Erkenntnis,
wie Weltbilder meiner Kindheit (geprägt durch den \enquote{Orient}
romantisierende Märchen, Rassismen reproduzierende Kinderbücher und
-lieder und dem Wiederaufleben völkischer Ideologie in der mich
umgebenden Gesellschaft nach dem Mauerfall\footnote{Siehe dazu unter anderem
  Peter Nowak (2015):
  \url{https://www.heise.de/tp/news/Warum-gibt-es-prozentual-mehr-Rassismus-in-Ostdeutschland-2784260.html}.})
zwangsläufig meine heutigen Gedanken beeinflussen, bleibt
unabgeschlossen. Wie gehe ich damit um? Ich versuche immerhin die
Tatsache, mir dessen bewusst zu sein und immer wieder mein Denken und
Handeln zu hinterfragen, als kleinen Erfolg zu verbuchen.
Zufriedenstellend ist es nicht.

Mit kleinen Schritten versuche ich, aktiv dagegen anzugehen. Zum
Beispiel überprüfe ich meine Sprache und mein Konsumverhalten. Und ich
bemerke -- trotz der mir zur Verfügung stehenden Möglichkeiten -- das
Gefühl der Machtlosigkeit angesichts eines omnipräsenten,
imperialistischen, kapitalistischen Systems, das von Beginn an auf der
Unterdrückung Anderer für Privilegien aufgebaut ist, die auch meine
sind.

Als Bibliothekarin bin ich grundsätzlich neugierig. Zugleich sehe ich
eine gewisse Verantwortung in meinem Beruf -- eine Überzeugung, die ich
mit Antonia Paula Herm und vermutlich vielen weiteren im Bibliotheks-
und informationsvermittelnden Bereich Tätigen teile -- und zwar, dass
\enquote{Akteur*innen im Bildungs- und Wissenschaftsbereich eine
besondere Verantwortung haben, der historischen Verantwortung
Deutschlands {[}...{]} gerecht zu werden.}\footnote{Antonia Paula Herm:
  Koloniale Spuren in bibliothekarischen Sammlungen und
  Wissensordnungen, Masterarbeit am Institut für Bibliotheks- und
  Informationswissenschaft der Humboldt-Universität zu Berlin (2019), S.
  48. (unveröffentlicht). Siehe auch Beitrag von Paula Herm in dieser
  Ausgabe: \url{https://doi.org/10.18452/23801}.}

Die eingangs genannte Konferenz habe ich eher zufällig besucht. Zum
50-jährigen Jubiläum von ARLIS reiste ich als Vertretung der deutschen
\enquote{Schwesterorganisation}, der \emph{Arbeitsgemeinschaft für
Kunst- und Museumsbibliotheken (AKMB)} nach Glasgow, mit dem Auftrag dem
deutschsprachigen Kollegium im Anschluss davon zu berichten. Tatsächlich
wurde ich erst bei der Ankündigung der Kongressthemen hellhörig -- ein
ganzer Themenkomplex wurde \enquote{critical librarianship and
decolonising the curriculum} gewidmet.\footnote{Das detaillierte
  Programm inklusive Abstracts der Beiträge gibt es hier:
  \url{https://duncanchappell.wixsite.com/arlis2019}; wer Interesse am
  Konferenzbericht \enquote{A Pioneering Past. A Vibrant Future} hat,
  kann diesen im Heft 1/2 der AKMB-News (Jg. 26; 2020) ab S. 87
  nachlesen: \url{https://doi.org/10.11588/akmb.2020.1/2.76456)}.}
\enquote{Decolonising} -- ein Begriff der mir bis dahin kaum begegnet
war. Ich hatte zuvor schon von Postkolonialismus gehört und bemerkt,
dass die Verbrechen der Kolonialgeschichte vieler europäischer Länder
mittlerweile nicht mehr ganz so verborgen unter den Teppichen bleiben,
unter die man sie lange zu kehren versuchte.

\hypertarget{was-aber-hatte-so-ein-thema-auf-einer-bibliothekskonferenz-zu-suchen}{%
\subsubsection{Was aber hatte so ein Thema auf einer
Bibliothekskonferenz zu
suchen?}\label{was-aber-hatte-so-ein-thema-auf-einer-bibliothekskonferenz-zu-suchen}}

Die Keynote von David Dibosa war mit \enquote{Re-Worlding our Knowledge}
überschrieben und brachte die Teilnehmenden in anregende, teils
emotionale Diskussionen zu den Themen \enquote{Rassismus in unserer
Sprache}, \enquote{Aktivismus in der Bibliothekspraxis},
\enquote{Problematische Sammlungsbestände identifizieren und darüber
kommunizieren} und die, wie sich zeigte, nur vermeintliche Neutralität
und Unabhängigkeit unserer Arbeits- und Sichtweisen.

Mehrfach wurde während des Vortrags deutlich, dass in fast all unseren
bibliothekarischen Angeboten vorrangig die westliche bzw. eurozentrische
Sicht\footnote{Präziser wäre es vermutlich, statt der
  \enquote{eurozentrischen} oder \enquote{westlichen} Sicht auf den
  globalen Norden zu verweisen, wobei auch um dessen Definition derzeit
  noch gerungen wird. Eine Visualisierung, welche Länder zum globalen
  Norden und welche zum globalen Süden gezählt werden, findet sich hier
  bei SterpMap (2013):
  \url{https://www.stepmap.de/karte/globaler-sueden-globaler-norden-V7WKiFRkie}.}
vermittelt wird, während Perspektiven der Marginalisierten oft nicht
vorkommen oder gar ignoriert werden.

Im Laufe der Gespräche erkannte auch ich einige Dinge, die ich in meinem
Arbeitsumfeld unterbewusst wahrnahm, aber nie benennen konnte. Dazu
zählen ebenso die Unausgewogenheit von Sammlungen (zum Beispiel in Bezug
auf Religion, Geschlecht, Hautfarbe oder sexuelle Orientierung), wie
auch rassistische Bezeichnungen -- nicht nur in bibliothekarischen
Nachweisinstrumenten, sondern auch in unserer Alltagssprache und die
Auswahl der Informations- und Bezugsquellen auch \enquote{meiner}
eigenen Bibliothek, der Bibliothek der Hochschule für Grafik und
Buchkunst Leipzig.

Diese Erkenntnis und auch der motivierende Aufruf David Dibosas, aktiv
diesen Prozessen entgegenzuwirken, motivierte mich, zu einer
intensiveren Beschäftigung mit der \enquote{Dekolonialisierung}. Im
Rahmen der mir zur Verfügung stehenden Möglichkeiten beginne ich der
Unausgewogenheit durch gezielte Erwerbung nicht-weiß, nicht-europäisch,
nicht-männlich, nicht-heteronormativ gelesenen Publizierenden und
Kunstschaffenden entgegenzuwirken und das Angebot und die Bezugsketten
meiner Lieferfirmen zu hinterfragen. Bibliotheksangebote möglichst in
mehreren Sprachen (mindestens aber in Englisch) zu kommunizieren und
unsere hauseigene Klassifikation auf diskriminierende Bezeichnungen und
Zuordnungen zu prüfen. Weitere Schritte sind mich weiterzubilden und
meine Erkenntnisse beziehungsweise meinen Weg zu diesen zu
teilen.

Zum Beispiel in einem LIBREAS-Artikel.

%autor
\begin{center}\rule{0.5\linewidth}{0.5pt}\end{center}

\textbf{Yvonne Schürer} war nach dem Diplomstudium an der HTWK Leipzig
für die Staatliche Bücher- und Kupferstichsammlung Greiz tätig. Seit
2014 leitet sie die Bibliothek der Hochschule für Grafik und Buchkunst
in Leipzig. Derzeit studiert sie berufsbegleitend das Masterstudium am
Institut für Bibliotheks- und Informationswissenschaft der
Humboldt-Universität zu Berlin. Die Abschlussarbeit, die sie im Oktober
2021 eingereicht hat, trägt den Titel \enquote{Was bedeutet Dekolonialisieren
für Bibliotheken?}. \href{mailto:schuerer@hgb-leipzig.de}{\nolinkurl{schuerer@hgb-leipzig.de}}

\end{document}
