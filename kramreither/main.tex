\documentclass[a4paper,
fontsize=11pt,
%headings=small,
oneside,
numbers=noperiodatend,
parskip=half-,
bibliography=totoc,
final
]{scrartcl}

\usepackage[babel]{csquotes}
\usepackage{synttree}
\usepackage{graphicx}
\setkeys{Gin}{width=.4\textwidth} %default pics size

\graphicspath{{./plots/}}
\usepackage[ngerman]{babel}
\usepackage[T1]{fontenc}
%\usepackage{amsmath}
\usepackage[utf8x]{inputenc}
\usepackage [hyphens]{url}
\usepackage{booktabs} 
\usepackage[left=2.4cm,right=2.4cm,top=2.3cm,bottom=2cm,includeheadfoot]{geometry}
\usepackage{eurosym}
\usepackage{multirow}
\usepackage[ngerman]{varioref}
\setcapindent{1em}
\renewcommand{\labelitemi}{--}
\usepackage{paralist}
\usepackage{pdfpages}
\usepackage{lscape}
\usepackage{float}
\usepackage{acronym}
\usepackage{eurosym}
\usepackage{longtable,lscape}
\usepackage{mathpazo}
\usepackage[normalem]{ulem} %emphasize weiterhin kursiv
\usepackage[flushmargin,ragged]{footmisc} % left align footnote
\usepackage{ccicons} 
\setcapindent{0pt} % no indentation in captions

%%%% fancy LIBREAS URL color 
\usepackage{xcolor}
\definecolor{libreas}{RGB}{112,0,0}

\usepackage{listings}

\urlstyle{same}  % don't use monospace font for urls

\usepackage[fleqn]{amsmath}

%adjust fontsize for part

\usepackage{sectsty}
\partfont{\large}

%Das BibTeX-Zeichen mit \BibTeX setzen:
\def\symbol#1{\char #1\relax}
\def\bsl{{\tt\symbol{'134}}}
\def\BibTeX{{\rm B\kern-.05em{\sc i\kern-.025em b}\kern-.08em
    T\kern-.1667em\lower.7ex\hbox{E}\kern-.125emX}}

\usepackage{fancyhdr}
\fancyhf{}
\pagestyle{fancyplain}
\fancyhead[R]{\thepage}

% make sure bookmarks are created eventough sections are not numbered!
% uncommend if sections are numbered (bookmarks created by default)
\makeatletter
\renewcommand\@seccntformat[1]{}
\makeatother

% typo setup
\clubpenalty = 10000
\widowpenalty = 10000
\displaywidowpenalty = 10000

\usepackage{hyperxmp}
\usepackage[colorlinks, linkcolor=black,citecolor=black, urlcolor=libreas,
breaklinks= true,bookmarks=true,bookmarksopen=true]{hyperref}
\usepackage{breakurl}

%meta
%meta

\fancyhead[L]{B. Kramreither\\ %author
LIBREAS. Library Ideas, 40 (2021). % journal, issue, volume.
\href{https://doi.org/10.18452/23810}{\color{black}https://doi.org/10.18452/23810}
{}} % doi 
\fancyhead[R]{\thepage} %page number
\fancyfoot[L] {\ccLogo \ccAttribution\ \href{https://creativecommons.org/licenses/by/4.0/}{\color{black}Creative Commons BY 4.0}}  %licence
\fancyfoot[R] {ISSN: 1860-7950}

\title{\LARGE{Ethnographische Forschungsdaten – eine Verantwortung!}}% title
\author{Birgit Kramreither} % author

\setcounter{page}{1}

\hypersetup{%
      pdftitle={Ethnographische Forschungsdaten – eine Verantwortung!},
      pdfauthor={Birgit Kramreither},
      pdfcopyright={CC BY 4.0 International},
      pdfsubject={LIBREAS. Library Ideas, 40 (2021)},
      pdfkeywords={Bibliothek, Dekolonisierung, library, decolonization},
      pdflicenseurl={https://creativecommons.org/licenses/by/4.0/},
      pdfcontacturl={http://libreas.eu},
      baseurl={https://doi.org/10.18452/23810},
      pdflang={de},
      pdfmetalang={de}
     }



\date{}
\begin{document}

\maketitle
\thispagestyle{fancyplain} 

%abstracts
\begin{abstract}
\noindent
Kurzfassung: Die Geschichte des Fachs Kultur- und Sozialanthropologie in
Wien spiegelt sich im Literaturbestand der Fachbereichsbibliothek
Kultur- und Sozialanthropologie wider. Aus heutiger Sicht muss mit
ethnographischem Datenmaterial zum Schutz der Beforschten sehr sorgsam
umgegangen werden. Die Verantwortung darüber liegt nicht zuletzt bei
Repositorien, Archiven und Bibliotheken.
\end{abstract}

%body
\hypertarget{die-fachbereichsbibliothek-kultur--und-sozialanthropologie-der-universituxe4tsbibliothek-wien}{%
\section{Die Fachbereichsbibliothek Kultur- und
Sozialanthropologie der Universitätsbibliothek
Wien}\label{die-fachbereichsbibliothek-kultur--und-sozialanthropologie-der-universituxe4tsbibliothek-wien}}

Die Universitätsbibliothek Wien (UB Wien) als eine der größten ihrer Art
in Europa (Bestand 7,6 Mio. Medien, knapp 500 Mitarbeiter*innen) ist als
Universalbibliothek für alle Fachrichtungen der Universität Wien eine
wichtige Partnerin für Lehre und Forschung.

Die gesamte Literaturversorgung für 190 Studien an 15 Fakultäten und 5
Zentren an der Universität Wien (90.000 Studierende)\footnote{Alle
  Zahlen stammen von der Website der Universität Wien
  \url{https://www.univie.ac.at/} (aufgerufen am 08.07.2021).} wird über
ihre Bestände, ob Print- oder elektronische Versionen, abgedeckt.
Verschiedene Akquisestrategien gewährleisten die Aktualität und
Relevanz. Die Investition in Lizenzen für E-Ressourcen haben gerade im
letzten Jahr gezeigt, wie wichtig der uneingeschränkte Zugang zu
Literatur für Studierende, Lehrende und den wissenschaftlichen Apparat
ist.

Seit ihrem Bestehen (650-Jahr-Jubiläum im Jahr 2015) versucht die UB
Wien mit ihren Beständen den sich im Laufe der Zeit ändernden
wissenschaftlichen Strömungen gerecht zu werden und für die
wissenschaftlichen Diskurse das Rüstzeug zu liefern. Für die Forschung
werden darüber hinaus weitreichende historische Bestände bereitgehalten,
die als Quellen für die Auseinandersetzungen mit der Vergangenheit unter
verschiedensten Fragestellungen konsultiert werden. Dies stellt die
Basis für Erkenntnisse in der heutigen Zeit dar. Daher versteht sich die
UB Wien nicht nur als Lieferantin für Literatur für die rezente
Forschung, sondern auch als Wissensspeicher und Bewahrerin
wissenschaftlicher Fachliteratur über die Jahrhunderte hinweg.

Die Fachbereichsbibliothek Kultur- und Sozialanthropologie (FB KSA) ist
eine von 44 Bibliotheken, die als wissenschaftliche Spezialbibliotheken
im UB-Verband organisiert ist. Sie hat eine eigenständige
Literaturbeschaffung und -bereitstellung für das jeweilige Fach, deren
Studierende und Forschende zu gewährleisten. Räumlich ist sie dem
Institut Kultur- und Sozialanthropologie (IKSA) angeschlossen, personell
und budgetär an die UB Wien angebunden. Insgesamt besitzt die FB KSA
einen Bestand von circa 70.000 Medien, aufgeteilt in einer
Freihandaufstellung mit Lesebereich für knapp 40 Benutzer*innen und
einem geschlossenen Magazin für Zeitschriften (circa 1.800 Titel) und
circa 3.000 historischen Werken\footnote{FB KSA:
  \url{https://bibliothek.univie.ac.at/fb-kultur_sozialanthropologie/}
  (aufgerufen am 08.07.2021).}.

Im Fach der Kultur- und Sozialanthropologie sind die vormals
völkerkundlichen Forschungen zu schriftlosen Kulturen von Anbeginn der
Disziplin im deutschsprachigen Raum über die Achse Wien/Berlin seit Ende
des 19. Jahrhunderts auf universitärer Ebene durchgeführt worden.
Allerdings fanden bereits viel früher Expeditionen, Forschungsreisen und
Missionierungen in ferne Regionen statt.

Berichte über fremde Völker und Kulturen wurden in Büchern und in
ethnologischen Zeitschriften publiziert. Das älteste Werk in der FB KSA
ist ein Spanisch-Nahuatl-Wörterbuch aus dem 16. Jahrhundert.\footnote{Molina,
  Alonso de: Vocabulario en lengua mexicana y castellana, 1571.}

Nicht zuletzt durch die dampfbetriebene Schifffahrt erfolgte eine
Ausdehnung der ethnologischen Forschungen. Die wissenschaftliche Neugier
und der Forschungsdrang führten Ethnograph*innen, Reisende,
Abenteurer*innen und Missionar*innen in die entlegensten Winkel der
Erde, um fremde Länder und deren Menschen zu erkunden. Leider trugen die
Forschungen zur Ausbeutung, Missionierung und Sklaverei der
Kolonialmächte bei.

Ab der Mitte des 19. Jahrhunderts versuchte die junge Wissenschaft der
Ethnologie mit ihren Theorien Abstammungen herzuleiten, kulturelle
Entwicklungen zu erklären und mit ihren Methoden zu untermauern. Das
Beforschen fremder Kulturen stand und steht bis heute im Mittelpunkt der
Forschung.

Der 1913 gegründete Lehrstuhl \enquote{Anthropologie und Ethnologie} an
der Universität Wien wurde 1929 in das Institut für Physische
Anthropologie und in das Institut für Ethnologie geteilt. Verschiedene
Entwicklungstheorien und Forschungsansätze wurden in dieser Zeit
verfolgt, bis die Styler (SDV) Missionare die \enquote{Wiener Schule}
mit der Kulturkreislehre am Beginn des 20. Jahrhunderts am Institut
begründeten. Ab 1938 kam es zu massiven personellen und inhaltlichen
Umbrüchen am Institut. Die Styler Missionare wurden entlassen und gingen
ins Exil in die Schweiz. Einige Mitglieder des IKSA nutzten die
Gelegenheit für Feldforschungen ins Ausland zu gehen und während der
Kriegsjahre dort zu bleiben. Andere dienten an der Front und kamen dann
wieder ans IKSA zurück. Im Gegensatz zu anderen Instituten, wo ab 1938
überwiegend jüdische oder als politisch links geltende
Wissenschaftler*innen vertrieben, verhaftet oder ermordet wurden,
konnten jene Ethnolog*innen, die nach Wien zurückkommen wollten, nach
1945 ihre Forschungen und die Lehre am Institut fortsetzen. Eine Öffnung
hin zu internationalen Entwicklungen im Fach wurde seitens des
Institutvorstandes in der Nachkriegszeit angestrebt und internationale
Kooperationen wurden aufgebaut. Ethnolog*innen aus Europa und Übersee
entwickelten neue ethnosoziologische, kultur- und sozialanthropologische
Theorien, die vom Wiener Institut aufgegriffen und rezipiert wurden.
Eine moderne Fachausrichtung entstand, die sich mit aktuellen Themen wie
Flucht, Migration, Krieg, Ökologie, Visuelle Anthropologie oder Urbane
Anthropologie auseinandersetzt und auch künftig gesellschaftlichen
Phänomenen auf den Grund gehen wird.

Der Literaturbestand in der FB KSA bildet all diese Entwicklungen des
Fachs ab und ermöglicht so eine Auseinandersetzung mit verschiedenen
Theorien und Sichtweisen, die heute neue Erkenntnisse hervorbringt.
Allerdings muss aus Sicht einer angestrebten antirassistischen
Bibliotheksarbeit hinsichtlich rassistischer und diskriminierender
Inhalte und Fotos aufgrund ihrer Entstehungsgeschichte und durch
historische Forschungsinteressen ein kritischer Zugang zu und sensibler
Umgang mit dem Bestand eingefordert werden. Als eine der Maßnahmen wurde
in den letzten Jahren in mehreren Buchausstellungen, kuratiert vom Team
FB KSA, auf die Thematik aufmerksam gemacht.

Für die Benützung der FB KSA kann davon ausgegangen werden, dass sich
der Benützer*innen\-kreis in erster Linie aus Studierenden und Forschenden
des IKSA zusammensetzt. Demnach kann von forschungsgeleiteten Interessen
an der vorhandenen Literatur ausgegangen werden. Trotzdem möchtenwir
Möglichkeiten zur kritischen Auseinandersetzung mit dem Literaturbestand
ausloten und eine aktive Kennzeichnung seitens der Benützer*innen
anbieten. Die genaue Form ist noch nicht festgelegt.

Durch eine seit vielen Jahren praktizierte gezielte Ankaufspolitik
gelang eine diametrale Schwerpunktsetzung zum historischen Bestand. Die
Aufarbeitung der Vergangenheit des Faches, neue kritische Ansätze zu
aktuellen ethnologischen Fragestellungen, Erwerbungen wissenschaftlicher
Fachliteratur von renommierten Verlagen und vermehrter Ankauf von
Fachliteratur aus dem \enquote{Globalen Süden} sind Bestrebungen, die
rezenten wissenschaftlichen Ausrichtungen aufzugreifen und für den
wissenschaftlichen Diskurs zur Verfügung stellen.

Für die gesamte Universitätsbibliothek Wien soll ebenfalls an einer
Bewusstseinsbildung über rassistische und diskriminierende
Literaturbestände gearbeitet werden. Den eigenen Vorurteilen und
Rassismen muss Raum gegeben werden, um sich bewusst damit
auseinandersetzen zu können. Eine klare Positionierung zu Rassismus und
Diskriminierung soll festgeschrieben werden. Seit einiger Zeit wird
bereits an diesen Themen im Einzelnen gearbeitet, ein gemeinsames
Engagement wird angestrebt und verschiedene Möglichkeiten zur internen
Zusammenarbeit werden gerade ausgelotet.

\hypertarget{forschungsdaten-in-der-kultur--und-sozialanthropologie}{%
\section{Forschungsdaten in der Kultur- und
Sozialanthropologie}\label{forschungsdaten-in-der-kultur--und-sozialanthropologie}}

Viele Jahrzehnte lang wurde die KSA als eine Art koloniale
Hilfswissenschaft gesehen und zur Etablierung von Machtstrukturen
benutzt. Die gewonnenen historischen Forschungsdaten müssen im Kontext
von Evolutionismus, Sozialdarwinismus und Rassismus gesehen und
eingeordnet werden. Der koloniale, historische und
wissenschaftstheoretische Hintergrund spielt bei der Verfolgung der
damaligen Ziele eine wichtige Rolle. Die durch die Physische
Anthropologie festgelegte Bewertung der geistigen sowie körperlichen
Fähigkeiten der Menschen und die Ausbeutung, moralisch begründet durch
angebliche rassische Minderwertigkeit, führten zur Ausprägung von
Machtverhältnissen und Herrschaftsstrukturen in den damaligen Kolonien.

Diese furchtbaren Umstände scheinen heute überwunden zu sein, jedoch bei
genauerer Betrachtung lässt sich eine privilegierte Herkunft der
Forschenden mit einem reichen, eurozentristischen Hintergrund
feststellen, die zu einer neuen Machtposition der Forschenden gegenüber
den Beforschten führen kann. Eine kritische Reflexion des eigenen
Verhaltens und die Einhaltung aller ethischen sowie rechtlichen
Standards müssen die Grundlage in der rezenten ethnographischen
Forschung bilden.

Heute ist die Kultur- und Sozialanthropologie eine eigenständige
Wissenschaft, die ihre Methoden wie teilnehmende Beobachtung durch
andere gängige Forschungspraktiken wie Interviews, Feldtagebücher,
Skizzen, Fotos, Filme ergänzt und so zur Untermauerung der qualitativen
Forschungsergebnisse führt. Durch den engen Kontakt zu den beforschten
Gruppen müssen die Feldforscher*innen ganz bewusst ihre eigene Position
kritisch hinterfragen. Emotionale Schwierigkeiten durch eigene
Betroffenheit, Konflikte mit der Gruppe oder sonstige aufwühlende
Ereignisse müssen reflexiv verarbeitet werden und in den
Forschungskontext einfließen. Dadurch erst können der
Forschungszusammenhang klar dargestellt und die Daten für eine
Nachnutzung sinnvoll eingesetzt werden.

Feldforschung spielt in der Kultur- und Sozialanthropologie seit jeher
eine große Rolle. Interviews mit den Beforschten, die in ihrer Sprache
über ihr Leben, ihre Traditionen, ihre Mythen, aber auch über ihre
Probleme -- je nach Forschungsgegenstand -- berichten, bilden meist die
Basis und werden transkribiert und analysiert. Oft kommt es zu
regelmäßigen Aufenthalten der Forschenden bei den Gemeinschaften über
viele Jahre hinweg.

Eine Beforschung über mehrere Generationen führt zu wichtigen
Erkenntnissen zu aktuellen Fragen wie zu Auswirkungen des Klimawandels,
der Trinkwasserknappheit, der Vernichtung des Regenwaldes als Lebensraum
für die indigene Bevölkerung und zu vielen anderen medizinischen,
naturwissenschaftlichen und gesellschaftlichen Aspekten.

Kultur- und Sozialanthropolog*innen leben meist mehrere Monate mit den
beforschten Ethnien in oftmals kleinen Gruppen zusammen. Für die
teilnehmende Beobachtung und für Interviews wird eine
Einverständniserklärung am besten mitgefilmt beziehungsweise
mitgeschnitten. Das ist heute Standard, denn ohne Einverständnis wäre
die Aufnahme unethisch und gegen die aktuell gültigen
Forschungsvorgaben. Genauso ist ein verantwortungsbewusster Umgang mit
den erhobenen Daten Pflicht.

Um die Persönlichkeitsrechte der Beforschten zu gewährleisten, werden
umfangreiche Überlegungen angestellt, wie einerseits das
Forschungsmaterial sicher und für lange Zeit gespeichert werden kann und
andererseits der Schutz der Befragten gegenüber Verfolgung und
Ausgrenzung vor Ort sichergestellt wird. Selbst bei der Vergabe von
Metadaten muss auf \linebreak Anonymisierungs- und Pseudonymisierungsmöglichkeiten
geachtet werden. Aufgrund der kleinen Gruppengröße ist jede Person
leicht zu identifizieren und muss vor Diskriminierung und Verfolgung
geschützt werden.

Innerhalb der beforschten Gemeinschaft kann es zu problematischen
Szenarien kommen, die durch nicht sachgemäßes Agieren mit
personenbezogenen Daten seitens der Forscher*in entstehen können. Die
Träger*innen von speziellem Wissen in einer Gemeinschaft sind schnell
entlarvt, da es nicht viele Personen in der Gemeinschaft gibt, die zum
Beispiel besonderes rituelles Wissen anwenden. Es liegt sofort für alle
auf der Hand, wer diese Person ist. Damit ist eine Anonymisierung
zwecklos und die Person, sollte sie in einem Umfeld agieren, wo
Handlungen dieser Art politisch problematisch sind, ist aufgedeckt.
Aussagen zu politischer Gesinnung, sexueller Orientierung oder
Glaubensfragen können Ausgrenzung, Gewalt, Verfolgung, Gefängnisstrafen
und im schlimmsten Fall Tod nach sich ziehen.

So kommt es beispielsweise in Papua Guinea immer wieder zu gewaltsamen
Übergriffen gegenüber Mitgliedern der indigenen Gruppen, da die
traditionellen Handlungen und Glaubensvorstellungen den
\enquote{Fortschrittsbestrebungen} der Regierung im Wege stehen. Durch
politische Veränderungen im Staat können ehemals bedenkenlose
Forschungen über zum Beispiel Kurd*innen heute an Brisanz gewinnen und
zur potentiellen Gefahr für die Beforschten noch Jahre später werden.
Forschungsdaten über Homosexualität, HIV-Erkrankung und religiöse
Gesinnung würden zu massiven Problemen führen, wenn sie öffentlich
zugänglich gemacht wären.

Ein relativ neuer Aspekt stellt die \enquote{Rückgabe} der
Forschungsdaten an die beforschte Gemeinschaft dar. Die Menschen wissen
um die Wichtigkeit ihrer Datenquellen, um Landnutzungsrechte oder
Besitzrechte nachweisen zu können. So konnte in einigen Fällen in
Nordamerika und Australien anhand von historischen Dokumenten Landbesitz
der indigenen Bevölkerung eindeutig geklärt werden. Die Bereitstellung
von digitalisierten Besitzurkunden und Plänen stellt eine große Hilfe in
diesen Fragestellungen dar. Die Bewohner*innen des \enquote{Globalen
Südens} und anderer Weltregionen brauchen Zugriff auf Archive und
sonstige Wissensspeicher, welcher nur durch weitreichende
Digitalisierungsmaßnahmen gewährleistet werden kann, da ein Reisen oder
Forschen in anderen Ländern oft nicht möglich ist.

Durch das Verstehen der Wichtigkeit der ethnologischen Forschung für die
eigene Gruppe kom\-mt es vermehrt zu Forderungen der Beforschten, Zugang
zu \enquote{ihren} Daten zu erhalten beziehungsweise das
Forschungsdesign und das Forschungsmanagement aktiv mitzubestimmen
(FAIR- und CARE-Prinzipien\footnote{\enquote{CARE Principles for
  Indigenous Data Governance} siehe
  \url{https://www.gida-global.org/care}.}). So verlangen beispielsweise
Gruppen in Alaska, dass die Forschungsdaten zu ihren Lebensumständen,
die durch Arbeitsmigration (Ölproduktion) und tradierte
Wertvorstellungen geprägt sind, nicht archiviert werden dürfen.

\hypertarget{das-ethnographisches-datenarchiv-eda}{%
\section{Das Ethnographisches Datenarchiv
(EDA)}\label{das-ethnographisches-datenarchiv-eda}}

Als Projekt von Wolfgang Kraus, Professor am IKSA 2017, konzipiert,
wurde nach Archivierungsstrategien für das Forschungsmaterial von in
naher Zukunft pensionierter Ethnolog*innen des IKSA und nach
Möglichkeiten der Datenarchivierung aus rezenten Feldforschungen
gesucht. Verschiedene technische wie inhaltliche Überlegungen wurden
getestet, durch Kooperationen weiterentwickelt oder verworfen und nach
drei Jahren Projektphase ein nahezu fertiges Erfassungstool für die
ethnographische Forschung etabliert.\footnote{Nähere Informationen siehe
  Artikel im Literaturanhang.} Igor Eberhard, vormals Mitarbeiter im
Projekt, wurde 2020 fix angestellt und das Ethnographische Datenarchiv
als ein Service der UB Wien dauerhaft verankert. Neben vielen
Teillösungen war die Einführung von Kontextdaten als Ergänzung zu einem
digitalen Objekt und die Etablierung eines komplexen Objekts, das
sogenannte \enquote{Containerobjekt} als digitale Objektkategorie für
die spezifischen Anforderungen an die Erfassung ethnographischer Daten
entwickelt worden.

Im Rahmen von geförderten Feldforschungen werden seitens der Fördergeber
Forschungsdatenmanagementpläne eingefordert, die eine Archivierung
digitaler Daten vorsehen. Spätestens hier müssen sich Forscher*innen
Gedanken zur Digitalisierung und Speicherung ihrer Daten machen und eine
funktionierende Lösung für ihre sensiblen Daten finden. Das
Ethnographische Datenarchiv befasst sich genau mit diesen
Fragestellungen. Durch die intensive Zusammenarbeit mit nationalen wie
internationalen Kooperationspartnern (zum Beispiel
Fachinformationsdienst Kultur- und Sozialanthropologie, Qualiservice der
Universität Bremen, Phonogrammarchiv der Österreichischen Akademie der
Wissenschaften) und einem umfassenden Experimentieren rund um die
verschiedenen Schritte zur Digitalisierung und Archivierung mittels
eigener historischer wie rezenten Forschungsdaten sowie zur technischen
Ausstattung wurden Workflows entwickelt und erprobt. Die Auslotung
sensibler rechtlicher wie ethischer Rahmenbedingungen stellte eine
besondere Herausforderung dar.

Durch die intensive Zusammenarbeit mit Phaidra,\footnote{\url{https://phaidra.univie.ac.at/}
  (aufgerufen am 08.07.2021).} dem Langzeitrepositorium der Universität
Wien, konnten viele Erkenntnisse aus der Projektphase umgesetzt werden
beziehungsweise werden neu auftretende Fragestellungen diskutiert und
Lösungsansätze erprobt. Zurzeit steht zum Beispiel das Fehlen eines für
das Fach geeigneten Thesaurus im Mittelpunkt weiterer Überlegungen. Das
Rechtemanagement ist ebenfalls Gegenstand weiterer Diskussionen und muss
ausgereift werden.

2020 wurde EDA eine fixe Einrichtung an der UB Wien. Personell und
organisatorisch angedockt an die Fachbereichsbibliothek Kultur- und
Sozialanthropologie, stellt EDA eine weitreichende Infrastruktur für die
Erfassung qualitativer Feldforschungsdaten im Langzeitrepositorium
Phaidra zur Verfügung.

Über die eigens entwickelte und der ethnographischen Forschung
angepasste Metadatenmaske werden diverse Schemata zur Erfassung
bereitgestellt. Eine Besonderheit stellt die so wichtige
Kontexteinbettung der ethnologischen Forschung dar. Sie wird durch die
Erfassung von Kontextdaten als eigenes personalisiertes Dokument
ermöglicht. So können beispielsweise Forscher*innenbiographien,
weiterführende Informationen zu Forschungsfragen, Reflexionen der
Forschenden zur persönlichen Feldforschungssituation oder Beschreibungen
der politischen wie rechtlichen Gegebenheiten vor Ort dokumentiert
werden. Durch ihre Kontextualisierung ist die Nachhaltigkeit für eine
weitere Datennutzung gegeben. Die Urheberschaft der Kontextdaten wird
ausgewiesen und ein eigenes Rechtemanagement kann vergeben werden. Die
Kontextdaten werden als eigenständige Objekte in Phaidra hochgeladen und
mit dem hierarchisch übergeordneten Objekt, dem Containerobjekt,
verlinkt.

Darüber hinaus wurde zur Verlinkung der verschiedenen inhaltlich
zusammengehörenden digitalisierten Materialien und für die
Zusammenführung von digitalisiertem Objekt, Metadaten und Kontextdaten
das Containerobjekt als komplexe Objektkategorie vom EDA entwickelt. So
können beispielsweise das Audiodigitalisat eines auf einer Kassette
aufgenommenen Interviews aus einer Feldforschung, ein Foto der Hülle der
Kassette mit wichtigen Informationen zu den handelnden Personen und eine
Translation des Interviews als einzelne Objekte und zugleich gemeinsam
in einem Container gespeichert werden. Die Forscher*in vergibt selbst
die Zugriffsrechte und kann sich auf eine Langzeitarchivierung ihrer
Forschungsdaten seitens der Universität Wien in Phaidra verlassen. Im
Optimalfall werden die Forschenden vom EDA-Team umfassend zur Erfassung
der eignen Forschungsdaten geschult und für die Aufnahme über die
EDA-Eingabemaske autorisiert. Zusätzlich ist eine genaue
Forschungsdokumentation mit Zeit- und Ortsangaben, Namen der agierenden
Personen, forschungsleitende Fragestellungen, Interviewsituationen und
vielem mehr als Basisinformationen für die Datenerfassung notwendig.

Eine Nachnutzung der Daten ist nur mit der allumfassenden Dokumentation
zielführend, daher wird das Forschungsdatenmanagement mit der richtigen
Planung nicht nur für geförderte Forschungsprojekte immer wichtiger. Das
Forschungsdatenmanagement kann die Wissenschaftsproduktion der
Forschenden erhöhen und (bei Einhaltung der Richtlinien) nachhaltig und
fair zugänglich machen.

Im Rahmen von EDA ist die Weiternutzung der Daten in vielen Fällen nur
durch eine Anfrage mit Erklärung des Forschungsinteresses an die
Produzent*innen der Daten möglich. Die vollständige oder eingeschränkte
Freigabe erfolgt durch die Forscher*in selbst. Oft sind schon die
Metadaten ein Problem, wenn daraus ein Tatbestand für Verfolgung oder
Repressalien der beforschten Person abgeleitet werden kann. Hier muss
eine komplette Nichtveröffentlichung (oder eine Sperrfrist) vereinbart
werden.

Eine besondere Form der heutigen Digitalisierungsbestrebungen stellen
Objektdatenbanken dar, die eine Auseinandersetzung mit rituellen
Objekten in den Herkunftsländern erst ermöglichen. Seit Beginn der
Kolonialisierung sind Kultgegenstände, wie Masken, Statuen, Waffen und
Instrumente geraubt oder oft unter ihrem Wert getauscht worden.
Expeditionen sind mit dem Auftrag ausgezogen, Objekte für die neuen
völkerkundlichen Museen in Europa herbeizuschaffen. Zum Beispiel in
Afrika gibt es daher nur wenige rituelle oder Kunstobjekte vor Ort und
der private Kunst- und Sammlermarkt boomt bis heute.

Das Wissen um die Verwendung und den Sinn der Objekte ist in den
besitzenden europäischen Institutionen meist nicht vorhanden; in den
außereuropäischen Ländern fehlen die Kultgegenstände zu dem bis heute
dort tradierten Wissen. Über die Digitalisate in Objektdatenbanken kann
ein kultureller Austausch und ein gegenseitiges Lernen und Kennenlernen
erfolgen. Die Rückgabe besonderer Gegenstände kann ermöglicht oder eine
Kontextualisierung in europäischen Museen und Sammlungen hergestellt
werden. Eine Win-win-Situation für beide Hemisphären.

Die Fachbereichsbibliothek Kultur- und Sozialanthropologie nimmt
einerseits durch die Auseinandersetzung mit ihrem historischen
Buchbestand, den Bewusstseinsbestrebungen hinsichtlich rassistischer und
diskriminierender Merkmale und andererseits mit der äußersten
Sorgfaltspflicht im Umgang mit Forschungsdaten ihre Verantwortung als
Gedächtnis- und Wissensspeicher wahr. Sie versucht es zumindest\ldots{}

\begin{center}\rule{0.5\linewidth}{0.5pt}\end{center}

Für weitere Kontakte:

Ethnographisches Datenarchiv (EDA):
\href{mailto:igor.eberhard@univie.ac.at}{\nolinkurl{igor.eberhard@univie.ac.at}};
\href{mailto:eda.ksa@univie.ac.at}{\nolinkurl{eda.ksa@univie.ac.at}}

Phaidra: \href{mailto:phaidra@univie.ac.at}{\nolinkurl{phaidra@univie.ac.at}}

\hypertarget{literaturempfehlungen}{%
\section{Literaturempfehlungen}\label{literaturempfehlungen}}

Eberhard, Igor: Herausforderung ethnographische Daten: Erfahrungen und
Ergebnisse aus dem Pilotprojekt Ethnographische Datenarchivierung an der
Universität Wien. 2020. In: Künstliche Intelligenz in Bibliotheken : 34.
Österreichischer Bibliothekartag Graz 2019, hrsg. v. Köstner, Christina
und Stadler, Elisabeth und Stumpf, Markus. DOI:
\url{http://dx.doi.org/10.25364/guv.2020.voebs15}

Eberhard, Igor: Der Kontext bestimmt alles. ABI-Technik, 01.05.2020,
Vol. 40 (2), S. 169--176. DOI:
\url{https://doi.org/10.1515/abitech-2020-2007}

Eberhard, Igor: Forschen zwischen Leerstellen und Negativräumen.
Schwierigkeiten und Unmöglichkeiten von Open Science bei
ethnographischem und sozialwissenschaftlichem Forschen In: Mitteilungen
der Vereinigung Österreichischer Bibliothekarinnen \& Bibliothekare,
01.09.2019, Vol. 72 (2), S. 516--523. DOI:
\url{https://doi.org/10.31263/voebm.v72i2.3053}

Eberhard, Igor; Kraus, Wolfgang: Der Elefant im Raum. Ethnographisches
Forschungsdaten-management als Herausforderung für Repositorien. In:
Mitteilungen der Vereinigung Österr. Bibliothekarinnen \& Bibliothekare,
19.07.2018, Vol. 71 (1), S. 41--52. DOI:
\url{https://doi.org/10.31263/voebm.v71i1.2018}

%autor
\begin{center}\rule{0.5\linewidth}{0.5pt}\end{center}

\textbf{Mag. Birgit Kramreither} Leiterin der Fachbereichsbibliothek
Kultur- und Sozialanthropologie, Universitätsbibliothek Wien.
\href{mailto:birgit.kramreither@univie.ac.at}{\nolinkurl{birgit.kramreither@univie.ac.at}}

\end{document}
